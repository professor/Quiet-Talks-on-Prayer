
% Contents


% \begin{enumerate}
% \item The Meaning and Mission of Prayer
%   \begin{enumerate}
%   \item Prayer the Greatest Outlet of Power
%   \item Prayer the Deciding Factor in a Spirit Conflict
%   \item The Earth, the Battle-Field in Prayer
%   \item Does Prayer Influence God?
%   \end{enumerate}
% \item Hindrances to Prayer
%   \begin{enumerate}
%   \item Why the Results Fail
%   \item Why the Results are Delayed
%   \item The Great Outside Hindrance
%   \end{enumerate}
% \item How to Pray
%  \begin{enumerate}
%   \item The "How" of Relationship
%   \item The "How" of Method
%   \item The Listening Side of Prayer
%   \item Something about God's Will in Connection with Prayer
%   \item May We Pray with Assurance for the Conversion of Our Loved Ones
%   \end{enumerate}
% \item Jesus' Habits of Prayer
%   \begin{enumerate}
%   \item A Pen Sketch
%   \item Dissolving Views
%   \item Deepening Shadows
%   \item Under the Olive Trees
%   \item A Composite Picture
%   \end{enumerate}
% \end{enumerate}

% \part{The Meaning and Mission of Prayer}
%   \begin{enumerate}
%   \item Prayer the Greatest Outlet of Power
%   \item Prayer the Deciding Factor in a Spirit Conflict
%   \item The Earth, the Battle-Field in Prayer
%   \item Does Prayer Influence God?
%   \end{enumerate}

\chapter{Prayer the Greatest Outlet of Power}

\underline{Five Outlets of Power.}

A great sorrow has come into the heart of God. Let it be told only in
hushed voice--one of His worlds is \textit{a prodigal}! Hush your voice yet
more-- \textit{ours} is that prodigal world. Let your voice soften down still
more-- \textit{we} have \textit{consented} to the prodigal part of the story. But, in
softest tones yet, He has won some of us back with His strong tender love.
And now let the voice ring out with great gladness--we won ones may be the
pathway back to God for the others. That is His earnest desire. That
should be our dominant ambition. For that purpose He has endowed us with
peculiar power.

There is one inlet of power in the life--anybody's life--any kind of
power: just one inlet--the Holy Spirit. He is power. He is in every one
who opens his door to God. He eagerly enters every open door. He comes in
by our invitation and consent. His presence within is the vital thing.

But with many of us while He is in, He is not in control: in as guest; not
as host. That is to say He is hindered in His natural movements; tied up,
so that He cannot do what He would. And so we are not conscious or only
partially conscious of His presence. And others are still less so. But to
yield to His mastery, to cultivate His friendship, to give Him full
swing--that will result in what is called power. One inlet of power--the
Holy Spirit in control.

There are five outlets of power: five avenues through which this One
within shows Himself, and reveals His power.

First: through the life, what we are. Just simply what we are. If we be
right the power of God will be constantly flowing out, though we be not
conscious of it. It throws the keenest kind of emphasis on a man being
right in his life. There will be an eager desire to serve. Yet we may
constantly do more in what we are than in what we do. We may serve better
in the lives we live than in the best service we ever give. The memory of
that should bring rest to your spirit when a bit tired, and may be
disheartened because tired.

Second: through the lips, what we say. It may be said stammeringly and
falteringly. But if said your best with the desire to please the Master it
will be God-blest. I have heard a man talk. And he stuttered and blushed
and got his grammar badly tangled, but my heart burned as I listened. And
I have heard a man talk with smooth speech, and it rolled off me as easily
as it rolled out of him. Do your best, and leave the rest. If we are in
touch with God His fire burns whether the tongue stammer or has good
control of its powers.

Third: through our service, what we do. It may be done bunglingly and
blunderingly. Your best may not be the best, but if it be your best it
will bring a harvest.

Fourth: through our money, what we do not keep, but loosen out for God.
Money comes the nearest to omnipotence of anything we handle.

And, fifth: through our prayer, what we claim in Jesus' name.

And by all odds the greatest of these is the outlet through prayer. The
power of a life touches just one spot, but the touch is tremendous. What
is there we think to be compared with a pure, unselfish, gently strong
life. Yet its power is limited to one spot where it is being lived. Power
through the lips depends wholly upon the life back of the lips. Words that
come brokenly are often made burning and eloquent by the life behind them.
And words that are smooth and easy, often have all their meaning sapped by
the life back of them. Power through service may be great, and may be
touching many spots, yet it is always less than that of a life. Power
through money depends wholly upon the motive back of the money. Begrudged
money, stained money, soils the treasury. That which comes nearest to
omnipotence also comes nearest to impotence. But the power loosened out
through prayer is as tremendous, at the least, to say no more just now, is
as tremendous as the power of a true fragrant life and, mark you, \textit{and},
may touch not one spot but wherever in the whole round world you may
choose to turn it.

The greatest thing any one can do for God and for man is to pray. It is
not the only thing. But it is the chief thing. A correct balancing of the
possible powers one may exert puts it first. For if a man is to pray
right, he must first \textit{be} right in his motives and life. And if a man \textit{be}
right, and put the practice of praying in its right place, then his
serving and giving and speaking will be fairly fragrant with the presence
of God.

The great people of the earth to-day are the people who pray. I do not
mean those who talk about prayer; nor those who say they believe in
prayer; nor yet those who can explain about prayer; but I mean these
people who \textit{take} time and \textit{pray}. They have not time. It must be taken
from something else. This something else is important. Very important, and
pressing than prayer. There are people that put prayer first, and group
the other items in life's schedule around and after prayer.

These are the people to-day who are doing the most for God; in winning
souls; in solving problems; in awakening churches; in supplying both men
and money for mission posts; in keeping fresh and strong these lives far
off in sacrificial service on the foreign field where the thickest
fighting is going on; in keeping the old earth sweet awhile longer.

It is wholly a secret service. We do not know who these people are, though
sometimes shrewd guesses may be made. I often think that sometimes we pass
some plain-looking woman quietly slipping out of church; gown been turned
two or three times; bonnet fixed over more than once; hands that have not
known much of the softening of gloves; and we hardly giver her a passing
thought, and do not know, nor guess, that perhaps \textit{she} is the one who is
doing far more for her church, and for the world, and for God than a
hundred who would claim more attention and thought, \textit{because she prays};
truly prays as the Spirit of God inspires and guides.

Let me put it this way: God will do as a result of the praying of the
humblest one here what otherwise He \textit{would} not do. Yes, I can make it
stronger than that, and I must make it stronger, for the Book does.
Listen: God will do in answer to the prayer of the weakest one here what
otherwise he \textit{could} not do. "Oh!" someone thinks, "you are getting that
too strong now." Well, you listen to Jesus' own words in that last long
quiet talk He had with the eleven men between the upper room and the
olive-green. John preserves much of that talk for us. Listen: "Ye did not
choose Me, but I chose you, and appointed you, that ye should go and bear
fruit, and that your fruit should abide: that"--listen, a part of the
purpose why we have been chosen--"that whatsoever ye shall ask of the
Father in My name, He \textit{may} give it you."[1] Mark that word "may"; not
"shall" this time but \textit{may}. "Shall" throws the matter over on God--His
purpose. "May" throws it over upon us--our cooperation. That is to say our
praying makes it possible for God to do what otherwise He could not do.

And if you think into it a bit, this fits in with the true conception of
prayer. In its simplest analysis prayer--all prayer--has, must have, two
parts. First, a God to give. "Yes," you say, "certainly, a God wealthy,
willing, all of that." And, just as certainly, there must be a second
factor, \textit{a man to receive}. Man's willingness is God's channel to the
earth. God never crowds nor coerces. Everything God does for man and
through man He does with man's consent, always. With due reverence, but
very plainly, let it be said that God can do nothing for the man with shut
hand and shut life. There must be an open hand and heart and life
\textit{through} which God can give what He longs to. An open life, an open hand,
open upward, is the pipe line of communication between the heart of God
and this poor befooled old world. Our prayer is God's opportunity to get
into the world that would shut Him out.



\underline{In touch with a planet.}


Prayer opens a whole planet to a man's activities. I can as really be
touching hearts for God in far away India or China through prayer, as
though I were there. Not in as many ways as though there, but as truly.
Understand me, I think the highest possible \textit{privilege} of service is in
those far off lands. There the need is greatest, the darkness densest, and
the pleading call most eloquently pathetic. And if one \textit{may} go
there--happy man!--if one be \textit{privileged} to go to the honoured place of
service he may then use all five outlets direct in the spot where he is.

Yet this is only one spot. But his relationship is as wide as his Master's
and his sympathies should be. A man may be in Africa, but if his heart be
in touch with Jesus it will be burning for \textit{a world}. Prayer puts us into
direct dynamic touch with a world.

A man may go aside to-day, and shut his door, and as really spend a
half-hour in India--I am thinking of my words as I say them, it seems so
much to say, and yet it is true--as really spend a half hour of his life
in India for God as though he were there in person. \textit{Is} that true? If it
be true, surely you and I must get more half-hours for this secret
service. Without any doubt he may turn his key and be for a bit of time as
potentially in China by the power of prayer, as though there in actual
bodily form. I say \textit{potentially} present. Of course not consciously
present. But in the \textit{power exerted upon men} he may be truly present at
the objective point of his prayer. He may give a new meaning to the
printed page being read by some native down in Africa. He may give a new
tongue of flame to the preacher or teacher. He may make it easier for men
to accept the story of Jesus, and then to yield themselves to
Jesus--yonder men swept and swayed by evil spirits, and by prejudices for
generations--make it easier for them to accept the story, and, if need be,
to cut with loved ones, and step out and up into a new life.

Some earnest heart enters an objection here, perhaps. You are thinking
that if you were there you could influence men by your personal contact,
by the living voice. So you could. And there must be the personal touch.
Would that there were many times more going for that blessed personal
touch. But this is the thing to mark keenly both for those who may go, and
for those who must stay: no matter where you are you do more through your
praying than through your personality. If you were in India you could \textit{add
your personality to your prayer}. That would be a great thing to do. But
whether there or here, you must first win the victory, every step, every
life, every foot of the way, in secret, in the spirit-realm, and then add
the mighty touch of your personality in service. You can do \textit{more} than
pray, \textit{after} you have prayed. But you can \textit{not} do more than pray \textit{until}
you have prayed. And just there is where we have all seemed to make a
slip at times, and many of us are yet making it--a bad slip. We think we
can do more where we are through our service: then prayer to give power to
service. \textit{No}--with the blackest underscoring of emphasis, let it be
said--NO. We can do no thing of real power until we have done the prayer
thing.

Here is a man by my side. I can talk to him. I can bring my personality to
bear upon him, that I may win him. But before I can influence his will a
jot for God, I must first have won the victory in the secret place.
Intercession is winning the victory over the chief, and service is taking
the field after the chief is driven off. Such service is limited by the
limitation of personality to one place. This spirit-telegraphy called
prayer puts a man into direct dynamic touch with a planet.

There are some of our friends who think themselves of the practical sort
who say, "the great thing is work: prayer is good, and right, but the
great need is to be doing something practical." The truth is that when one
understands about prayer, and puts prayer in its right place in his life,
he finds a new motive power burning in his bones to be \textit{doing}; and
further he finds that it is the doing that grows out of praying that is
mightiest in touching human hearts. And he finds further yet with a great
joy that he may be \textit{doing} something for an entire world. His service
becomes as broad as his Master's thought.



\underline{Intercession is Service.}


It helps greatly to remember that intercession is service: the chief
service of a life on God's plan. It is unlike all other forms of service,
and superior to them in this: that it has fewer limitations. In all other
service we are constantly limited by space, bodily strength, equipment,
material obstacles, difficulties involved in the peculiar differences of
personality. Prayer knows no such limitations. It ignores space. It may be
free of expenditure of bodily strength, where rightly practiced, and one's
powers are under proper control. It goes directly, by the telegraphy of
spirit, into men's hearts, quietly passes through walls, and past locks
unhindered, and comes into most direct touch with the inner heart and will
to be affected.

In service, as ordinarily understood, one is limited to the space where
his body is, the distance his voice can reach, the length of time he can
keep going before he must quit to eat, or rest, or sleep. He is limited by
walls, and locks, by the prejudices of men's minds, and by those peculiar
differences of temperament which must be studied in laying siege to men's
hearts.

The whole circle of endeavour in winning men includes such an infinite
variety. There is speaking the truth to a number of persons, and to one at
a time; the doing of needed kindly acts of helpfulness, supplying food,
and the like; there is teaching; the almost omnipotent ministry of money;
the constant contact with a pure unselfish life; letter writing; printer's
ink in endless variety. All these are in God's plan for winning men. But
the intensely fascinating fact to mark is this:--that the real victory in
all of this service is won in secret, beforehand, by prayer, and these
other indispensable things are the moving upon the works of the enemy, and
claiming the victory already won. And when these things are put in their
proper order, prayer first, and the other things second; \textit{second}, I say,
not omitted, not slurred over; done with all the earnestness and power of
brain and hand and heart possible; but done \textit{after} the victory has been
won in secret, against the real foe, and done \textit{while} the winner is still
claiming the victory already assured,--then will come far greater
achievements in this outer open service.

Then we go into this service with that fine spirit of expectancy that
sweeps the field at the start, and steadily sticks on the stubbornly
contested spots until the whipped foe turns tail, and goes. Prayer is
striking the winning blow at the concealed enemy. Service is gathering up
the results of that blow among the men we see and touch. Great patience
and tact and persistence are needed in the service because each man must
be influenced in his own will. But the shrewd strategy that wins puts the
keen stiff secret fighting first.



\underline{The Spirit Switchboard.}


Electricity is a strange element. It is catalogued in the study of
physics. It is supposed to be properly classed among the forces of nature.
Yet it seems to have many properties of the spirit world. Those who know
most of it say they know least of what it is. Some of the laws of its
being have been learned, and so its marvellous power harnessed for man's
use, but in much ignorance of what it is. It seems almost to belong
somewhere in between the physical and spirit realms. It furnishes many
similes of graphic helpfulness in understanding more nearly much truth of
the Spirit life.

In the power-house where the electricity is being wooed into man's
harnessing, or generated, as the experts say, is found a switchboard, or
switch-room with a number of boards. Here in a large city plant a man may
go and turn a switch, that is, move a little handle, a very short
distance. It is a very simple act, easily performed, involving almost no
strength. But that act has loosened the power in the house back of the
switchboard out along the wires, and perhaps lighted a whole section of
the city. He goes in again at another hour, and turns \textit{this} set of
switches, and \textit{this}, and sets in motion maybe scores of cars, carrying
swiftly, hundreds of passengers. Again he goes in, and moves the little
handles and sets in motion the wheels in some factory employing hundreds
of operatives.

It is a secret service, usually as far as any observers are concerned. It
is a very quiet, matter of fact service. But the power influenced is
unmeasured and immeasurable. And no one, seemingly, thus far, can explain
the mysterious but tremendous agent involved. Does the fluid--it a fluid?
or, what?--pass \textit{through} the wire? or, \textit{around} the wire? The experts say
they do not know. But the laws which it obeys are known. And as men comply
with them its almost omnipotence is manifested.

Just such a switch-room in the spirit realm is one's prayer-room. Every
one who will may have such a spirit switching-board in his life. There he
may go and in compliance with the laws of the power used loosen out the
gracious persuasive irresistible power of God \textit{where he wills to}; now in
Japan; now in China; among the hungry human hearts of India's plains and
mountains; again in Africa which is full as near to where Jesus sits as is
England or America; and now into the house across the alley from your
home; and down in the slum district; and now into your preacher's heart
for next Sunday's work; and now again unto the hearts of those you will be
meeting in the settlement house, or the mission school.

Children are not allowed at the electrical switchboard, nor any unskilled
hand. For misuse means possibility of great damage to property and life.
And the spirit switchboard does not yield to the unskilled touch. Though
sometimes there seems to be much tampering by those with crude fingers,
and with selfish desire to turn this current to personal advantage merely.

It takes skill here. Yet such is our winsome God's wondrous plan that
skill may come to any one who is willing; simply that--who is willing; and
it comes \textit{very simply} too.

Strange too, as with the electrical counterpart, the thing is beyond full
or satisfying explanation.

How does it come to pass that a man turns a few handles, and miles away
great wheels begin to revolve, and enormous power is manifested? Will some
one kindly explain? Yet we know it is so, and men govern their actions by
that knowledge.

How does it come to pass that a woman in Iowa prays for the conversion of
her skeptical husband, and he, down in the thick of the most absorbing
congress Washington has known since the civil war, and in full ignorance
of her purpose becomes conscious and repeatedly conscious of the presence
and power of the God in whose existence he does not believe; and months
afterwards with his keen, legally trained mind, finds the calendar to fit
together the beginning of her praying with the beginning of his unwelcome
consciousness? Will some one kindly explain? Ah! who can, adequately! Yet
the facts, easy ascertainable, are there, and evidenced in the complete
change in the life and calling of the man.

How comes it to pass that a woman in Missouri praying for a friend of keen
intellectual skeptically in Glasgow, who can skillfully measure and parry
argument, yet finds afterwards that the time of her praying is the time of
his, at first decidedly unwelcome, but finally radical change of
convictions! Yet groups of thoughtful men and women know these two
instances to be even so though unable to explain how.

And as the mysterious electrical power is being used by obedience to its
laws, even so is the power of prayer being used by many who understand
simply enough of its laws to obey, and to bring the stupendous results.



\underline{The Broad Inner Horizon.}


This suggests at once that the rightly rounded Christian life has two
sides; the \textit{out}-side, and the \textit{inner} side. To most of us the outer side
seems the greater. The living, the serving, the giving, the doing, the
absorption in life's work, the contact with men, with the great majority
the sheer struggle for existence--these take the greater thought and time
of us all. They seem to be the great business of life even to those of us
who thoroughly believe in the inner life.

But when the real eyes open, the inner eyes that see the unseen, the
change of perspective is first ludicrous, then terrific, then pathetic.
Ludicrous, because of the change of proportions; terrific, because of the
issues at stake; pathetic, because of strong men that see not, and push on
spending splendid strength whittling sticks. The outer side is narrow in
its limits. It has to do with food and clothing, bricks and lumber, time
and the passing hour, the culture of the mind, the joys of social contact,
the smoothing of the way for the suffering. And it needs not to be said,
that these are right; they belong in the picture; they are its physical
background.

The inner side \textit{includes all of these}, and stretches infinitely beyond.
Its limits are broad; broad as the home of man; with its enswathing
atmosphere added. It touches the inner spirit. It moves in upon the
motives, the loves, the heart. It moves out upon the myriad spirit-beings
and forces that swarm ceaselessly about the earth staining and sliming
men's souls and lives. It moves up to the arm of God in cooperation with
His great love-plan for a world.

Shall we follow for a day one who has gotten the true perspective? Here is
the outer side: a humble home, a narrow circle, tending the baby,
patching, sewing, cooking, calling; \textit{or}, measuring dry goods, chopping a
typewriter, checking up a ledger, feeding the swift machinery, endless
stitching, gripping a locomotive lever, pushing the plow, tending the
stock, doing the chores, tiresome examination papers; and all the rest of
the endless, endless, doing, day by day, of the commonplace treadmill
things, that must be done, that fill out the day of the great majority of
human lives. This one whom we are following unseen is doing quietly,
cheerily his daily round, with a bit of sunshine in his face, a light in
his eye, and lightness in his step, and the commonplace place becomes
uncommon by reason of the presence of this man with the uncommon spirit.
He is working for God. No, better, he is working with God. He has an
unseen Friend at his side. That changes all. The common drudgery ceases to
be common, and ceases to be drudgery because it is done for such an
uncommon Master. That is the outer, the narrow side of this life: not
narrow in itself but in its proportion to the whole.

Now, hold your breath, and look, for here is the inner side where the
larger work of life is being done. Here is the quiet bit of time alone
with God, with the Book. The door is shut, as the Master said. Now it is
the morning hour with a bit of made light, for the sun is busy yet farther
east. Now it is the evening hour, with the sun speeding towards western
service, and the bed invitingly near. There is a looking up into God's
face; then keen but reverent reading, and then a simple intelligent
pleading with its many variations of this--"Thy will be done, in the
Victor's name." God Himself is here, in this inner room. The angels are
here. This room opens out into and is in direct touch with a spirit space
as wide as the earth. The horizon of this room is as broad as the globe.
God's presence with this man makes it so.

To-day a half hour is spent in China, for its missionaries, its native
Christians, its millions, the printed page, the personal contact, the
telling of the story, the school, the dispensary, the hospital. And in
through the petitions runs this golden thread--"Victory in Jesus' name:
victory in Jesus' name; to-day: to-day: Thy will be being done: the other
will undone: victory in Jesus' name." Tomorrow's bit of time is largely
spent in India perhaps. And so this man with the narrow outer horizon and
the broad inner horizon pushes his spirit-way through Japan, India,
Ceylon, Persia, Arabia, Turkey, Africa, Europe's papal lands, the South
American States, the home land, its cities, frontiers, slums, the home
town, the home church, the man across the alley; in and out; out and in;
the tide of prayer sweeps quietly, resistlessly day by day.

This is the true Christian life. This man is winning souls and refreshing
lives in these far-off lands and in near-by places as truly as though he
were in each place. This is the Master's plan. The true follower of Jesus
has as broad a horizon as his Master. Jesus thought in continents and
seas. His follower prays in continents and seas. This man does not know
what is being accomplished. Yes! He \textit{does} know, too. He knows by the
inference of faith.

This room where we are meeting and talking together might be shut up so
completely that no light comes in. A single crack breaking somewhere lets
in a thin line of light. But that line of light shining in the darkness
tells of a whole sun of light flooding the outer world.

There comes to this man occasional, yes frequent, evidences of changes
being wrought, yet he knows that these are but the thin line of glory
light which speaks of the fuller shining. And with a spirit touched with
glad awe that he can and may help God, and a heart full alike of peace and
of yearning, and a life fragrant with an unseen Presence he goes steadily
on his way, towards the dawning of the day.




\chapter{Prayer the Deciding Factor in a Spirit Conflict}



\underline{A Prehistoric Conflict.}


In its simplest meaning prayer has to do with a conflict. Rightly
understood it is the deciding factor in a spirit conflict. The scene of
the conflict is the earth. The purpose of the conflict is to decide the
control of the earth, and its inhabitants. The conflict runs back into the
misty ages of the creation time.

The rightful prince of the earth is Jesus, the King's Son. There is a
pretender prince who was once rightful prince. He was guilty of a breach
of trust. But like King Saul, after his rejection and David's anointing in
his place, he has been and is trying his best by dint of force to hold the
realm and oust the rightful ruler.

The rightful Prince is seeking by utterly different means, namely by
persuasion, to win the world back to its first allegiance. He had a fierce
set-to with the pretender, and after a series of victories won the great
victory of the resurrection morning.

There is one peculiarity of this conflict making it different from all
others; namely, a decided victory, and the utter vanquishing of the
leading general has not stopped the war. And the reason is remarkable. The
Victor has a deep love-ambition to win, not merely against the enemy, but
\textit{into men's hearts, by their free consent}. And so, with marvellous
love-born wisdom and courage, the conflict is left open, for men's sake.

It is a spirit conflict. The earth is swung in a spirit atmosphere. There
are unnumbered thousands of spirit beings good and evil, tramping the
earth's surface, and filling its atmosphere. They are splendidly organized
into two compact organizations.

Man is a spirit being; an embodied spirit being. He has a body and a mind.
He is a spirit. His real conflicts are of the spirit sort; in the spirit
realm, with other spirit beings.

Satan is a spirit being; an unembodied spirit being. That is, unembodied,
save as in much cunning, with deep, dark purpose he secures embodiment in
human beings.

The only sort of power that influences in the spirit realm is \textit{moral}
power. By which is not meant \textit{goodness}, but that sort of power either bad
or good which is not of a physical sort: that higher, infinitely higher
and greater power than the mere physical. Moral power is the opposite of
violent or physical power.

God does not use force, violent physical force. There are some exceptions
to this statement. There have been righteous wars, righteous on one side.
Turning to the Bible record, in emergencies, in extreme instances God has
ordered war measures. The nations that Israel was told to remove by the
death of war would have inevitably worn themselves out through their
physical excesses, and disobedience of the laws of life. But a wide view
of the race revealed an emergency which demanded a speedier movement. And
as an exception, for the sake of His plan for the ultimate saving of a
race, and a world, God gave an extermination order. The emergency makes
the exception. There is one circumstance under which the taking of human
life is right, namely, when it can be clearly established that God the
giver and sovereign of life has so directed. But the rule clearly is that
God does not use force.

But note sharply in contrast with this that physical force is one of
Satan's chief weapons. But mark there two intensely interesting facts:
first, he can use it only as he secures man as his ally, and uses it
through him. And, second, in using it he has with great subtlety sought to
shift the sphere of action. He knows that in the sphere of spirit force
pure and simple he is at a disadvantage: indeed, worse yet, he is
defeated. For there is a moral force on the other side greater than any at
his command. The forces of purity and righteousness he simply \textit{can}not
withstand. Jesus is the personification of purity and righteousness. It
was on this moral ground, in this spirit sphere that He won the great
victory. He ran a terrific gauntlet of tests, subtle and fierce, through
those human years, and came out victor with His purity and righteousness
unstained.



\underline{Prayer is Projecting One's Spirit Personality.}


Now prayer is a spirit force, it has to do wholly with spirit beings and
forces. It is an insistent claiming, by a man, an embodied spirit being,
down on the contested earth, that the power of Jesus' victory over the
great evil-spirit chieftain shall extend to particular lives now under his
control. The prayer takes on the characteristic of the man praying. He is
a spirit being. It becomes a spirit force. It is a projecting into the
spirit realm of his spirit personality. Being a spirit force it has
certain qualities or characteristics of unembodied spirit beings. An
unembodied spirit being is not limited by space as we embodied folk are.
It can go as swiftly as we can think. If I want to go to London it will
take at least a week's time to get my body through the intervening space.
But I can think myself into London more quickly than I can say the words,
and be walking down the Strand. Now a spirit being can go as quickly as I
can think.

Further, spirit beings are not limited by material obstructions such as
the walls of this building. When I came in here to-day I came in by this
door. You all came in by these doors. We were obliged to come in either by
doors or windows. But the spirit beings who are here listening to us, and
deeply concerned with our discussion did not bother with the doors. They
came in through the walls, or the roof, if they were above us, or through
the floor here, if they happened to be below this level.

Prayer has these qualities of spirit beings of not being limited by space,
or by material obstacles. Prayer is really projecting my spirit, that is,
my real personality to the spot concerned, and doing business there with
other spirit beings. For example there is a man in a city on the Atlantic
seaboard for whom I pray daily. It makes my praying for him very tangible
and definite to recall that every time I pray my prayer is a spirit force
instantly traversing the space in between him and me, and going without
hindrance through the walls of the house where he is, and influencing the
spirit beings surrounding him, and so influencing his own will.

When it became clear to me some few years ago that my Master would not
have me go yet to those parts of the earth where the need is greatest, a
deep tinge of disappointment came over me. Then as I realized the wisdom
of His sovereignty in service, it came to me anew that I could exert a
positive influence in those lands for Him by prayer. As many others have
done, I marked out a daily schedule of prayer. There are certain ones for
whom I pray by name, at certain intervals. And it gives great simplicity
to my faith, and great gladness to my heart to remember that every time
such prayer is breathed out, my spirit personality is being projected
yonder, and in effect I am standing in Shanghai, and Calcutta and Tokyo in
turn and pleading the power of Jesus' victory over the evil one there, and
on behalf of those faithful ones standing there for God.

It is a fiercely contested conflict. Satan is a trained strategist, and an
obstinate fighter. He refuses to acknowledge defeat until he must. It is
the fight of his life. Strange as it must seem, and perhaps absurd, he
apparently hopes to succeed. If we knew all, it might seem less strange
and absurd, because of the factors on his side. There is surely much down
in the world of the sort which we can fully appreciate to give colour to
his expectations. Prayer is insisting upon Jesus' victory, and the retreat
of the enemy on each particular spot, and heart and problem concerned.

The enemy yields only what he must. He yields only what is taken.
Therefore the ground must be taken step by step. Prayer must be definite.
He yields only when he must. Therefore the prayer must be persistent. He
continually renews his attacks, therefore the ground taken must be \textit{held}
against him in the Victor's name. This helps to understand why prayer
must be persisted in after we have full assurance of the result, and even
after some immediate results have come, or, after the general results have
commenced coming.



\underline{Giving God a Fresh Footing.}


The Victor's best ally in this conflict is the man, who while he remains
down on the battle-field, puts his life in full touch with his
Saviour-Victor, and then incessantly, insistently, believingly claims
\textit{victory in Jesus' name}. He is the one foe among men whom Satan cannot
withstand. He is projecting an irresistible spirit force into the spirit
realm. Satan is obliged to yield. We are so accustomed through history's
long record to seeing victories won through force, physical force, alone,
that it is difficult for us to realize that moral force defeats as the
other never can. Witness the demons in the gospels, and in modern days in
China,[2] clearly against their own set purpose, notwithstanding intensest
struggle on their part obliged to admit defeat, and even to ask favours of
their Conqueror. The records of personal Christian service give
fascinating instances of fierce opposition utterly subdued and individuals
transformed through such influence.

Had we eyes to see spirit beings and spirit conflicts we would constantly
see the enemy's defeat in numberless instances through the persistent
praying of some one allied to Jesus in the spirit of his life. Every time
such a man prays it is a waving of the red-dyed flag of Jesus Christ above
Satan's head in the spirit world. Every such man who freely gives himself
over to God, and gives himself up to prayer is giving God a new spot in
the contested territory on which to erect His banner of victory.

The Japanese struggled for weeks to get a footing on the Port Arthur
peninsula, even after the naval victories had practically rendered Russia
helpless on the seas. It was an unusual spectacle to witness such
difficulty in getting a landing after such victories. But with the bulldog
tenacity that has marked her fighting Japan fought for a footing. Nothing
could be done till a footing was gotten.

Prayer is man giving God a footing on the contested territory of this
earth. The man in full touch of purpose with God praying, insistently
praying--that man is God's footing on the enemy's soil. The man wholly
given over to God gives Him a new sub-headquarters on the battle-field
from which to work out. And the Holy Spirit within that man, on the new
spot, will insist on the enemy's retreat in Jesus the Victor's name. That
is prayer. Shall we not, every one of us, increase God's footing down upon
His prodigal earth!




\chapter{The Earth, the Battle-Field in Prayer}



\underline{Prayer a War Measure.}


This world is God's prodigal son. The heart of God's bleeds over His
prodigal. It has been gone so long, and the home circle is broken. He has
spent all the wealth of His thought on a plan for winning the prodigal
back home. Angels and men have marvelled over that plan, its sweep, its
detail, its strength and wisdom, its tenderness. He needs man for His
plan. He will \textit{use} man. That is true. He will \textit{honour} man in service.
That is true. But these only touch the edge of the truth. The pathway from
God to a human heart is through a human heart. When He came to the great
strategic move in His plan, He Himself came down as a man and made that
move. \textit{He needs man for His plan.}

The greatest agency put into man's hands is prayer. To understand that at
all fully one needs to define prayer. And to define prayer adequately one
must use the language of war. Peace language is not equal to the
situation. The earth is in a state of war. It is being hotly besieged and
so one must use war talk to grasp the facts with which prayer is
concerned. \textit{Prayer from God's side is communication between Himself and
His allies in the enemy's country}. Prayer is not persuading God. It does
not influence God's purpose. It is not winning Him over to our side; never
that. He is far more eager for what we are rightly eager for than we ever
are. What there is of wrong and sin and suffering that pains you, pains
Him far more. He knows more about it. He is more keenly sensitive to it
than the most sensitive one of us. Whatever of heart yearning there may be
that moves you to prayer is from Him. God takes the initiative in all
prayer. It starts with Him. True prayer moves in a circle. It begins in
the heart of God, sweeps down into a human heart upon the earth, so
intersecting the circle of the earth, which is the battle-field of prayer,
and then it goes back again to its starting point, having accomplished its
purpose on the downward swing.



\underline{Three Forms of Prayer.}


Prayer is the word commonly used for all intercourse with God. But it
should be kept in mind that that word covers and includes three forms of
intercourse. All prayer grows up through, and ever continues in three
stages.

The first form of prayer is \textit{communion}. That is simply being on good
terms with God. It involves the blood of the cross as the basis of our
getting and being on good terms. It involves my coming to God through
Jesus. Communion is fellowship with God. Not request for some particular
thing; not asking, but simply enjoying Himself, loving Him, thinking about
Him, how beautiful, and intelligent, and strong and loving and lovable He
is; talking to Him without words. That is the truest worship, thinking how
worthy He is of all the best we can possibly bring to Him, and infinitely
more. It has to do wholly with God and a man being on good terms with each
other. Of necessity it includes confession on my part and forgiveness upon
God's part, for only so can we come into the relation of fellowship.
Adoration, worship belong to this first phase of prayer. Communion is the
basis of all prayer. It is the essential breath of the true Christian
life. It concerns just two, God and myself, yourself. Its influence is
directly subjective. \textit{It affects me.}

The second form of prayer is \textit{petition}. And I am using that word now in
the narrower meaning of asking something for one's self. Petition is
definite request of God for something I need. A man's whole life is
utterly dependent upon the giving hand of God. Everything we need comes
from Him. Our friendships, ability to make money, health, strength in
temptation, and in sorrow, guidance in difficult circumstances, and in all
of life's movements; help of all sorts, financial, bodily, mental,
spiritual--all come from God, and necessitate a constant touch with Him.
There needs to be a constant stream of petition going up, many times
wordless prayer. And there will be a constant return stream of answer and
supply coming down. The door between God and one's own self must be kept
ever open. The knob to be turned is on our side. He opened His side long
ago, and propped it open, and threw the knob away. The whole life hinges
upon this continual intercourse with our wondrous God. This is the second
stage or form of prayer. It concerns just two: God and the man dealing
with God. It is subjective in its influence: \textit{its reach is within}.

The third form of prayer is \textit{intercession}. True prayer never stops with
petition for one's self. It reaches out for others. The very word
intercession implies a reaching out for some one else. It is standing as a
go-between, a mutual friend, between God and some one who is either out of
touch with Him, or is needing special help. Intercession is the climax of
prayer. It is the outward drive of prayer. It is the effective end of
prayer \textit{outward}. Communion and petition are upward and downward.
Intercession rests upon these two as its foundation. Communion and
petition store the life with the power of God; intercession lets it out on
behalf of others. The first two are necessarily for self; this third is
for others. They ally a man fully with God: it makes use of that alliance
for others. Intercession is the full-bloomed plant whose roots and
strength lie back and down in the other two forms. \textit{It} is the form of
prayer that helps God in His great love-plan for winning a planet back to
its true sphere. It will help through these talks to keep this simple
analysis of prayer in mind. For much that will be said will deal chiefly
with this third form, intercession, the outward movement of prayer.



\underline{The Climax of Prayer.}


To God man is first an objective point, and then, without ceasing to be
that, he further becomes a distributing centre. God ever thinks of a man
doubly: first for his own self, and then for his possible use in reaching
others. Communion and petition fix and continue one's relation to God, and
so prepare for the great outreaching form of prayer--intercession. Prayer
must begin in the first two but reaches its climax in the third. Communion
and petition are of necessity self-wide. Intercession is world-wide in its
reach. And all true rounded prayer will ever have all three elements in
it. There must be the touch with God. One's constant needs make constant
petition. But the heart of the true follower has caught the warm contagion
of the heart of God and reaches out hungrily for the world. Intercession
is the climax of prayer.

Much is said of the subjective and objective value of prayer; its
influence upon one's self, and its possible influence upon persons and
events quite outside of one's self. Of necessity the first two sorts of
prayer here named are subjective; they have to do wholly with one's self.
Of equal necessity intercessory prayer is objective; it has to do wholly
with others. There is even here a reflex influence; in the first two
directly subjective; here incidentally reflex. Contact with God while
dealing with Him for another of necessity influences me. But that is the
mere fringe of the garment. The main driving purpose is outward.

Just now in certain circles it seems quite the thing to lay great stress
upon the subjective value of prayer and to whittle down small, or, deny
entirely its value in influencing others. Some who have the popular ear
are quite free with tongue and pen in this direction. From both without
and within distinctly Christian circles their voices come. One wonders if
these friends lay the greater emphasis on the subjective value of prayer
so as to get a good deep breath for their hard drive at the other. Yet the
greater probability is that they honestly believe as they say, but have
failed to grasp the full perspective of the picture. In listening to such
statements one remembers with vivid distinctness that the scriptural
\textit{standpoint} always is this: that things quite outside of one's self, that
in the natural order of prevailing circumstances would not occur, are made
to occur through prayer. Jesus constantly so \textit{assumed}. The first-flush,
commonsense view of successful prayer is that some actual result is
secured through its agency.

It is an utter begging of the question to advance such a theory as a
sufficient explanation of prayer. For prayer in its simplest conception
supposes something changed that is not otherwise reachable. Both from the
scriptural, and from a rugged philosophical standpoint the objective is
the real driving point of all full prayer. The subjective is in order to
the objective, as the final outward climactic reach of God's great
love-plan for a world.



\underline{Six Facts Underlying Prayer.}


It will help greatly to step back and up a bit for a fresh look at certain
facts that underlie prayer. Everything depends on a right point of view.
There may be many view-points, from which to study any subject; but of
necessity any one view-point must take in all the essential facts
concerned. If not, the impression formed will be wrong, and a man will be
misled in his actions. In these talks I make no attempt to prove the
Bible's statements, nor to suggest a common law for their interpretation.
That would be a matter for quite a separate series of talks. It clears the
ground to assume certain things. I am assuming the accuracy of these
scriptural statements. And I am glad to say I have no difficulty in doing
so.

Now there are certain facts constantly stated and assumed in this old
Book. They are clearly stated in its history, they are woven into its
songs, and they underlie all these prophetic writings, from Genesis clear
to the end of John's Patmos visions. Possibly they have been so familiar
and taken for granted so long as to have grown unfamiliar. The very old
may need stating as though very new. Here is a chain of six facts:

First:--The earth is the Lord's and the fullness thereof.[3] His by
creation and by sovereign rule. The Lord sat as King at the flood.[4]

Second:--God gave the dominion of the earth to man. The kingship of its
life, the control and mastery of its forces.[5]

Third:--Man, who held the dominion of the earth in trust from God,
transferred his dominion to somebody else, by an act which was a double
act. He was deceived into doing that act. It was an act of disobedience
and of obedience. Disobedience to God, and obedience to another one, a
prince who was seeking to get the dominion of the earth into his own
hands. That act of the first man did this. The disobedience broke with
God, and transferred the allegiance from God. The obedience to the other
one transferred the allegiance, and through that, the dominion to this
other one.

The fourth fact is this:--The dominion or kingship of this earth so far
as given to man, is now not God's, for He gave it to man. And it is not
man's, for he has transferred it to another. It is in the control of that
magnificent prince whose changed character supplies his name--Satan, the
hater, the enemy. Jesus repeatedly speaks of "the prince"--that is the
ruling one--"of this world."[6] John speaks in his vision-book of a time
coming when "the kingdom (not kingdoms, as in the old version) of the
world is become the kingdom of our Lord, and of His Christ."[7] By clear
inference previous to that time it is somebody's else kingdom than His.
The kingship or rulership of the earth which was given to man is now
Satan's.

The fifth fact:--God was eager to swing the world back to its original
sway: for His own sake, for man's sake, for the earth's sake. You see, we
do not know God's world as it came from His hand. It is a rarely beautiful
world even yet--the stars above, the plant life, the waters, the exquisite
colouring and blending, the combinations of all these--an exquisitely
beautiful world even yet. But it is not the world it was, nor that some
coming day it will be. It has been sadly scarred and changed under its
present ruler. Probably Eve would not recognize in the present world her
early home-earth as it came fresh from the hand of its Maker.

God was eager to swing the old world back to its original control. But to
do so He must get a man, one of the original trustee class through whom He
might swing it back to its first allegiance. It was given to man. It was
swung away by man. It must be swung back by man. And so a Man came, and
while Jesus was perfectly and utterly human, we spell that word Man with a
capital M because He was a man quite distinct from all men. Because He was
more truly human than all other men He is quite apart from other men. This
Man was to head a movement for swinging the world back to its first
allegiance.

The sixth fact is this:--These two, God's Man, and the pretender-prince,
had a combat: the most terrific combat ever waged or witnessed. From the
cruel, malicious cradle attack until Calvary's morning and two days longer
it ran. Through those thirty-three years it continued with a terrificness
and intensity unknown before or since. The master-prince of subtlety and
force did his best and his worst, through those Nazareth years, then into
the wilderness,--and Gethsemane--and Calvary. And that day at three
o'clock and for a bit longer the evil one thought he had won. And there
was great glee up in the headquarters of the prince of this world. They
thought the victory was theirs when God's Man lay in the grave under the
bars of death, within the immediate control of the lord of death. But the
third morning came and the bars of death were snapped like cotton thread.
\textit{Jesus rose a Victor.} For it was not possible that such as \textit{He} could be
held by death's lord. And then Satan knew that he was defeated. Jesus,
God's Man, the King's rightful prince, had gotten the victory.

But, please mark very carefully four sub-facts on Satan's side. First, he
refuses to acknowledge his defeat. Second, he refuses to surrender his
dominion until he must. He yields only what he must and when he must.
Third, he is supported in his ambitions by man. He has man's consent to
his control. The majority of men on the earth to-day, and in every day,
have assented to his control. He has control only through man's consent.
(Satan \textit{can}not get into a man's heart without his consent, and God \textit{will}
not.) And, fourth, he hopes yet to make his possession of the earth
permanent.



\underline{The Victor's Great Plan.}


Now, hold your breath and note, on the side of the Victor-prince, this
unparalleled and unimitated action: He has left the conflict open, and the
defeated chief on the field that He may win not simply against the chief,
but through that victory may win the whole prodigal race back to His
Father's home circle again. But the great pitched battle is yet to come. I
would better say \textit{a} pitched battle, for the greatest one is past. Jesus
rides into the future fight a Victor. Satan will fight his last fight
under the shadow and sting of a defeat. Satan is apparently trying hard to
get a Jesus. That is to say Jesus was God's Man sent down to swing the
world back. Satan is trying his best to get \textit{a man}--one of the original
trustee class, to whom the dominion of the earth was intrusted--a man who
will stand for him even as Jesus stood for God. Indeed a man who will
personify himself even as Jesus was the personification of God, the
express image of His person. When he shall succeed in that the last
desperate crisis will come.

\textit{Now prayer is this: A man}, one of the original trustee class, who
received the earth in trust from God, and who gave its control over to
Satan; a man, \textit{on the earth}, the poor old Satan-stolen, sin-slimed,
sin-cursed, contested earth; a man, on the earth, \textit{with his life in full
touch with the Victor, and sheer out of touch with the pretender-prince,
insistently claiming that Satan shall yield before Jesus'-victory, step by
step, life after life}. Jesus is the victor. Satan knows it, and fears
Him. He must yield before His advance, and he must yield before this man
who stands for Jesus down on the earth. And he \textit{will} yield. Reluctantly,
angrily, as slowly as may be, stubbornly contesting every inch of ground,
his clutches will loosen and he will go before this Jesus-man.

Jesus said "the prince of the world cometh: and he hath nothing in Me."[8]
When you and I say, as we may say, very humbly depending on His grace,
very determinedly in the resolution of our own imperial will, "though the
prince of this world come he shall have nothing in me, no coaling station
however small on the shores of my life," then we shall be in position
where Satan must yield as we claim--victory in the Victor's Name.




\chapter{Does Prayer Influence God?}



\underline{How God Gives.}


Some one may object to all this that the statements of God's word do not
agree with this point of view.

At random memory brings up a few very familiar passages, frequently
quoted. "Call unto Me, and I will answer thee, and will shew thee great
things, and difficult, that thou knowest not."[9] "And call upon Me in the
day of trouble; I will deliver thee and thou shalt glorify Me."[10] "Ask,
and it shall be given you; seek, and ye shall find; knock, and it shall be
opened unto you."[11] Here it seems, as we have for generations been
accustomed to think, that our asking is the thing that influences God to
do. And further, that many times persistent, continued asking is necessary
to induce God to do. And the usual explanation for this need of
persistence is that God is testing our faith, and seeking to make certain
changes in us, before granting our requests. This explanation is without
doubt quite true, \textit{in part}. Yet the thing to mark is that it explains
\textit{only} in part. And when the whole circle of truth is brought into view,
this explanation is found to cover only a small part of the whole.

We seem to learn best about God by analogies. The analogy never brings all
there is to be learned. Yet it seems to be the nearest we can get. From
what we know of ourselves we come to know Him.

Will you notice how men give? Among those who give to benevolent
enterprises there are three sorts of givers, with variations in each.

There is the man who gives because he is influenced by others. If the
right man or committee of men call, and deftly present their pleas,
playing skillfully upon what may appeal to him; his position; his egotism;
the possible advantage to accrue; what men whom he wants to be classed
with are doing, and so on through the wide range that such men are
familiar with; if they persist, by and by he gives. At first he seems
reluctant, but finally gives with more or less grace. That is one sort of
giver.

There is a second sort: the man of truly benevolent heart who is desirous
of giving that he may be of help to other men. He listens attentively when
pleas come to him, and waits only long enough to satisfy himself of the
worth of the cause, and the proper sort of amount to give, and then gives.

There is a third sort, the rarest sort. This second man a stage farther
on, who \textit{takes the initiative}. He looks about him, makes inquiries, and
thinks over the great need in every direction of his fellow men. He
decides where his money may best be used to help; and then himself offers
to give. But his gift may be abused by some who would get his money if
they could, and use it injudiciously, or otherwise than he intends. So he
makes certain conditions which must be met, the purpose of which is to
establish sympathetic relations in some particular with those whom he
would help. An Englishman's heart is strongly moved to get the story of
Jesus to the inland millions of Chinese. He requests the China-Inland
Mission to control the expenditure of almost a million dollars of his
money in such a way as best to secure the object in his heart. An American
gives a large sum to the Young Men's Christian Association of his home
city to be expended as directed. His thought is not to build up this
particular organization, but to benefit large numbers of the young men of
his town who will meet certain conditions which he thinks to be for their
good. He has learned to trust this organization, and so it becomes his
trustee.

Another man feels that if the people of New York City can be given good
reading they can thereby best be helped in life. And so he volunteers
money for a number of libraries throughout that city. And thousands who
yearn to increase their knowledge come into sympathy with him in that one
point through his gift. In all such cases the giver's thought is to
accomplish certain results in those whose purpose in certain directions is
sympathetic with his own.

Any human illustration of God must seem crude. Yet of these three sorts of
givers there is one and only one that begins to suggest how God gives. It
may seem like a very sweeping statement to make, yet I am more and more
disposed to believe it true that \textit{most persons} have unthinkingly thought
of God's answering prayer as the first of these three men give. Many
others have had in mind some such thought as the second suggests. Yet to
state the case even thus definitely is to make it plain that neither of
these ways in any manner illustrate God's giving. The third comes the
nearest to picturing the God who hears and answers prayer. Our God has a
great heart yearning after His poor prodigal world, and after each one in
it. He longs to have the effects of sin removed, and the original image
restored. He takes the initiative. Yet everything that is done for man
must of necessity be through man's will; by his free and glad consent. The
obstacles in the way are not numberless nor insurmountable, but they are
many and they are stubborn. There is a keen, cunning pretender-prince who
is a past-master in the fine art of handling men. There are wills warped
and weakened; consciences blurred; minds the opposite of keen,
sensibilities whose edge has been dulled beyond ordinary hope of being
ever made keen again. Sin has not only stained the life, but warped the
judgment, sapped the will, and blurred the mental vision. And God has a
hard time just because every change must of necessity be through that
sapped and warped will.

Yet the difficulty though great is never complex but very simple. And so
the statement of His purpose is ever exquisitely simple. Listen again:
"Call unto Me, and I will answer thee and shew thee great things and
difficult which thou knowest not." If a man call he has already turned his
face towards God. His will has acted, and acted doubly; away from the
opposite, and \textit{towards} God, a simple step but a tremendous one. The
calling is the point of sympathetic contact with God where their purposes
become the same. The caller is beset by difficulties and longs for
freedom. The God who speaks to him saw the difficulties long ago and
eagerly longed to remove them. Now they have come to agreement. And
through this willing will God eagerly works out His purpose.



\underline{A Very Old Question.}


This leads to a very old question: Does prayer influence God? No question
has been discussed more, or more earnestly. Skeptical men of fine
scientific training have with great positiveness said "no." And Christian
men of scholarly training and strong faith have with equal positiveness
said "yes." Strange to say both have been right. Not right in all their
statements, nor right in all their beliefs, nor right in all their
processes of thinking, but right in their ultimate conclusions as
represented by these short words, "no," and "yes." Prayer does not
influence God. Prayer surely does influence God. It does not influence His
purpose. It does influence His action. Everything that ever has been
prayed for, of course I mean every right thing, God has already purposed
to do. But He does nothing without our consent. He has been hindered in
His purposes by our lack of willingness. When we learn His purposes and
make them our prayers we are giving Him the opportunity to act. It is a
double opportunity: manward and Satanward. We are willing. Our willingness
checkmates Satan's opposition. It opens the path to God and rids it of the
obstacles. And so the road is cleared for the free action already planned.

The further question of nature's laws being sometimes set aside is wholly
a secondary matter. Nature's laws are merely God's habit of action in
handling secondary forces. They involve no purpose of God. His purposes
are regarding moral issues. That the sun shall stay a bit longer than
usual over a certain part of the earth is a mere detail with God. It does
not affect His power for the whole affair is under His finger. It does not
affect His purpose for that as concerning far more serious matters. The
emergencies of earth wrought by sin necessitate just such incidents, that
the great purpose of God for man shall be accomplished.

Emergencies change all habits of action, divine and human. They are the
real test of power. If a man throw down the bundle he is carrying and make
a quick wild dash out into the middle of the street, dropping his hat on
the way, and grasp convulsively for something on the ground when no cause
appears for such action we would quickly conclude that the proper place
for him is an asylum. But if a little toddling child is almost under the
horse's hoofs, or the trolley car, no one thinks of criticising, but
instead admires his courage, and quick action, and breathlessly watches
for the result. Emergencies call for special action. They should control
actions, where they exist. Emergencies explain action, and explain
satisfactorily what nothing else could explain.

\textit{The world is in a great emergency through sin.} Only as that tremendous
fact grips us shall we be men of prayer, and men of action up to the limit
of the need, and to the limit of the possibilities. Only as that intense
fact is kept in mind shall we begin to understand God's actions in
history, and in our personal experiences. The greatest event of earth, the
cross, was an emergency action.

The fact that prayer does not make any change in God's thought or
purpose, reveals His marvellous love in a very tender way.

Suppose I want something very much and \textit{need} as well as want. And I go to
God and ask for it. And suppose He is reluctant about giving: had not
thought about giving me that thing; and rather hesitates. But I am
insistent, and plead and persist and by and by God is impressed with my
earnestness, and sees that I really need the thing, and answers my prayer,
and gives me what I ask. Is not that a loving God so to listen and yield
to my plea? Surely. How many times just such an instance has taken place
between a child and his father, or mother. And the child thinks to
himself, "How loving father is; he has given me the thing I asked for."

But suppose God is thinking about me all the time, and planning, with
love-plans for me, and longing to give me much that He has. Yet in His
wisdom He does not give because I do not know my own need, and have not
opened my hand to receive, yes, and, further yet, likely as not, not
knowing my need I might abuse, or misuse, or fail to use, something given
before I had felt the need of it. And now I come to see and feel that need
and come and ask and He, delighted with the change in me, eagerly gives.
Tell me, is not that a very much more loving God than the other conception
suggests? The truth is \textit{that} is God. Jesus says, "Your Father knoweth
what things ye have need of \textit{before ye ask}." And He is a Father. And
with God the word father means mother too. Then what He \textit{knows} we need He
has \textit{already planned} to give. The great question for me then in praying
for some personal thing is this: Do \textit{I} know what \textit{He} knows I need? Am I
thinking about what He is thinking about for me?

And then remember that God is so much more in His loving planning than the
wisest, most loving father we know. Does a mother think into her child's
needs, the food, and clothing and the extras too, the luxuries? That is
God, only He is more loving and wiser than the best of us. I have
sometimes thought this: that if God were to say to me: "I want to give you
something as a special love-gift; an extra because I love you: what would
you like to have?" Do you know I have thought I would say, "Dear God,
\textit{you} choose. \textit{I} choose what \textit{you} choose." He is thinking about me. He
knows what I am thinking of, and what I would most enjoy, and He is such a
lover-God that He would choose something Just a bit finer than I would
think. I might be thinking of a dollar, but likely as not He is thinking
of a double eagle. I am thinking of blackberries, big, juicy blackberries,
but really I do not know what blackberries are beside the sort He knows
and would choose for me. That is our God. Prayer does not and cannot
change the purpose of such a God. For every right and good thing we might
ask for He has already planned to give us. But prayer does change the
action of God. Because He cannot give against our wills, and our
willingness as expressed by our asking gives Him the opportunity to do as
He has already planned.



\underline{The Greatest Prayer.}


There is a greatest prayer, \textit{the} greatest that can be offered. It is the
substratum of every true prayer. It is the undercurrent in the stream of
all Spirit-breathed prayer. Jesus Himself gives it to us in the only form
of prayer He left for our use. It is small in size, but mighty in power.
Four words--"Thy will be done." Let us draw up our chairs, and \textit{brew} it
over mentally, that its strength and fragrance may come up into our
nostrils, and fill our very beings.

"\textit{Thy}": That is God. On one side, He is wise, with all of the
intellectual strength, and keenness and poised judgment that that word
among men brings to us. On another side, He is strong, with all that that
word can imply of might and power irresistible. On still another side He
is good, pure, holy with the finest thought those words ever suggest to us
in those whom we know best, or in our dreams and visions. Then on a side
remaining, the tender personal side, He is--loving? No, that is quite
inadequate. He is \textit{love}. Its personification is He. Now remember that we
do not know the meaning of those words. Our best definition and thought of
them, even in our dreams, when we let ourselves out, but hang around the
outskirts. The heart of them we do not know. Those words mean infinitely
more than we think. Their meaning is a projection along the lines of our
thought of them, but measurelessly beyond our highest reach.

And then, this God, wise, strong, good, and love, \textit{is kin to us}. We
belong to Him.

    "We are His flock;
    He doth us feed.
    And for His sheep,
    He doth us take."

We are His children by creation, and by a new creation in Jesus Christ. He
is ours, by His own act. That is the "Thy"--a God wise, strong, pure, who
is love, and who is a Father-mother-God, and is \textit{our} God.

"Thy \textit{will}." God's will is His desires, His purposes, that which He
wishes to occur, and that to which He gives His strength that it may
occur. The earth is His creation. Men are His children. Judging from wise
loving parents among men He has given Himself to thinking and studying and
planning for all men, and every man, and for the earth. His plan is the
most wise, pure, loving plan that can be thought of, \textit{and more.} It takes
in the whole sweep of our lives, and every detail of them. Nothing escapes
the love-vigilance of our God. What \textit{can} be so vigilant and keen as love?
Hate, the exact reverse, comes the nearest. It is ever the extremes that
meet. But hate cannot come up to love for keen watchfulness at every
turn. Health, strength, home, loved ones, friendships, money, guidance,
protecting care, the necessities, the extras that love ever thinks of,
service--all these are included in God's loving thought for us. That is
His will. It is modified by the degree of our consent, and further
modified by the circumstances of our lives. Life has become a badly
tangled skein of threads. God with infinite patience and skill is at work
untangling and bringing the best possible out of the tangle. What is
absolutely best is rarely relatively best. That which is best in itself is
usually not best under certain circumstances, with human lives in the
balance. God has fathomless skill, and measureless patience, and a love
utterly beyond both. He is ever working out the best thing possible under
every circumstance. He could oftentimes do more, and do it in much less
time if our human wills were more pliant to His. He can be trusted. And of
course \textit{trust} means \textit{trust in the darkest dark} where you cannot see. And
trust means trust. It does not mean test. Where you trust you do not test.
Where you test you do not trust. Making this our prayer means trusting
God. That is God, and that His will, and that the meaning of our offering
this prayer. "Thy will \textit{be}." A man's will is the man in action, within
the limits of his power. God's will for man is Himself in action, within
the limits of our cooperation. \textit{Be} is a verb, an action-word, in the
passive voice. It takes some form of the verb to be to express the
passive voice of any action-word. It takes the intensest activity of will
to put this passive voice into human action. The greatest strength is
revealed in intelligent yielding. Here the prayer is expressing the utter
willingness of a man that God's will shall be done in him, and through
him. A man never \textit{loses} his will, unless indeed he lose his manhood. But
here he makes that will as strong as it can be made, as a bit of steel,
better like the strong oak, strong enough to sway and bend in the wind.
Then he uses all its strength in becoming passive to a higher will. And
that too when the purpose of that higher will is not clear to his own
limited knowledge and understanding.

"Thy will be \textit{done}." That is, be accomplished, be brought to pass. The
word stands for the action in its perfected, finished state. Thy will be
fully accomplished in its whole sweep and in all its items. It speaks not
only the earnest desire of the heart praying, but the set purpose that
everything in the life is held subject to the doing of this purpose of
God. It means that surrender of purpose that has utterly changed the lives
of the strongest men in order that the purpose of God might be dominant.
It cut off from a great throne earth's greatest jurist, the Hebrew
lawgiver, and led him instead to be allied to a race of slaves. It led
that intellectual giant Jeremiah from an easy enjoyable leadership to
espouse a despised cause and so be himself despised. It led Paul from the
leadership of his generation in a great nation to untold suffering, and to
a block and an ax. It led Jesus the very Son of God, away from a kingship
to a cross. In every generation it has radically changed lives, and
life-ambitions. "Thy will be done" is the great dominant purpose-prayer
that has been the pathway of God in all His great doings among men.

That will is being done everywhere else in God's great world of worlds,
save on the earth and that portion of the spirit world allied to this
earth. Everywhere else there is the perfect music of harmony with God's
will. Here only is heard the harsh discordant note.

With this prayer go two clauses that really particularize and explain it.
They are included in it, and are added to make more clear the full intent.
The first of these clauses gives the sweep of His will in its broadest
outlines. The second touches the opposition to that will both for our
individual lives and for the race and the earth.

The first clause is this, "Thy kingdom come." In both of these short
sentences, "Thy will be done," "Thy kingdom come," the emphatic word is
"Thy." That word is set in sharpest possible contrast here. There is
another kingdom now on the earth. There is another will being done. This
other kingdom must go if God's kingdom is to come. These kingdoms are
antagonistic at every point of contact. They are rivals for the same
allegiance and the same territory. They cannot exist together. Charles II
and Cromwell cannot remain in London together. "Thy kingdom come," of
necessity includes this, "the other kingdom go." "Thy kingdom come" means
likewise "Thy king come," for in the nature of things there cannot be a
kingdom without a king. That means again by the same inference, "the other
prince go," the one who makes pretensions to being rightful heir to the
throne. "Thy will be done" includes by the same inference this:--"the
other will be undone." This is the first great explanatory clause to be
connected with this greatest prayer, "Thy kingdom come." It gives the
sweep of God's will in its broadest outlines.

The second clause included in the prayer, and added to make clear the
swing of action is this--"deliver us from the evil one." These two
sentences, "Thy will be done," and "deliver us from the evil one," are
naturally connected. Each statement includes the other. To have God's will
fully done in us means emancipation from every influence of the evil one,
either direct or indirect, or by hereditary taint. To be delivered from
the evil one means that every thought and plan of God for our lives shall
be fully carried out.

There are the two great wills at work in the world ever clashing in the
action of history and in our individual lives. In many of us, aye, in all
of us, though in greatly varying degree, these two wills constantly clash.
Man is the real battle-field. The pitch of the battle is in his will. God
will not do His will in a man without the man's will consenting. And Satan
cannot. At the root the one thing that works against God's will is the
evil one's will. And on the other hand the one thing that effectively
thwarts Satan's plans is a man wholly given up to God's will.

The greatest prayer then fully expressed, sweeps first the whole field of
action, then touches the heart of the action, and then attacks the
opposition. It is this:--Thy kingdom come: Thy will be done: deliver us
from the evil one. Every true prayer ever offered comes under this simple
comprehensive prayer. It may be offered, it \textit{is} offered with an infinite
variety of detail. It is greatest because of its sweep. It includes all
other petitions, for God's will includes everything for which prayer is
rightly offered. It is greatest in its intensity. It hits the very
bull's-eye of opposition to God.




\part{ Hindrances to Prayer}

% \item Hindrances to Prayer
%   \begin{enumerate}
%   \item Why the Results Fail
%   \item Why the Results are Delayed
%   \item The Great Outside Hindrance
%   \end{enumerate}




\chapter{Why the Results Fail}



\underline{Breaking with God.}


God answers prayer. Prayer is God and man joining hands to secure some
high end. He joins with us through the communication of prayer in
accomplishing certain great results. This is the main drive of prayer. Our
asking and expecting and God's doing jointly bring to pass things that
otherwise would not come to pass. Prayer changes things. This is the great
fact of prayer.

Yet a great many prayers are not answered. Or, to put it more accurately,
a great many prayers fail utterly of accomplishing any results. Probably
it is accurate to say that \textit{thousands} of prayers go up and bring nothing
down. This is certainly true. Let us say it just as bluntly and plainly as
it can be said. As a result many persons are saying: "Well, prayer is not
what you claim for it: we prayed and no answer came: nothing was changed."

From all sorts of circles, and in all sorts of language comes this
statement. Scholarly men who write with wisdom's words, and thoughtless
people whose thinking never even pricks the skin of the subject, and all
sorts of people in between group themselves together here. And they are
right, quite right. The bother is that what they say is not all there is
to be said. There is yet more to be said, that is right too, and that
changes the final conclusion radically. Partial truth is a very mean sort
of lie.

The prayer plan like many another has been much disturbed, and often
broken. And one who would be a partner with God up to the limit of his
power must understand the things that hinder the prayer plan. There are
three sorts of hindrances to prayer. First of all there are things in us
that \textit{break off connection} with God, the source of the changing power.
Then there are certain things in us that \textit{delay, or diminish} the results;
that interfere with the full swing of the prayer plan of operations. And
then there is a great \textit{outside} hindrance to be reckoned upon. To-day we
want to talk together of the first of these, namely, the hindrances that
\textit{break off connections} between God and His human partner.

Here again there is a division into three. There are three things directly
spoken of in the book of God that hinder prayer. One of these is a
familiar thing. What a pity that repugnant things may become so familiar
as no longer to repel. It is this:--\textit{sin} hinders prayer. In Isaiah's
first chapter God Himself speaking says, "When you stretch out your
hands"--the way they prayed, standing with outstretched hands--"I will
shut My eyes; when you make many prayers, I will shut My ears."[12] Why?
What's the difficulty? These outstretched hands are \textit{soiled!} They are
actually holding their sin-soiled hands up into God's face; and He is
compelled to look at the thing most hateful to Him. In the fifty-ninth
chapter of this same book,[13] God Himself is talking again. Listen
"Behold! the \textit{Lord's} hand is not shortened: \textit{His} ear is not heavy."
There is no trouble on the \textit{up} side. God is all right. "But"--listen with
both your ears--"your \textit{iniquities} ... your \textit{sins} ... your \textit{hands} ...
your \textit{fingers} ... your \textit{lips} ... your \textit{tongue} ..." the slime of sin is
oozing over everything! Turn back to that sixty-sixth Psalm[14]--"if I
regard iniquity in my heart the Lord will not hear me." How much more if
the sin of the heart get into the hands or the life! And the fact to put
down plainly in blackest ink once for all is this--\textit{sin hinders prayer}.
There is nothing surprising about this. That we can think the reverse is
the surprising thing. Prayer is transacting business with God. Sin is
\textit{breaking with God}.

Suppose I had a private wire from my apartments here to my home in
Cleveland, and some one should go outside and drag the wire down until it
touches the ground--a good square touch with the ground--the electricians
would call it grounded, could I telegraph over that wire? Almost any child
knows I could not. Suppose some one \textit{cuts} the wire, a good clean cut; the
two ends are apart: not a mile; not a yard; but distinctly apart. Could I
telegraph on that wire? Of course not. Yet I might sit in my room and tick
away by the hour wholly absorbed, and use most beautiful persuasive
language--what is the good? The wire's cut. All my fine pleading goes into
the ground, or the air. Now \textit{sin cuts the wire;} it runs the message into
the ground.

"Well," some one will object, "now you're cutting us all out, are you not?
Are we not all conscious of a sinful something inside here that has to be
fought, and held under all the while?" It certainly seems to be true that
the nearer a man gets to God the more keenly conscious he is of a sinful
tendency within even while having continual victory. But plainly enough
what the Book means here is this:--if I am holding something in my life
that the Master does not like, if I am failing to obey when His voice has
spoken, that to me is sin. It may be wrong in itself. It may \textit{not} be
wrong in itself. It may not be wrong for another. Sometimes it is not the
thing involved but the One involved that makes the issue. If that faithful
quiet inner voice has spoken and I know what the Master would prefer and I
fail to keep in line, that to me is sin. Then prayer is useless; sheer
waste of breath. Aye, worse, it is deceptive. For I am apt to say or
think, "Well, I am not as good as you, or you, but then I am not so bad;
\textit{I pray.}" And the truth is because I have broken with God the
praying--saying words in that form--is utterly worthless.

You see \textit{sin is slapping God in the face}. It may be polished, cultured
sin. Sin seems capable of taking quite a high polish. Or it may be the
common gutter stuff. A man is not concerned about the grain of a club that
strikes him a blow. How can He and I talk together if I have done that,
and stick to it--not even apologized. And of what good is an apology if
the offense is being repeated. And if we cannot talk together of course
working together is out of the question. And prayer is working together
with God. Prayer is \textit{pulling with God} in His plan for a world.

Shall we not put out the thing that is wrong? or put in the thing the
Master wants in? For \textit{Jesus'} sake? Aye for \textit{men's} sake: poor befooled
men's sake who are being kept out and away because God cannot get at them
through us!

Shall we bow and ask forgiveness for our sin, and petty stubbornness that
has been thwarting the Master's love-plan? And yet even while we ask
forgiveness there are lives out yonder warped and dwarfed and worse
because of the hindrance in us; yes, and remaining so as we slip out of
this meeting. May the fact send us out to walk very softly these coming
days.



\underline{A Coaling Station for Satan's Fleet.}


There is a second thing that is plainly spoken of that hinders prayer.
James speaks of it in his letter.[15] "Ye have not because ye \textit{ask}
not"--that explains many parched up lives and churches and unsolved
problems: no pipe lines run up to tap the reservoir, and give God an
opening into the troubled territory. Then he pushes on to say--"Ye ask,
\textit{and receive not}"--ah! there's just the rub; it is evidently an old
story, this thing of not receiving--why? "because ye ask amiss to spend it
\textit{in your pleasures}." That is to say selfish praying; asking for something
just because I want it; want it for myself.

Here is a mother praying for her boy. He is just growing up towards young
manhood; not a Christian boy yet; but a good boy. She is thinking, "I want
\textit{my} boy to be an honour to me; he bears my name; my blood is in his
veins; I don't want my boy to be a prodigal. I want him to be a fine man,
an honour to the family; and if he is a true Christian, he likely will be;
\textit{I wish he were a Christian}." And so she prays, and prays repeatedly and
fervently. God might touch her boy's heart and say, "I want you out here
in India to help win my prodigal world back." \textit{Oh!} she did not mean that!
\textit{Her} boy in far, far off \textit{India!} Oh, no! Not that!! Yes, what \textit{she}
wanted--that was the whole thought--selfishness; the stream turning in to
a dead sea within her own narrow circle; no thought of sympathy with God
in His eager outreach for His poor sin-befooled world. The prayer itself
in its object is perfectly proper, and rightly offered and answered times
without number; but the \textit{motive} wholly, uglily selfish and the
selfishness itself becomes a foothold for Satan and so the purpose of the
prayer is thwarted.

Here is a wife praying that her husband might become a Christian. Perhaps
her thought is: "I wish John \textit{were} a Christian: it would be so good: it
really seems the proper thing: he would go to church with me, and sit in
the pew Sunday morning: I'd like that." Perhaps she thinks: "He would be
careful about swearing; he would quit drinking; and be nicer and gentler
at home." Maybe she thinks: "He would ask a blessing at the meals; that
would be so nice." Maybe she thinks: "We would have family prayers."
\textit{Maybe} that does not occur to her these days. This is what I say: \textit{If}
her thought does not go beyond some such range, of course \textit{you} would say
it is selfish. She is thinking of herself; not of the loving grieved God
against whom her husband is in rebellion; not of the real significance to
the man. God might touch her husband's heart, and then say: "I want you to
help Me win My poor world back." And the change would mean a reduced
income, and a different social position. \textit{Oh!} she had not meant \textit{that!}
Yes--what \textit{she} wanted for herself!

Here is a minister praying for a revival in his church. Maybe he is
thinking; no, not exactly thinking; it is just half thinking itself out in
his sub-consciousness--"I wish we had a good revival in our church;
increased membership; larger attendance; easier finances; may be an extra
hundred or two in my own pocket; increased prestige in the denomination; a
better call or appointment: I wish we might have a revival." Now no true
minister ever talked that way even to himself or deliberately thought it.
To do so would be to see the mean contemptibility of it. But you know how
sly we all are in our underneath scarcely-thought-out thoughts. This is
what I say: \textit{if} that be the sort of thing underneath a man's praying of
course the motive is utterly selfish; a bit of the same thing that brought
Satan his change of name and character.

Please notice that the reason for the prayer not being answered here is
not an arbitrary reluctance upon God's part to do a desirable thing. He
never fails to work whenever He has a half chance as far as it is possible
to work, even through men of faulty conceptions and mixed motives. The
reason lies much deeper. It is this: selfishness gives Satan a footing. It
gives a coaling station for his fleet on the shore of your life. And of
course he does his best to prevent the prayer, or when he cannot wholly
prevent, to spoil the results as far as he can.

Prayer may properly be offered--\textit{will} be properly offered for many wholly
personal things; for physical strength, healing in sickness, about dearly
loved ones, money needed; indeed regarding things that may not be
necessary but only desirable and enjoyable, for ours is a loving God who
would have His dear ones enjoy to the full their lives down here. But the
\textit{motive} determines the propriety of such requests. Where the whole
purpose of one's life is \textit{for Him} these things may be asked for freely as
His gracious Spirit within guides. And there need be no bondage of morbid
introspection, no continual internal rakings. \textit{He knows if the purpose of
the heart is to please Him}.



\underline{The Shortest Way to God.}


A third thing spoken of as hindering prayer is an unforgiving spirit. You
have noticed that Jesus speaks much about prayer and also speaks much
about forgiveness. But have you noticed how, over and over again He
\textit{couples} these two--prayer \textit{and} forgiveness? I used to wonder why. I do
not so much now. Nearly everywhere evidence keeps slipping in of the sore
spots. One may try to keep his lips closed on certain subjects, but it
seems about impossible to keep the ears entirely shut. And continually the
evidence keeps sifting in revealing the thin skin, raw flesh, wounds
never healed over, and some jaggedly open, almost everywhere one goes.
Jesus' continual references reveal how strikingly alike is the oriental
and the occidental; the first and the twentieth centuries.

Run through Matthew alone a moment. Here in the fifth chapter:[16] "If
thou are coming to the altar"--that is approaching God; what we call
prayer--"and rememberest that thy brother hath aught \textit{against thee}"--that
side of it--"leave there thy gift and go thy way, \textit{first} be reconciled,"
and so on. Here comes a man with a lamb to offer. He approaches solemnly,
reverently, towards the altar of God. But as he is coming there flashes
across his mind the face of \textit{that man}, with whom he has had difficulty.
And instantly he can feel his grip tightening on the offering, and his
teeth shutting closer at the quick memory. Jesus says, "If that be so lay
your lamb right down." What! go abruptly away! Why! how the folks around
the temple will talk! "Lay the lamb right down, and go thy way." The
shortest way to God for that man is not the way to the altar, but around
by that man's house. "\textit{First}, be reconciled"--keep your perspective
straight--follow the right order--"\textit{first} be reconciled"--not \textit{second;
"then} come and offer thy gift."

In the sixth chapter[17] He gives the form of prayer which we commonly
call the Lord's prayer. It contains seven petitions. At the close He
stops to emphasize just one of the seven. You remember which one; the one
about forgiveness. In the eighteenth chapter[18] Jesus is talking alone
with the disciples about prayer. Peter seems to remember the previous
remarks about forgiveness in connection with prayer; and he asks a
question. It is never difficult to think of Peter asking a question or
making a few remarks. He says, "Master, how many times \textit{must} I forgive a
man? \textit{Seven} times!" Apparently Peter thinks he is growing in grace. He
can actually \textit{think} now of forgiving a man seven times in succession. But
the Master in effect says, "Peter, you haven't caught the idea.
Forgiveness is not a question of mathematics; not a matter of \textit{keeping
tab} on somebody: not seven times but \textit{seventy times seven.}" And Peter's
eyes bulge open with an incredulous stare--"four hundred and ninety
times!... one man--straightway!!" Apparently the Master is thinking, that
he will lose count, or get tired of counting and conclude that forgiveness
is preferable, or else by practice \textit{breathe in the spirit of
forgiveness--the} thing He meant.

Then as He was so fond of doing Jesus told a story to illustrate His
meaning. A man owed his lord a great debt, twelve millions of dollars;
that is to say practically an \textit{unpayable} amount. By comparison with money
to-day, in the western world, it would be about twelve billions. And he
went to him and asked for time. He said: "I'm short just now; but I mean
to pay; I don't mean to shirk: be easy with me; and I'll pay up the whole
sum in time." And his lord generously forgave him the whole debt. That is
Jesus' picture of God, as He knows Him who knows Him best. Then this
forgiven man went out and found a fellow servant who owed him--how much do
you think? Have you ever thought that Jesus had a keen sense of the
ludicrous? Surely it shows here. He owed him about sixteen dollars and
a-quarter or a-half! And you can almost feel the clutch of this fellow's
fingers on the other's throat as he sternly demands:--"Pay me that thou
owest." And his fellow earnestly replies, "Please be easy with me; I mean
to pay; I'm rather short just now: but I'm not trying to shirk; be easy
with me." Is it possible the words do not sound familiar! But he would
not, but put him in the jail. The last place to pay a debt! That is Jesus'
picture of man as He knows him who knows him best. And in effect He says
what we have been forgiven by God is as an unpayable amount. And what are
not willing to forgive is like sixteen dollars and a fraction by contrast.
What little puny folks some of us are in our thinking and feeling!

"Oh, well," some one says, "you do not know how hard it is to forgive."
You think not? I know this much:--that some persons, and some things you
\textit{can}not forgive of yourself. But I am glad to say that I know this too
that if one allows the Spirit of Jesus to sway the heart He will make you
love persons you \textit{can}not like. No natural affinity or drawing together
through disposition, but a real yearning love in the heart. Jesus' love,
when allowed to come in as freely as He means, fills your heart with pity
for the man who has wounded you. An infinite, tender pity that he has sunk
so low as to be capable of such actions.

But the fact to put down in the sharpest contrast of white and black is
that we must forgive freely, frankly, generously, "\textit{even as God}," if we
are to be in prayer touch with God.

And the reason is not far to find; a double reason, Godward and Satanward.
If prayer be partnership in the highest sense then the same spirit must
animate both partners, the human and the divine, if the largest results
are to come. And since unforgiveness roots itself down in hate Satan has
room for both feet in such a heart with all the leeway in action of such
purchase. That word \textit{unforgiving}! What a group of relatives it has, near
and far! Jealousy, envy, bitterness, the cutting word, the polished shaft
of sarcasm with the poisoned tip, the green eye, the acid saliva--what
kinsfolk these!



\underline{Search Me.}


Sin, selfishness, an unforgiving spirit--what searchlights these words
are! Many a splendid life to-day is an utter cipher in the spirit
atmosphere because of some such hindrance. And God's great love-plan for
His prodigal world is being held back; and lives being lost even where
ultimately souls shall be saved because of the lack of human prayer
partners.

May we not well pray:--Search me, oh God, and know my heart and help me
know it; try me and know my innermost, undermost thoughts and purposes and
ambitions, and help me know them; and see what way there be in me that is
a grief to Thee; and then lead me--and here the prayer may be a purpose as
well as a prayer--lead me out of that way unto \textit{Thy} way, \textit{the} way
everlasting. For Jesus' sake; aye for men's sake, too.




\chapter{Why the Results are Delayed}



\underline{God's Pathway to Human Hearts.}


God touches men through men. The Spirit's path to a human heart is through
another human heart. With reverence be it said, yet with blunt plainness
that in His plan for winning men to their true allegiance God is limited
by the human limitations. That may seem to mean more than it really does.
For our thought of the human is of the scarred, warped, shrivelled
humanity that we know, and great changes come when God's Spirit controls.
But the fact is there, however limited our understanding of it.

God needs man for His plan. That is the fact that stands out strong in
thinking about prayer. God's greatest agency; man's greatest agency, for
defeating the enemy and winning men back is intercession. God is counting
mightily upon that. And He can count most mightily upon the man that
faithfully practices that.

The results He longs for are being held back, and made smaller because so
many of us have not learned how to pray simply and skilfully. We need
training. And God understands that. He Himself will train. But we must be
willing; actively willing. And just there the great bother comes in. A
strong will perfectly yielded to God's will, or perfectly willing to be
yielded, is His mightiest ally in redeeming the world.

Answers to prayer are delayed, or denied, out of kindness, \textit{or}, that more
may be given, \textit{or}, that a far larger purpose may be served. But deeper
down by far than that is this: \textit{God's purposes are being delayed}; delayed
because of our unwillingness to learn how to pray, \textit{or}, our slowness--I
almost said--our stupidity in learning. It is a small matter that my
prayer be answered, or unanswered; not small to me; everything perhaps to
me; but small in proportion. It is a tremendous thing that \textit{God's purpose}
for a world is being held back through my lack. The thought that prayer is
\textit{getting things} from God; chiefly that, is so small, pitiably small, and
yet so common. The true conception understands that prayer is partnership
with God in His planet-sized purposes, and includes the "all things"
beside, as an important detail of the whole.

The real reason for the delay or failure lies simply in the difference
between God's view-point and ours. In our asking either we have not
reached the \textit{wisdom} that asks best, \textit{or}, we have not reached the
\textit{unselfishness} that is willing to sacrifice a good thing, for a better,
or the best; the unselfishness that is willing to sacrifice the smaller
personal desire for the larger thing that affects the lives of many.

We learn best by pictures, and by stories which are pen or word pictures.
This was Jesus' favourite method of teaching. There are in the Bible four
great, striking instances of delayed, or qualified answers to prayer.
There are some others; but these stand out sharply, and perhaps include
the main teachings of all. Probably all the instances of hindered prayer
with which we are familiar will come under one of these. That is to say,
where there are good connections upward as suggested in our last talk,
\textit{and}, excepting those that come under the talk succeeding this, namely,
the great outside hindrance. These four are Moses' request to enter
Canaan; Hannah's prayer for a son; Paul's thorn; and Jesus' prayer in
Gethsemane.

Let us look a bit at these in turn.



\underline{For the Sake of a Nation.}


First is the incident of Moses' ungranted petition. Moses was the leader
of his people. He is one of the giants of the human race from whatever
standpoint considered. His codes are the basis of all English and American
jurisprudence. From his own account of his career, the secret of all his
power as a maker of laws, the organizer of a strangely marvellous nation,
a military general and strategist--the secret of all was in his direct
communication with God. He was peculiarly a man of prayer. Everything was
referred to God, and he declared that everything--laws, organization,
worship, plans--came to him from God. In national emergencies where moral
catastrophe was threatened he petitioned God and the plans were changed in
accordance with his request. He makes personal requests and they are
granted. He was peculiarly a man who dealt directly with God about every
sort of thing, national and personal, simple and complex. The record
commonly credited to him puts prayer as the simple profound explanation of
his stupendous career and achievements. He prayed. God worked along the
line of his prayer. The great things recorded are the result. That is the
simple inferential summary.

Now there is one exception to all this in Moses' life. It stands out the
more strikingly that it is an exception; the one exception of a very long
line. Moses asked repeatedly for one thing. It was not given him. God is
not capricious nor arbitrary. There must be a reason. \textit{There is.} And it
is fairly luminous with light.

Here are the facts. These freed men of Egypt are a hard lot to lead and to
live with. Slow, sensuous, petty, ignorant, narrow, impulsive, strangers
to self-control, critical, exasperating--what an undertaking God had to
make a nation, \textit{the} nation of history, about which centred His deep
reaching, far-seeing love ambition for redeeming a world out of such
stuff! Only paralleled by the church being built upon such men as these
Galilean peasants! What victories these! What a God to do such things!
Only a God could do either and both! What immense patience it required to
shape this people. What patience God has. Moses had learned much of
patience in the desert sands with his sheep; for he had learned much of
God. But the finishing touches were supplied by the grindstone of friction
with the fickle temper of this mob of ex-slaves.

Here are the immediate circumstances. They lacked water. They grew very
thirsty. It was a serious matter in those desert sands with human lives,
and young children, and the stock. No, it was not serious: really a very
small matter, for \textit{God was along}, and the enterprise was of His starting.
It was His affair, all this strange journey. And they knew Him quite well
enough in their brief experience to be expecting something fully equal to
all needs with a margin thrown in. There was that series of stupendous
things before leaving Egypt. There was the Red Sea, and fresh food daily
delivered at every man's tent door, and game, juicy birds, brought down
within arms' reach, yes, and--surely this alone were enough--there was
living, cool water gushing abundantly, gladly out of the very heart of a
flinty rock--if such a thing can be said to have a heart! Oh, yes it was a
very small matter to be lacking anything with such a lavish God along.

\textit{But they forgot.} Their noses were keener than their memories. They had
better stomachs than hearts. The odorous onions of Egypt made more
lasting impressions than this tender, patient, planning God. Yet here
even their stomachs forgot those rock-freed waters. These people must be
kinsfolk of ours. They seem to have some of the same family traits.

Listen: they begin to complain, to criticise. God patiently says nothing
but provides for their needs. But Moses has not yet reached the high level
that later experiences brought him. He is standing to them for God. Yet he
is very un-Godlike. Angrily, with hot word, he \textit{smites} the rock. Once
smiting was God's plan; then the quiet word ever after. How many a time
has the once smitten Rock been smitten again in our impatience! \textit{The
waters came}! Just like God! They were cared for, though He had been
disobeyed and dishonoured. And there are the crowds eagerly drinking with
faces down; and up yonder in the shadow standeth God \textit{grieved}, deeply
grieved at the false picture this immature people had gotten of Him that
day through Moses. Moses' hot tongue and flashing eye made a deep moral
scar upon their minds, that it would take years to remove. Something must
be done for the people's sake. Moses disobeyed God. He dishonoured God.
Yet the waters came, for \textit{they needed water}. And God is ever
tender-hearted. But they must be taught the need of obedience, the evil of
disobedience. Taught it so they never could forget.

Moses was a leader. Leaders may not do as common men. And leaders may not
be dealt with as followers. They stand too high in the air. They affect
too many lives. So God said to Moses:--"You will not go into Canaan. You
may lead them clear up to the line; you may even see over, but you may not
go in." That hurt Moses deep down. It hurt God deeper down, in a heart
more sensitive to hurt than was Moses'. Without doubt it was said with
\textit{reluctance}, for \textit{Moses'} sake. But \textit{it was said}, plainly, irrevocably,
for \textit{their} sakes. Moses' petition was for a reversal of this decision.
Once and again he asked. He wanted to see that wondrous land of God's
choosing. He felt the sting too. The edge of the knife of discipline cut
keenly, and the blood spurted. But God said:--"Do not speak to Me again of
this." The decision was not to be changed. For Moses' sake only He would
gladly have changed, judging by His previous conduct. For the sake of the
nation--aye, for the sake of the prodigal world to be won back through
this nation, the petition might not be granted. That ungranted petition
taught those millions the lesson of obedience, of reverence, as no
command, or smoking mount, or drowning Egyptians had done. It became
common talk in every tent, by every camp-fire of the tented nation. "Moses
disobeyed,--he failed to reverence God;--he cannot enter Canaan."--With
hushed tones, and awed hearts and moved, strangely moved faces it passed
from lip to lip. Some of the women and children wept. They all loved
Moses. They revered him. How gladly they would have had him go over. The
double-sided truth--obedience--disobedience--kept burning in through the
years.

In after years many a Hebrew mother told her baby, eager for a story, of
Moses their great leader; his appearance, deep-set eyes, long beard,
majestic mien, yet infinite tenderness and gentleness, the softness of
strength; his presence with God in the mount, the shining face. And the
baby would listen so quietly, and then the eyes would grow so big and the
hush of spirit come as the mother would repeat softly, "but he could not
come over into the land of promise because \textit{he did not obey God}." And
strong fathers reminded their growing sons. And so it was woven into the
warp and woof of the nation--\textit{obedience, reverent obedience to God}. And
one can well understand Moses looking down from above with grateful heart
that he had been denied for \textit{their} sakes. The unselfishness and wisdom of
later years would not have made the prayer. \textit{The prayer of a man was
denied that a nation might be taught obedience}.



\underline{That More Might be Given and Gotten.}


Now let us look a bit at the second of these, the portrait of Hannah the
Hebrew woman. First the broader lines for perspective. This peculiar
Hebrew nation had two deep dips down morally between Egypt and Babylon;
between the first making, and the final breaking. The national tide ebbed
very low twice, before it finally ran out in the Euphrates Valley. Elijah
stemmed the tide the second time, and saved the day for a later night. The
Hannah story belongs in the first of these ebb-tides; the first bad sag;
the first deep gap.

The giant lawgiver is long gone. His successor, only a less giant than
himself is gone too, and all that generation, and more. The giants gave
way to smaller-sized leaders. Now they are gone also. The mountain peaks
have been lost in the foothills, and these have yielded to dunes, and
levels; mostly levels; dead levels. These mountains must have had long
legs. The foothills are so far away, and are running all to toes. Now the
toes have disappeared.

It is a leaderless people, for the true Leader as originally planned has
been, first ignored, then forgot. The people have no ideals. They grub in
the earth content. There is a deep, hidden-away current of good. But it
needs leadership to bring it to the surface. A leaderless people! This is
the niche of the Hannah story.

The nation was rapidly drifting down to the moral level of the lowest. At
Shiloh the formal worship was kept up, but the very priests were tainted
with the worst impurity. A sort of sleepy, slovenly anarchy prevailed.
Every man did that which was right in his own eyes, with every indication
of a gutter standard. "There was none in the land possessing power of
restraint that might put them to shame in anything." No government; no
dominant spirit. Indeed the actual conditions of Sodom and her sister
cities of the plain existed among the people. This is the setting of the
simple graphic incident of Hannah. One must get the picture clearly in
mind to understand the story.

Up in the hill country of Ephraim there lived a wise-hearted religious
man, a farmer, raising stock, and grain; and fruit, too, likely. He was
earnest but not of the sort to rise above the habit of his time. His farm
was not far from Shiloh, the national place of worship, and he made yearly
trips there with the family. But the woman-degrading curse of Lamech was
over his home. He had two wives. Hannah was the loved one. (No man ever
yet gave his heart to two women.) She was a gentle-spoken, thoughtful
woman, with a deep, earnest spirit. But she had a disappointment which
grew in intensity as it continued. The desire of her heart had been
withheld. She was childless.

Though the thing is not mentioned the whole inference is that she prayed
earnestly and persistently but to her surprise and deep disappointment the
desired answer came not. To make it worse her rival--what a word, for the
other one in the home with her--her rival provoked her sore to make her
fret. And that thing \textit{went on} year after year. That teasing, nagging,
picking of a small nature was her constant prod. What an atmosphere for a
home! Is it any wonder that "she was in bitterness of soul" and "wept
sore"? Her husband tenderly tries to comfort her. But her inner spirit
remains chafed to the quick. And all this goes on for years; the yearning,
the praying, the failure of answer, the biting, bitter atmosphere,--for
\textit{years}. And she wonders why.

Why was it? Step back and up a bit and get the broader view which the
narrow limits of her surroundings, and shall I say, too, though not
critically, of her spirit, shut out from her eyes. Here is what she saw:
her fondest hope unrealized, long praying unanswered, a constant ferment
at home. Here is what she wanted:--\textit{a son}. That is her horizon. Beyond
that her thought does not rise.

Here is what God saw:--a nation--no, much worse--\textit{the} nation, in which
centred His great love-plan for winning His prodigal world, going to
pieces. The messenger to the prodigal was being slyly, subtly seduced by
the prodigal. The saviour-nation was being itself lost. The plan so long
and patiently fostered for saving a world was threatened with utter
disaster.

Here is what He wanted--\textit{a leader}! But there were no leaders. And, worse
yet, there were no men out of whom leaders might be made, no men of
leader-size. And worse yet \textit{there were no women} of the sort to train and
shape a man for leadership. That is the lowest level to which a people
ever gets, aye, ever \textit{can} get. God had to get a woman before He could get
a man. Hannah had in her the making of the woman He needed. God honoured
her by choosing her. But she must be changed before she could be used. And
so there came those years of pruning, and sifting, and discipline. Shall
we spell that word discipline with a final g instead of e--discipling, so
the love of it may be plainer to our near-sightedness? And out of those
years and experiences there came a new woman. A woman with vision
broadened, with spirit mellowed, with strength seasoned, with will so
sinewy supple as to yield to a higher will, to sacrifice the dearest
\textit{personal pleasure} for the world-wide purpose; willing that he who was
her dearest treasure should be the nation's \textit{first}.

Then followed months of prayer while the man was coming. Samuel was born,
no, farther back yet, was conceived in the atmosphere of prayer and
devotion to God. The prenatal influences for those months gave the sort of
man God wanted. And a nation, \textit{the} nation, the \textit{world-plan,} was saved!
This man became a living answer to prayer. The romantic story of the
little boy up in the Shiloh tabernacle quickly spread over the nation. His
very name--Samuel, God hears--sifted into people's ears the facts of a
God, and of the power of prayer. The very sight of the boy and of the man
clear to the end kept deepening the brain impression through eyeballs that
God answers prayer. And the seeds of that re-belief in God that Samuel's
leadership brought about were sown by the unusual story of his birth.

\textit{The answer was delayed that more might be given and gotten}. And Hannah's
exultant song of praise reveals the fineness to which the texture of her
nature had been spun. And it tells too how grateful she was for a God who
in great patience and of strong deliberate purpose delayed the answer to
her prayer.



\underline{The Best Light for Studying a Thorn.}


The third great picture in this group is that of Paul and his
needle-pointed thorn. Talks about the certainty of prayer being answered
are very apt to bring this question: "What about Paul's thorn?" Sometimes
asked by earnest hearts puzzled; \textit{some}times with a look in the eye almost
exultant as though of gladness for that thorn because it seems to help out
a theory. These pictures are put into the gallery for our help. Let us
pull up our chairs in front of this one and see what points we may get to
help our hearts.

First a look at Paul himself. The best light on this thorn is through the
man. The man explains the thorn. We have a halo about Paul's head; and
rightly, too. What a splendid man of God he was! God's chosen one for a
peculiar ministry. One of the twelve could be used to open the door to the
great outside world, but God had to go aside from this circle and get a
man of different training for this wider sphere. Cradled and schooled in a
Jewish atmosphere, he never lost the Jew standpoint, yet the training of
his home surroundings in that outside world, the contact with Greek
culture, his natural mental cast fitted him peculiarly for his appointed
task to the great outside majority. His keen reasoning powers, his vivid
imagination, his steel-like will, his burning devotion, his unmovable
purpose, his tender attachment to his Lord,--what a man! Well might the
Master want to win such a man for service' sake. But Paul had some weak
traits. Let us say it very softly, remembering as we instinctively will,
that where we think of one in him there come crowding to memory's door
many more in one's self. A man's weak point is usually the extreme
opposite swing of the pendulum on his strong point. Paul had a tremendous
will. He was a giant, a Hercules in his will. Those tireless journeys with
their terrific experiences, all spell out \textit{will} large and black. But,
gently now, he went to extremes here. Was it due to his overtired nerves?
Likely enough. He was obstinate, \textit{sometimes;} stubborn; set in his way:
\textit{sometimes} head down, jaw locked, driving hard. Say it all \textit{softly}, for
we are speaking of dear old saintly Paul; but, to help, \textit{say} it, for it
is true.

God had a hard time holding Paul to \textit{His} plans. Paul had some of his own.
We can all easily understand that. Take a side glance or two as he is
pushing eagerly, splendidly on. Turn to that sixteenth chapter of
Acts,[19] and listen: "Having been forbidden of the Holy Spirit to speak
the word in (the province of) Asia," coupled with the fact of sickness
being allowed to overtake him in Galatia where the "forbidding" message
came. And again this, "they assayed to go into Bithynia; and the Spirit of
Jesus suffered them not."[20] Tell me, is this the way the Spirit of God
leads? That I should go driving ahead until He must pull me up with a
sharp turn, and twist me around! It is the way He is obliged to do many
times, no doubt, with most of us. But His chosen way? His own way? Surely
not. Rather this, the keeping close, and quiet and listening for the next
step. Rather the "I go not up yet unto this feast" of Jesus.[21] And then
in a few days going up, evidently when the clear intimation came. These
words, "assayed to go," "forbidden," "suffered not"--what flashlights they
let into this strong man's character.

But there is much stronger evidence yet. Paul had an ambition to preach to
the \textit{Jerusalem Jews}. It burned in his bones from the early hours of his
new life. The substratum of "\textit{Jerusalem}" seemed ever in his thoughts and
dreams. If \textit{he} could just get to those Jerusalem Jews! He knew them. He
had trained with them. He was a leader among the younger set. When they
burned against these Christians he burned just a bit hotter. They knew
him. They trusted him to drive the opposite wedge. If only \textit{he} could have
a chance down there he felt that the tide might be turned. But from that
critical hour on the Damascene road "\textit{Gentiles--Gentiles}" had been
sounded in his ears. And he obeyed, of course he obeyed, with all his
ardent heart. \textit{But, but}--those \textit{Jerusalem Jews}! If he might go to
Jerusalem! Yet very early the Master had proscribed the Jerusalem service
for Paul. He made it a matter of a special vision,[22] in the holy temple,
kindly explaining why. "They will not receive of \textit{thee} testimony
concerning Me." Would that not seem quite sufficient? Surely. Yet this
astonishing thing occurs:--Paul attempts to argue with the Master \textit{why} he
should be allowed to go. This is going to great lengths; a subordinate
arguing with his commanding general after the orders have been issued! The
Master closes the vision with a peremptory word of command, "\textit{depart}. I
will send thee \textit{far hence} (from Jerusalem, where you long to be), to the
Gentiles." That is a picture of this man. It reveals the weak side in
this giant of strength and of love. And \textit{this} is the man God has to use
in His plan. He is without doubt the best man available. And in his
splendour he stands head and shoulders above his generation and many
generations. Yet (with much reverence) God has a hard time getting Paul to
work always along the line of \textit{His} plans.

That is the man. Now for the thorn. Something came into Paul's life that
was a constant irritation. He calls it a thorn. What a graphic word! A
sharp point prodding into his flesh, ever prodding, sticking, sticking in;
asleep, awake, stitching tent canvas, preaching, writing, that thing ever
cutting its point into his sensitive flesh. Ugh! It did not disturb him so
much at first, because \textit{there was God} to go to. He went to God and said,
"\textit{Please} take this away." But it stayed and stuck. A second time the
prayer; a bit more urgent; the thing sticks so. The time test is the
hardest test of all. Still no change. Then praying the third time with
what earnestness one can well imagine.

Now note three things: First, \textit{There was an answer}. God answered \textit{the
man}. Though He did not grant the petition, He answered the man. He did
not ignore him nor his request. Then God told Paul frankly that it was not
best to take the thorn away. It was in the lonely vigil of a sleepless
night, likely as not, that the wondrous Jesus-Spirit drew near to Paul.
Inaudibly to outer ear but very plainly to his inner ear, He spoke in
tones modulated into tender softness as of dearest friend talking with
dear friend. "Paul," the voice said, "I know about that thorn--and how it
hurts--it hurts Me, too. For \textit{your} sake, I would quickly, so quickly
remove it. But--Paul"--and the voice becomes still softer--"it is a bit
better for \textit{others}' sake that it remain: the plan in My heart \textit{through
you} for thousands, yes, unnumbered thousands, Paul, can so best be worked
out." That was the first part of what He said. And Paul lies thinking with
a deep tinge of awe over his spirit. Then after a bit in yet quieter voice
He went on to say, "I will be so close to your side; you shall have such
revelations of My glory that the pain will be clear overlapped, Paul; the
glory shall outstrip the eating thorn point."

I can see old Paul one night in his own hired house in Rome. It is late,
after a busy day; the auditors have all gone. He is sitting on an old
bench, slowing down before seeking sleep. One arm is around Luke, dear
faithful Doctor Luke, and the other around young Timothy, not quite so
young now. And with eyes that glisten, and utterance tremulous with
emotion he is just saying:--"And dear old friends, do you know, I would
not have missed this thorn, for the wondrous glory"--and his heart gets
into his voice, there is a touch of the hoarseness of deep emotion, and a
quavering of tone, so he waits a moment--"the wondrous \textit{glory-presence of
Jesus} that came with it."

And so out of the experience came a double blessing. There was a much
fuller working of God's plan for His poor befooled world. And there was an
unspeakable nearness of intimacy with his Lord for Paul. \textit{The man was
answered and the petition denied that the larger plan of service might be
carried out}.



\underline{Shaping a Prayer on the Anvil of the Knees.}


The last of these pictures is like Raphael's Sistine Madonna in the
Dresden gallery; it is in a room by itself. One enters with a holy hush
over his spirit, and, with awe in his eyes, looks at \textit{Jesus in
Gethsemane}. There is the Kidron brook, the gentle rise of ground, the
grove of gnarled knotty old olive trees. The moon above is at the full.
Its brightness makes these shadowed recesses the darker; blackly dark.
Here is a group of men lying on the ground apparently asleep. Over yonder
deeper in among the trees a smaller group reclines motionless. They, too,
sleep. And, look, farther in yet is that lone figure; all alone; nevermore
alone; save once--on the morrow.

There is a foreshadowing of this Gethsemane experience in the requested
interview of the Greeks just a few intense days before. In the vision
which the Greeks unconsciously brought the agony of the olive grove began.
The climax is among these moon-shadowed trees. How sympathetic those inky
black shadows! It takes bright light to make black shadows. Yet they were
not black enough. Intense men can get so absorbed in the shadows as to
forget the light.

This great Jesus! Son of God: God the Son. The Son of Man: God--a man! No
draughtsman's pencil ever drew the line between His divinity and humanity;
nor ever shall. For the union of divine and human is itself divine, and
therefore clear beyond human ken. Here His humanity stands out,
pathetically, luminously stands out. Let us speak of it very softly and
think with the touch of awe deepening for this is holiest ground. The
battle of the morrow is being fought out here. Calvary is in Gethsemane.
The victory of the hill is won in the grove.

It is sheer impossible for man with sin grained into his fibre through
centuries to understand the horror with which a sinless one thinks of
actual contact with sin. As Jesus enters the grove that night it comes in
upon His spirit with terrific intensity that He is actually coming into
contact--with a meaning quite beyond us--coming into contact with sin. In
some way all too deep for definition He is to be "made sin."[23] The
language used to describe His emotions is so strong that no adequate
English words seem available for its full expression. An indescribable
horror, a chill of terror, a frenzy of fright seizes Him. The poisonous
miasma of sin seems to be filling His nostrils and to be stifling Him. And
yonder alone among the trees the agony is upon Him. The extreme grips Him.
May there not yet possibly be some other way rather than \textit{this--this!} A
bit of that prayer comes to us in tones strangely altered by deepest
emotion. "\textit{If it be possible--let this cup pass}." There is still a
clinging to a possibility, some possibility other than that of this
nightmare vision. The writer of the Hebrews lets in light here. The strain
of spirit almost snaps the life-thread. And a parenthetical prayer for
strength goes up. And the angels come with sympathetic strengthening. With
what awe must they have ministered! Even after that some of the red life
slips out there under the trees. By and by a calmer mood asserts itself,
and out of the darkness a second petition comes. It tells of the tide's
turning, and the victory full and complete. \textit{A changed, petition} this!
"\textit{Since this cup may not pass}--since only thus \textit{can} Thy great plan for a
world be wrought out--\textit{Thy--will}"--slowly but very distinctly the words
come--"\textit{Thy--will--be--done.}"

\textit{The changed prayer was wrought out upon His knees!} With greatest
reverence, and a hush in our voices, let us say that there alone with the
Father came the clearer understanding of the Father's actual will under
these circumstances.

    "Into the woods my Master went
    Clean forspent, forspent;
    Into the woods my Master came
    Forspent with love and shame.
    But the olives they were not blind to Him,
    The little gray leaves were kind to Him;
    The thorn-tree had a mind to Him
    When into the woods He came.

    "Out of the woods my Master went
    And He was well content;
    Out of the woods my Master came
    Content with death and shame.
    When death and shame would woo Him last
    From under the trees they drew Him last
    'Twas on a tree they slew Him--last
    When out of the woods He came."[24]

True prayer is wrought out upon the knees alone with God. With deepest
reverence, and in awed tones, let it be said, that \textit{that was true of
Jesus} in the days of His humanity. How infinitely more of us!

Shall we not plan to meet God alone, habitually, with the door shut, and
the Book open, and the will pliant so we may be trained for this holy
partnership of prayer. Then will come the clearer vision, the broader
purpose, the truer wisdom, the real unselfishness, the simplicity of
claiming and expecting, the delights of fellowship in service with Him;
then too will come great victories for God in His world. Although we
shall not begin to know by direct knowledge a tithe of the story until the
night be gone and the dawning break and the ink-black shadows that now
stain the earth shall be chased away by the brightness of His presence.




\chapter{The Great Outside Hindrance}



\underline{The Traitor Prince.}


There remains yet a word to be said about hindrances. It is a most
important word; indeed the climactic word. What has been said is simply
clearing the way for what is yet to be said. A very strange phase of
prayer must be considered here. Strange only because not familiar. Yet
though strange it contains the whole heart of the question. Here lies the
fight of the fight. One marvels that so little is said of it. For if there
were clear understanding here, and then faithful practicing, there would
be mightier defeats and victories: defeats for the foe; victories for our
rightful prince, Jesus.

The intense fact is this: \textit{Satan has the power to hold the answer
back--for awhile; to delay the result--for a time}. He has not the power
to hold it back finally, \textit{if} some one understands and prays with quiet,
steady persistence. The real pitch of prayer therefore is Satanward.

Our generation has pretty much left this individual Satan out. It is
partly excusable perhaps. The conceptions of Satan and his hosts and
surroundings made classical by such as Dante and Milton and Doré have
done much to befog the air. Almost universally they have been taken
literally whether so meant or not. One familiar with Satan's
characteristics can easily imagine his cunning finger in that. He is
willing even to be caricatured, or to be left out of reckoning, if so he
may tighten his grip.

These suggestions of horns and hoofs, of forked tail and all the rest of
it seek to give material form to this being. They are grotesque to an
extreme, and therefore caricatures. A caricature so disproportions and
exaggerates as to make hideous or ridiculous. In our day when every
foundation of knowledge is being examined there has been a natural but
unthinking turning away from the very being of Satan through these
representations of him. Yet where there is a caricature there must be a
true. To revolt from the true, hidden by a caricature, in revolting from
the caricature is easy, but is certainly bad. It is always bad to have the
truth hid from our eyes.

It is refreshing and fascinating to turn from these classical caricatures
to the scriptural conception of Satan. In this Book he is a being of great
beauty of person, of great dignity of position even yet, endowed with most
remarkable intellectual powers, a prince, at the head of a most
remarkable, compact organization which he has wielded with phenomenal
skill and success in furthering his ambitious purposes.

And he is not chained yet. I remember a conversation with a young
clergyman one Monday morning in the reading-room of a Young Men's
Christian Association. It was in a certain mining town in the southwest,
which is as full of evil resorts as such places usually are. The day
before, Sunday, had been one of special services, and we had both been
busy and were a bit weary. We were slowing down and chatting leisurely. I
remarked to my friend, "What a glad day it will be when the millennium
comes!" He quickly replied, "I think this is the millennium." "But," I
said, "I thought Satan was to be chained during that time. Doesn't it say
something of that sort in the Book?" "Yes," he replied, "it does. But I
think he is chained now." And I could not resist the answer that came
blurting its way out, "Well, if he is chained, he must have a fairly long
chain: it seems to permit much freedom of action." From all that can be
gathered regarding this mighty prince he is not chained yet. We would do
well to learn more about him. The old military maxim, "Study the enemy,"
should be followed more closely here.

It is striking that the oldest of the Bible books, and the latest, Job and
Revelation, the first word and the last, give such definite information
concerning him. These coupled with the gospel records supply most of the
information available though not all. Those three and a half years of
Jesus' public work is the period of greatest Satanic and demoniac
activity of which any record has been made. Jesus' own allusions to him
are frequent and in unmistakable language. There are four particular
passages to which I want to turn your attention now. Let it not be
supposed, however, that this phase of prayer rests upon a few isolated
passages. Such a serious truth does not hinge upon selected proof texts.
It is woven into the very texture of this Book throughout.

There are two facts that run through the Bible from one end to the other.
They are like two threads ever crossing in the warp and woof of a finely
woven fabric. Anywhere you run your shears into the web of this Book you
will find these two threads. They run crosswise and are woven inextricably
in. One is a black thread, inky black, pot-black. The other is a bright
thread, like a bit of glory light streaming across. These two threads
everywhere. The one is this--the black thread--there is an enemy. Turn
where you will from Genesis to Revelation--always an enemy. He is keen. He
is subtle. He is malicious. He is cruel. He is obstinate. He is a master.
The second thread is this: the leaders for God have always been men of
prayer above everything else. They are men of power in other ways,
preachers, men of action, with power to sway others but above all else men
of prayer. They give prayer first place. There is one striking exception
to this, namely, King Saul. And most significantly a study of this
exception throws a brilliant lime light upon the career of Satan. King
Sauls seems to furnish the one great human illustration in scripture of
heaven's renegade fallen prince. These special paragraphs to be quoted are
like the pattern in the cloth where the colours of the yarn come into more
definite shape. The gospels form the central pattern of the whole where
the colours pile up into sharpest contrast.



\underline{Praying is Fighting.}


But let us turn to the Book at once. For we \textit{know} only what it tells. The
rest is surmise. The only authoritative statements about Satan seem to be
these here. Turn first to the New Testament.

The Old Testament is the book of illustrations; the New of explanations,
of teaching. In the Old, teaching is largely by kindergarten methods. The
best methods, for the world was in its child stage. In the New the
teaching is by precept. There is precept teaching in the Old; very much.
There is picture teaching in the New; the gospels full of it. But picture
teaching, acted teaching, is the characteristic of the Old, and precept
teaching of the New. There is a wonderfully vivid picture in the Old
Testament, of this thing we are discussing. But first let us get the
teaching counterpart in the new, and then look at the picture.

Turn to Ephesians. Ephesians is a prayer epistle. That is a very
significant fact to mark. Of Paul's thirteen letters Ephesians is
peculiarly the prayer letter. Paul is clearly in a prayer mood. He is on
his knees here. He has much to say to these people whom he has won to
Christ, but it comes in the parenthesis of his prayer. The connecting
phrase running through is--"for this cause I pray.... I bow my knees."
Halfway through this rare old man's mind runs out to the condition of
these churches, and he puts in the always needed practical injunctions
about their daily lives. Then the prayer mood reasserts itself, and the
epistle finds its climax in a remarkable paragraph on prayer. From praying
the man goes urging them to pray.

We must keep the book open here as we talk: chapter six, verses ten to
twenty inclusive. The main drive of all their living and warfare seems
very clear to this scarred veteran:--"that ye may be able to withstand the
wiles of the devil." This man seems to have had no difficulty in believing
in a personal devil. Probably he had had too many close encounters for
that. To Paul Satan is a cunning strategist requiring every bit of
available resource to combat.

This paragraph states two things:--who the real foe is, against whom the
fight is directed; and, then with climactic intensity it pitches on the
main thing that routs him. Who is the real foe? Listen:--"For our
wrestling is not against flesh and blood"--not against men; never that;
something far, subtler--"but against the principalities"--a word for a
compact organization of individuals,--"against powers"--not only organized
but highly endowed intellectually, "against the world-rulers of this
darkness,"--they are of princely kin; not common folk--"against the hosts
of wicked spirits in the heavenlies"--spirit beings, in vast numbers,
having their headquarters somewhere above the earth. \textit{That} is the foe.
Large numbers of highly endowed spirit beings, compactly organized, who
are the sovereigns of the present realm or age of moral darkness, having
their \textit{headquarters} of activity somewhere above the earth, and below the
throne of God, but concerned with human beings upon the earth. In chapter
two of the epistle the head or ruler of this organization is referred to,
"the prince of the powers of the air."[25] That is the real foe.

Then in one of his strong piled up climactic sentences Paul tells how the
fight is to be won. This sentence runs unbroken through verses fourteen to
twenty inclusive. There are six preliminary clauses in it leading up to
its main statement. These clauses name the pieces of armour used by a
Roman soldier in the action of battle. The loins girt, the breastplate on,
the feet shod, the shield, the helmet the sword, and so on. A Roman
soldier reading this or, hearing Paul preach it, would expect him to
finish the sentence by saying "\textit{with all your fighting strength
fighting}."

That would be the proper conclusion rhetorically of this sentence. But
when Paul reaches the climax with his usual intensity he drops the
rhetorical figure, and puts in the thing with which in our case the
fighting is done--"with all prayer \textit{praying}." In place of the
expected word fighting is the word praying. The thing with which the
fighting is done is put in place of the word itself. Our fighting is
praying. Praying is fighting, spirit-fighting. That is to say, this old
evangelist-missionary-bishop says, we are in the thick of a fight. There
is a war on. How shall we best fight? First get into good shape to pray,
and then with all your praying strength and skill \textit{pray}. That word
\textit{praying} is the climax of this long sentence, and of this whole epistle.
This is the sort of action that turns the enemy's flank, and reveals his
heels. He simply \textit{cannot} stand before persistent knee-work.

Now mark the keenness of Paul's description of the man who does most
effective work in praying. There are six qualifications under the figure
of the six pieces of armour. A clear understanding of truth, a clean
obedient life, earnest service, a strongly simple trust in God, clear
assurance of one's own salvation and relation to God, and a good grip of
the truth for others--these things prepare a man for the real conflict of
prayer. \textit{Such a man}--\textit{praying}--\textit{drives back these hosts of the traitor
prince}. Such a man praying is invincible in his Chief, Jesus. The
equipment is simple, and in its beginnings comes quickly to the willing,
earnest heart.

Look a bit at how the strong climax of this long sentence runs. It is
fairly bristling with points. Soldier-points all of them; like bayonet
points. Just such as a general engaged in a siege-fight would give to his
men. "With all prayer and supplication"--there is \textit{intensity};
"praying"--that is \textit{the main drive}; "at all seasons"--\textit{ceaselessness},
night and day; hot and cold; wet and dry; "in the Spirit"--as \textit{guided by
the Chief;} "and watching thereunto"--\textit{sleepless vigilance;} watching is
ever a fighting word; watch the enemy; watch your own forces; "with all
perseverance"--\textit{persistence}; cheery, jaw-locked, dogged persistence,
bulldog tenacity; "and supplication"--\textit{intensity again}; "for all the
saints"--\textit{the sweep of the action}, keep in touch with the whole army;
"and on my behalf"--the human leader, rally around \textit{the immediate leader.}
This is the foe to be fought. And this the sort of fighting that defeats
this foe.



\underline{A double Wrestling Match.}


Now turn back to the illustration section of our Book for a remarkably
graphic illustration of these words. It is in the old prophecy of Daniel,
tenth chapter. The story is this: Daniel is an old man now. He is an
exile. He has not seen the green hills of his fatherland since boyhood. In
this level Babylon, he is homesick for the dear old Palestinian hills, and
he is heartsick over the plight of his people. He has been studying
Jeremiah's prophecies, and finds there the promise plainly made that after
seventy years these exiled Hebrews are to be allowed to return. Go back
again! The thought of it quickens his pulse-beats. He does some quick
counting. The time will soon be up. So Daniel plans a bit of time for
special prayer, a sort of siege prayer.

Remember who he is--this Daniel. He is the chief executive of the land. He
controls, under the king, the affairs of the world empire of his time. He
is a giant of strength and ability--this man. But he plans his work so as
to go away for a time. Taking a few kindred spirits, who understand
prayer, he goes off into the woods down by the great Tigris River. They
spend a day in fasting, and meditation and prayer. Not utter fasting, but
scant eating of plain food. I suppose they pray awhile; maybe separately,
then together; then read a bit from the Jeremiah parchment, think and talk
it over and then pray some more. And so they spend a whole day reading,
meditating, praying.

They are expecting an answer. These old-time intercessors were strong in
expectancy. But there is no answer. A second day, a third, a fourth, a
week, still no answer reaches them. They go quietly on without hesitation.
Two weeks. How long it must have seemed! Think of fourteen days spent
\textit{waiting}; waiting for something, with your heart on tenter hooks. There
is no answer. God might have been dead, to adapt the words of Catharine
Luther, so far as any answer reaching them is concerned. But you cannot
befool Daniel in that way. He is an old hand at prayer. Apparently he has
no thought of quitting. He goes quietly, steadily on. Twenty days pass,
with no change. Still they persist. Then the twenty-first day comes and
there is an answer. It comes in a vision whose glory is beyond human
strength to bear. By and by when they can talk, his visitor and he, this
is what Daniel hears: "Daniel, the first day you began to pray, your
prayer was heard, and I was sent with the answer." And even Daniel's eyes
open big--"the \textit{first} day--three weeks ago?" "Yes, three weeks ago I left
the presence of God with the answer to your prayer. But"--listen, here is
the strange part--"the prince of the kingdom of Persia withstood me,
resisted me, one and twenty days: but Michael, your prince, came to help
me, and I was free to come to you with the answer to your prayer."

Please notice four things that I think any one reading this chapter will
readily admit. This being talking with Daniel is plainly a spirit being.
He is opposed by some one. This opponent plainly must be a spirit being,
too, to be resisting a spirit being. Daniel's messenger is from God: that
is clear. Then the opponent must be from the opposite camp. And here comes
in the thing strange, unexpected, the evil spirit being \textit{has the power to
detain, hold back God's messenger} for three full weeks by earth's
reckoning of time. Then reenforcements come, as we would say. The evil
messenger's purpose is defeated, and God's messenger is free to come as
originally planned.

There is a double scene being enacted. A scene you can see, and a scene
you cannot see. An unseen wrestling match in the upper spirit realm, and
two embodied spirit beings down on their faces by the river. And both
concerned over the same thing.

That is the Daniel story. What an acted out illustration it is of Paul's
words. It is a picture glowing with the action of real life. It is a
double picture. Every prayer action is in doubles; a lower human level; an
upper spirit level. Many see only the seen, and lose heart. While we look
at the things that are seen, let us gaze intently at the things unseen;
for the seen things are secondary, but the unseen are chief, and the
action of life is being decided there.

Here is the lower, the seen;--a group of men, led by a man of executive
force enough to control an empire, prone on their faces, with minds clear,
quiet, alert, persistently, ceaselessly \textit{praying} day by day. Here is the
upper, the unseen:--a "wrestling," keen, stubborn, skilled, going on
between two spirit princes in the spirit realm. And by Paul's explanation
the two are vitally connected. Daniel and his companions are wrestlers
too, active participants in that upper-air fight, and really deciding the
issue, for they are on the ground being contested. These men are indeed
praying with all prayer and supplication at all times, in the Spirit, and
watching thereunto with all perseverance and supplication, and \textit{at length
victory comes}.



\underline{Prayer Concerns Three.}


Now a bit of a look at the central figure of the pattern. Jesus lets in a
flood of light on Satan's relation to prayer in one of His prayer
parables. There are two parables dealing distinctively with prayer: "the
friend at midnight,"[26] and "the unjust judge."[27] The second of these
deals directly with this Satan phase of prayer. It is Luke through whom we
learn most of Jesus' own praying who preserves for us this remarkable
prayer picture.

It comes along towards the end. The swing has been made from plain talking
to the less direct, parable-form of teaching. The issue with the national
leaders has reached its acutest stage. The culmination of their hatred,
short of the cross, found vent in charging Him with being inspired by the
spirit of Satan. He felt their charge keenly and answered it directly and
fully. His parable of the strong man being bound before his house can be
rifled comes in here. \textit{They} had no question as to what that meant. That
is the setting of this prayer parable. The setting is a partial
interpretation. Let us look at this parable rather closely, for it is full
of help for those who would become skilled in helping God win His world
back home again.

Jesus seems so eager that they shall not miss the meaning here that He
departs from His usual habit and says plainly what this parable is meant
to teach:--"that men ought always to pray, and not to faint." The great
essential, He says, is \textit{prayer}. The great essential in prayer is
\textit{persistence}. The temptation in prayer is that one may lose heart, and
give up, or give in. "Not-to-faint" tells how keen the contest is.

There are three persons in the parable; a judge, a widow, and an
adversary. The judge is utterly selfish, unjust, godless, and reckless of
anybody's opinion. The worst sort of man, indeed, the last sort of man to
be a judge. Inferentially he knows that the right of the case before him
is with the widow. The widow--well, she is a \textit{widow}. Can more be said to
make the thing vivid and pathetic! A very picture of friendlessness and
helplessness is a widow. A woman needs a friend. This woman has lost her
nearest, dearest friend; her protector. She is alone. There is an
adversary, an opponent at law, who has unrighteously or illegally gotten
an advantage over the widow and is ruthlessly pushing her to the wall. She
is seeking to get the judge to join with her against her adversary. Her
urgent, oft repeated request is, "avenge me of mine adversary." That is
Jesus' pictorial illustration of persistent prayer.

Let us look into it a little further. "Adversary" is a common word in
scripture for Satan. He is the accuser, the hater, the enemy, the
adversary. Its meaning technically is "an opponent in a suit at law." It
is the same word as used later by Peter, "Your adversary the devil as a
roaring lion, goeth about, seeking whom he may devour."[28] The word
"avenge" used four times really means, "do me justice." It suggests that
the widow has the facts on her side to win a clear case, and that the
adversary has been bully-ragging his case through by sheer force.

There is a strange feature to this parable, which must have a meaning. \textit{An
utterly godless unscrupulous man is put in to represent God!} This is
startling. In any other than Jesus it would seem an overstepping of the
bounds. But there is keenness of a rare sort here. Such a man is chosen
for judge to bring out most sharply this:--the sort of thing required to
win this judge is certainly not required \textit{with God}. The widow must
persist and plead because of the sort of man she has to deal with. But God
is utterly different in character. Therefore while persistence is urged in
prayer plainly it is not for the reason that required the widow to
persist. And if that reason be cut out it leaves only one other, namely,
that represented by the adversary.

Having purposely put such a man in the parable for God, Jesus takes pains
to speak of the real character of God. "And He is \textit{long-suffering} over
them." \textit{That} is God. That word "long-suffering" and its equivalent on
Jesus' lips suggests at once the strong side of love, namely, \textit{patience},
gentle, fine patience. It has bothered the scholars in this phrase to know
with whom or over what the long-suffering is exercised. "Over them" is the
doubtful phrase. Long-suffering over these praying ones? \textit{Or},
long-suffering in dealing righteously with some stubborn adversary--which?
The next sentence has a word set in sharpest contrast with this one,
namely "speedily." "Long-suffering" yet "speedily."

Here are gleams of bright light on a dark subject with apparently more
light obscured than is allowed to shine through. Jesus always spoke
thoughtfully. He chooses His words. Remembering the adversary against whom
the persistence is directed the whole story seems to suggest this: that
there is \textit{a great conflict on} in the upper spirit world. Concerning it
our patient God is long-suffering. He is a just and righteous God. These
beings in the conflict are all His creatures. He is just in His dealings
with the devil and this splendid host of evil spirits even as with all His
creation. He is long-suffering that no unfairness shall be done in His
dealings with these creatures of His. Yet at the same time He is doing His
best to bring the conflict to a speedy end, for the sake of His loyal
loved ones, and that right may prevail.

The upshot of the parable is very plain. It contains for us two
tremendous, intense truths. First is this: \textit{prayer concerns three}, not
two but three. God to whom we pray, the man on the contested earth who
prays, and the evil one against whom we pray. And the purpose of the
prayer is not to persuade or influence God, but to join forces with Him
against the enemy. Not towards God, but with God against Satan--that is
the main thing to keep in mind in prayer. The real pitch is not Godward
but Satanward.

The second intense truth is this:--the winning quality in prayer is
\textit{persistence}. The final test is here. This is the last ditch. Many who
fight well up to this point lose their grip here, and so lose all. Many
who are well equipped for prayer fail here, and doubtless fail because
they have not rightly understood. With clear, ringing tones the Master's
voice sounds in our ears again to-day, "always to pray, \textit{and} not to
faint."



\underline{A Stubborn Foe Routed.}


That is the parable teaching. Now a look at a plain out word from the
Master's lips. It is in the story of the demonized boy, the distressed
father, and the defeated disciples, at the foot of the transfiguration
mountain.[29] Extremes meet here surely. The mountain peak is in sharpest
contrast with the valley. The demon seems to be of the superlative degree.
His treatment of the possessed boy is malicious to an extreme. His purpose
is "to destroy" him. Yet there is a limit to his power, for what he would
do he has not yet been able to do. He shows extreme tenacity. He fought
bitterly against being disembodied again. (Can it be that embodiment eases
in some way the torture of existence for these prodigal spirits!) And so
far he fought well, and with success. The disciples had tried to cast him
out. They were expected to. They expected to. They had before. They
failed!--dismally--amid the sneering and jeering of the crowd and the
increasing distress of the poor father.

Then Jesus came. Was some of the transfiguring glory still lingering in
that great face? It would seem so. The crowd was "amazed" when they saw
Him, and "saluted" Him. His presence changed all. The demon angrily left,
doing his worst to wreck the house he had to vacate. The boy is restored;
and the crowd astonished at the power of God.

Then these disciples did a very keen thing. They made some bad blunders
but this is not one of them. They sought a private talk with Jesus. No
shrewder thing was ever done. When you fail, quit your service and get
away for a private interview with Jesus. With eyes big, and voices
dejected, the question wrung itself out of their sinking hearts, "Why
could not \textit{we} cast it out?" Matthew and Mark together supply the full
answer. Probably first came this:--"because of your little faith." They
had quailed in their hearts before the power of this malicious demon. And
the demon knew it. They were more impressed with the power of the demon
than with the power of God. And the demon saw it. They had not prayed
victoriously against the demon. The Master says, "faith only as big as a
mustard seed (you cannot measure the strength of the mustard seed by its
size) will say to this mountain--'Remove.'" Mark keenly:--the direction of
the faith is towards the obstacle. Its force is against the enemy. It was
the demon who was most directly influenced by Jesus' faith.

Then comes the second part of the reply:--"This kind can come out by
nothing but by prayer." Some less-stubborn demons may be cast out by the
faith that comes of our regular prayer-touch with God. This extreme sort
takes special prayer. This kind of a demon goes out by prayer. It can be
put out by nothing less. The real victory must be in the secret place. The
exercise of faith in the open battle is then a mere pressing of the
victory already won. These men had the language of Jesus on their lips,
but they had not gotten the victory first off somewhere alone. This demon
is determined not to go. He fights stubbornly and strongly. He succeeds.
Then this \textit{Man of Prayer} came. The quiet word of command is spoken. The
demon must go. These disciples were strikingly like some of us. They had
not \textit{realized} where the real victory is won. They had used the word of
command to the demon, doubtless coupling Jesus' name with it. But there
was not the secret touch with God that gives victory. Their eyes showed
their fear of the demon.

Prayer, real prayer, intelligent prayer, it is this that routs Satan's
demons, for it routs their chief. David killed the lion and bear in the
secret forests before he faced the giant in the open. These disciples were
facing the giant in the open without the discipline in secret. "This kind
can be compelled to come out by nothing but by prayer," means this:--"this
kind comes out, and must come out, before the man who prays." This thing
which Jesus calls prayer casts out demons. Would that we knew better by
experience what He meant by prayer. It exerts a positive influence upon
the hosts of evil spirits. They fear it. They fear the man who becomes
skilled in its use.

There are yet many other passages in this Bible fully as explicit as
these, and which give on the very surface just such plain teaching as
these. The very language of scripture throughout is full of this truth.
But these four great instances are quite sufficient to make the present
point clear and plain. This great renegade prince is an actual active
factor in the lives of men. He believes in the potency of prayer. He fears
it. He can hinder its results for a while. He does his best to hinder it,
and to hinder as long as possible.

\textit{Prayer overcomes him.} It defeats his plans and himself. He cannot
successfully stand before it. He trembles when some man of simple faith in
God prays. Prayer is insistence upon God's will being done. It needs for
its practice a man in sympathetic touch with God. Its basis is Jesus'
victory. It overcomes the opposing will of the great traitor-leader.




\part{How to Pray}


% 1. The "How" of Relationship.
% 2. The "How" of Method.
% 3. The Listening Side of Prayer.
% 4. Something about God's Will in Connection with Prayer.
% 5. May We Pray with Assurance for the Conversion of Our Loved Ones?

% \item How to Pray \begin{enumerate}  \item The "How" of Relationship  \item The "How" of Method  \item The Listening Side of Prayer  \item Something about God's Will in Connection with Prayer  \item May We Pray with Assurance for the Conversion of Our Loved Ones  \end{enumerate}


\chapter{The "How" of Relationship}



\underline{God's Ambassadors.}


If I had an ambition to be the ambassador of this country to our
mother-country, there would be two essential things involved. The first
and great essential would be to receive the appointment. I would need to
come into certain relation with our president, to possess certain
qualifications considered essential by him, and to secure from his hand
the appointment, and the official credentials of my appointment. That
would establish my relationship to the foreign court as the representative
of my own country, and my right to transact business in her name.

But having gotten that far I might go over there and make bad mistakes. I
might get our diplomatic relations tangled up, requiring many
explanations, and maybe apologies, and leaving unpleasant memories for a
long time to come. Such incidents have not been infrequent. Nations are
very sensitive. Governmental affairs must be handled with great nicety.
There would be a second thing which if I were a wise enough man to be an
ambassador I would likely do. I would go to see John Hay and Joseph H.
Choate, and have as many interviews with them as possible, and learn all I
possibly could from them of London official life, court etiquette,
personages to be dealt with, things to do, and things to avoid. How to be
a successful diplomat and further the good feeling between the two
governments, and win friends for our country among the sturdy Britons
would be my one absorbing thought. And having gotten all I could in that
way I would be constantly on the alert with all the mental keenness I
could command to practice being a successful ambassador.

The first of these would make me technically an ambassador. The second
would tend towards giving me some skill as an ambassador. Now there are
the same two how's in praying. First the relationship must be established
before any business can be transacted. Then skill must be acquired in the
transacting of the business on hand.

Just now, we want to talk about the first of these, the how of
relationship in prayer. The basis of prayer is right relationship with
God. Prayer is representing God in the spirit realm of this world. It is
insisting upon His rights down in this sphere of action. It is standing
for Him with full powers from Him. Clearly the only basis of such
relationship to God is \textit{Jesus}. We have been outlawed by sin. We were in
touch with God. We broke with Him. The break could not be repaired by us.
Jesus came. He was God \textit{and} Man. He touches both. We get back through
Him, and only so. The blood of the cross is the basis of all prayer.
Through it the relationship is established that underlies all prayer. Only
as I come to God through Jesus to get the sin score straightened, and only
as I keep in sympathy with Jesus in the purpose of my life can I practice
prayer.



\underline{Six Sweeping Statements.}


Jesus' own words make this very clear. There are two groups of teachings
on prayer in those three and a half years as given by the gospel records.
The first of these groups is in the Sermon on the Mount which Jesus
preached about half-way through the second year of His ministry. The
second group comes sheer at the end. All of it is in the last six months,
and most of it in the last ten days, and much of that on the very eve of
that last tragic day.

It is after the sharp rupture with the leaders that this second series of
statements is made. The most positive, and most sweeping utterances on
prayer are here. Of Jesus' eight promises regarding prayer six are here. I
want to ask you please to notice these six promises or statements; and
then, to notice their relation to our topic of to-day.

In Matthew 18:19, 20, is the first of these. "Again I say unto you, that
if two of you shall agree on earth, as touching anything that they Shall
ask, it shall be done for them of My Father who is in heaven." Notice the
place of prayer--"on earth"; and the sweep--"anything"; and the
positiveness--"it shall be done." Then the reason why is given. "For where
two or three are gathered together in My name, there am I in the midst of
them." That is to say, if there are two persons praying, there are three.
If three meet to pray, there are four praying. There is always one more
than you can see. And if you might perhaps be saying to yourself in a bit
of dejection, "He'll not hear me: I'm so sinful: so weak"--you would be
wrong in thinking and saying so, but then we do think and say things that
are not right--\textit{if} you might be thinking that, you could at once fall
back upon this: the Father always hears Jesus. And wherever earnest hearts
pray Jesus is there taking their prayer and making it His prayer.

The second of these: Mark 11:22-24, "Jesus answering saith unto them, have
faith in God"--with the emphasis double-lined under the word "God." The
chief factor in prayer is God. "Verily I say unto you, whosoever shall say
unto this mountain, be thou taken up and cast into the sea--" Choosing, do
you see the unlikeliest thing that might occur. Such a thing did not take
place. We never hear of Jesus moving an actual mountain. The need for such
action does not seem to have arisen. But He chooses the thing most
difficult for His illustration. Can you imagine a mountain moving off into
the sea--the Jungfrau, or Blanc, or Rainier? If you know mountains down in
your country you cannot imagine it actually occurring. "--And shall not
doubt in his heart--" That is Jesus' definition of faith. "--But shall
believe that what he saith cometh to pass; he shall have it. Therefore, I
say unto you, all things whatsoever ye pray and ask for, believe that ye
receive them, and ye shall have them." How utterly sweeping this last
statement! And to make it more positive it is preceded by the emphatic
"therefore--I--say--unto--you." Both whatsoever and whosoever are here.
Anything, and anybody. We always feel instinctively as though these
statements need careful guarding: a few fences put up around them. Wait a
bit and we shall see what the Master's own fence is.

The last four of the six are in John's gospel. In that last long quiet
talk on the night in which He was betrayed. John preserves much of that
heart-talk for us in chapters thirteen to seventeen.

Here in John 14:13, 14: "And whatsoever ye shall ask in My name, that will
I do, that the Father may be glorified in the Son. If ye shall ask
anything in My name, that will I do." The repetition is to emphasize the
unlimited sweep of what may be asked.

John 15:7: "If ye abide in Me, and My words abide in you--" That word
abide is a strong word. It does not mean to leave your cards; nor to hire
a night's lodging; nor to pitch a tent, or run up a miner's shanty, or a
lumberman's shack. It means moving in to stay. "--Ask whatsoever ye
will--" The Old Version says, "ye shall ask." But here the revised is more
accurate: "Ask; please ask; I ask you to ask." There is nothing said
directly about God's will. There is something said about our wills. "--And
it shall be done unto you." Or, a little more literally, "I will bring it
to pass for you."

I remember the remark quoted to me by a friend one day. His church
membership is in the Methodist Church of the North, but his service
crosses church lines both in this country and abroad. He was talking with
one of the bishops of that church whose heart was in the foreign mission
field. The bishop was eager to have this friend serve as missionary
secretary of his church. But he knew, as everybody knows, how difficult
appointments oftentimes are in all large bodies. He was earnestly
discussing the matter with my friend, and made this remark: "If you will
allow the use of your name for this appointment, \textit{I will lay myself out}
to have it made." Now if you will kindly not think there is any lack of
reverence in my saying so--and there is surely none in my thought--that is
the practical meaning of Jesus' words here. "If you abide in Me, and My
words sway you, you please ask what it is your will to ask. And--softly,
reverently now--I will lay Myself out to bring that thing to pass for
you." That is the force of His words here.

This same chapter, sixteenth verse: "Ye did not choose Me, but I chose
you, and appointed you, that ye should go and bear fruit, and that your
fruit should abide; that whatsoever ye shall ask of the Father in My name,
He may give it you." God had our prayer partnership with Himself in His
mind in choosing us. And the last of these, John 16:23, 24, second clause,
"Verily, verily, I say unto you, if ye shall ask anything of the Father,
He will give it you in My name. Hitherto have ye asked nothing in My name:
ask, and ye shall receive, that your joy may be fulfilled."

These statements are the most sweeping to be found anywhere in the
Scriptures regarding prayer. There is no limitation as to who shall ask,
nor the kind of thing to be asked for. There are three limitations
imposed: the prayer is to be \textit{through Jesus}; the person praying is to be
in fullest sympathy with Him; and this person is to have faith.



\underline{Words With a Freshly Honed Razor-Edge.}


Now please group these six sweeping statements in your mind and hold them
together there. Then notice carefully this fact. These words are not
spoken to the crowds. They are spoken to the small inner group of twelve
disciples. Jesus talks one way to the multitude. He oftentimes talks
differently to these men who have separated themselves from the crowd and
come into the inner circle.

And notice further that before Jesus spoke these words to this group of
men He had said something else first. Something very radical; so radical
that it led to a sharp passage between Himself and Peter, to whom He
speaks very sternly. This something else fixes unmistakably their relation
to Himself. Remember that the sharp break with the national leaders has
come. Jesus is charged with Satanic collusion. The death plot is
determined upon. The breach with the leaders is past the healing point.
And now the Master is frequently slipping away from the crowd with these
twelve men, and seeking to teach and train them. That is the setting of
these great promises. It must be kept continually in mind.

Before the Master gave Himself away to these men in these promises He said
this something else. It is this. I quote Matthew's account: "If any man
would come after Me let him deny himself and take up his cross (daily,
Luke's addition) and follow Me[30]." \textit{These words should be written
crosswise over those six prayer statements}. Jesus never spoke a keener
word. Those six promises are not meant for all. Let it be said very
plainly. They are meant only for those who will square their lives by
these razor-edged words.

I may not go fully into the significance of these deep-cutting words here.
They have been gone into at some length in a previous set of talks as
suggesting the price of power. To him whose heart burns for power in
prayer I urge a careful review of that talk in this new setting of it. "If
any man would come after Me" means a rock-rooted purpose; the jaw locked;
the tendrils of the purpose going down around and under the gray granite
of a man's will, and tying themselves there; and knotting the ties; sailor
knots, that you cannot undo.

"Come after Me" means all the power of Jesus' life, and has the other
side, too. It means the wilderness, the intense temptation. It may mean
the obscure village of Nazareth for you. It may mean that first Judean
year for you--lack of appreciation. It may mean for you that last six
months--the desertion of those hitherto friendly. It will mean without
doubt a Gethsemane. Everybody who comes along after Jesus has a Gethsemane
in his life. It will never mean as much to you as it meant to Him. That is
true. But, then, it will mean everything to you. And it will mean too
having a Calvary in your life in a very real sense, though different from
what that meant to Him. This sentence through gives the process whereby
the man with sin grained into the fibre of his will may come into such
relationship with God as to claim without any reservation these great
prayer promises. And if that sound hard and severe to you let me quickly
say that it is an easy way for the man who is \textit{willing.} The presence of
Jesus in the life overlaps every cutting thing.

If a man will go through Matthew 16:24, and habitually live there he may
ask what he wills to ask, and that thing will come to pass. The reason,
without question, why many people do not have power in prayer is simply
because they are unwilling--I am just talking very plainly--they are
unwilling to bare their breasts to the keen-edged knife in these words of
Jesus. And on the other side, if a man will quietly, resolutely follow the
Master's leading--nothing extreme--nothing fanatical, or morbid, just a
quiet going where that inner Voice plainly leads day by day, he will be
startled to find what an utterly new meaning prayer will come to have for
him.



\underline{The Controlling Purpose.}


Vital relationship is always expressed by purpose. The wise ambassador has
an absorbing purpose to further the interests of his government. Jesus
said, and it at once reveals His relationship to God, "I do always those
things that are well pleasing to him."

The relationship that underlies prayer has an absorbing purpose. Its
controlling purpose is to please Jesus. That sentence may sound simple
enough. But, do you know, there is no sentence I might utter that has a
keener, a more freshly honed razor-edge to it than that. That the purpose
which \textit{controls} my action in every matter be this: to please Him. If you
have not done so, take it for a day, a week, and use it as a touch stone
regarding thought, word and action. Take it into matters personal, home,
business, social, fraternal. It does not mean to ask, "Is this right? is
this wrong?" Not that. Not the driving of a keen line between wrong and
right. There are a great many things that can be proven to be not wrong,
but that are not best, that are not His preference.

It will send a business man running his eye along the shelves and counter
of his store. "The controlling purpose to please Jesus ... hm-m-m, I guess
maybe that stuff there ought to come out. Oh, it is not wrong: I can prove
that. My Christian brother-merchants handle it here, and over the country:
but \textit{to please Him}: a good, clean sixty per cent, profit too, cash money,
but \textit{to please Him}--" and the stuff must go down and out.

It would set some woman to thinking about the next time the young people
are to gather in her home for a delightful social evening with her own
daughters. She will think about some forms of pastime that are found
everywhere. They are not wrong, that has been conclusively proven. But \textit{to
please Him}. Hm-m. And these will go out. And then it will set her to
work with all her God-given woman-wit and exquisite tact to planning an
evening yet more delightful. It will make one think of his personal
habits, his business methods, and social intercourse, the organizations he
belongs to, with the quiet question cutting it razor-way into each.

And if some one listening may ask: Why put the condition of prayer so
strongly as that? I will remind you of this. The true basis of prayer is
sympathy, oneness of purpose. Prayer is not extracting favours from a
reluctant God. It is not passing a check in a bank window for money. That
is mandatory. The roots of prayer lie down in oneness of purpose. God up
yonder, His Victor-Son by His side, and a man down here, in \textit{such
sympathetic touch} that God can think His thoughts over in this man's
mind, and have His desires repeated upon the earth as this man's prayer.



\underline{The Threefold Cord of Jesus' Life.}


Think for a moment into Jesus' human life down here. His marvellous
activities for those few years over which the world has never ceased to
wonder. Then His underneath hidden-away prayer-life of which only
occasional glimpses are gotten. Then grouping around about that sentence
of His--"I do always the things that are pleasing to Him"--in John's
gospel, pick out the emphatic negatives on Jesus' lips, the "not's": not
My will, not My works, not My words. Jesus came to do somebody's else
will. The controlling purpose of His life was to please His Father. That
was the secret of the power of His earthly career. Right relationship to
God; a secret intimate prayer-life: marvellous power over men and with
men--those are the strands in the threefold cord of His life.

There is a very striking turn of a word in the second chapter of John's
gospel down almost at its close. The old version says that "Many believed
on His name beholding His signs which He did, but Jesus did not commit
Himself unto them" because He knew them so well. The word "believed," and
the word "commit" are the same word underneath our English. The sentence
might run "many \textit{trusted} Him beholding what He did; but He did not
\textit{trust} them for He knew them." I have no doubt most, or all of us here
to-day, trust Him. Let me ask you very softly now: Can He trust you? While
we might all shrink from saying "yes" to that, there is a very real sense
in which we may say "yes," namely, in the purpose of the life. Every life
is controlled by some purpose. What is yours? To please Him? If so He
knows it. It is a great comfort to remember that God judges a man not by
his achievements, but by his purposes: not by what I am, actually, but by
what I would be, in the yearning of my inmost heart, the dominant purpose
of my life. God will fairly flood your life with all the power He can
trust you to use wholly for Him.

Commercial practice furnishes a simple but striking illustration here. A
man is employed by a business house as a clerk. His ability and honesty
come to be tested in many ways constantly. He is promoted gradually, his
responsibilities increased. As he proves himself thoroughly reliable he is
trusted more and more, until by and by as need arises he becomes the
firm's confidential clerk. He knows its secrets. He is trusted with the
combination to the inner box in the vault. Because it has been proven by
actual test that he will use everything only for the best interests of his
house, and not selfishly.

Here, where we are dealing, the whole thing moves up to an infinitely
higher level, but the principle does not change. If I will come into the
relationship implied in these words:--it shall be the one controlling
desire and purpose of my life to do the things that please Him--then I may
ask for what I will, and it shall be done. That is how to pray: the how of
relationship. The man who will live in Matthew 16:24, and follow Jesus as
He leads: simply that: no fanaticism, no morbidism, no extremism, just
simply follow as He leads, day by day,--then those six promises of Jesus
with their wonderful sweep, their limitless sweep are his to use as he
will.




\chapter{The "How" of Method}



\underline{Touching the Hidden Keys.}


One of the most remarkable illustrations in recent times of the power of
prayer, may be found in the experience of Mr. Moody. It explains his
unparalleled career of world-wide soul winning. One marvels that more has
not been said of it. Its stimulus to faith is great. I suppose the man
most concerned did not speak of it much because of his fine modesty. The
last year of his life he referred to it more frequently as though impelled
to.

The last time I heard Mr. Moody was in his own church in Chicago. It was,
I think, in the fall of the last year of his life. One morning in the old
church made famous by his early work, in a quiet conversational way he
told the story. It was back in the early seventies, when Chicago had been
laid in ashes. "This building was not yet up far enough to do much in," he
said; "so I thought I would slip across the water, and learn what I could
from preachers there, so as to do better work here. I had gone over to
London, and was running around after men there." Then he told of going
one evening to hear Mr. Spurgeon in the Metropolitan Tabernacle; and
understanding that he was to speak a second time that evening to dedicate
a chapel, Mr. Moody had slipped out of the building and had run along the
street after Mr. Spurgeon's carriage a mile or so, so as to hear him the
second time. Then he smiled, and said quietly, "I was running around after
men like that."

He had not been speaking anywhere, he said, but listening to others. One
day, Saturday, at noon, he had gone into the meeting in Exeter Hall on the
Strand; felt impelled to speak a little when the meeting was thrown open,
and did so. At the close among others who greeted him, one man, a
minister, asked him to come and preach for him the next day morning and
night, and he said he would. Mr. Moody said, "I went to the morning
service and found a large church full of people. And when the time came I
began to speak to them. But it seemed the hardest talking ever I did.
There was no response in their faces. They seemed as though carved out of
stone or ice. And I was having a hard time: and wished I wasn't there; and
wished I hadn't promised to speak again at night. But I had promised, and
so I went.

"At night it was the same thing: house full, people outwardly respectful,
but no interest, no response. And I was having a hard time again. When
about half-way through my talk there came a change. It seemed as though
the windows of heaven had opened and a bit of breath blew down. The
atmosphere of the building seemed to change. The people's faces changed.
It impressed me so that when I finished speaking I gave the invitation for
those who wanted to be Christians to rise. I thought there might be a few.
And to my immense surprise the people got up in groups, pew-fulls. I
turned to the minister and said, 'What does this mean?' He said, 'I don't
know, I'm sure.' Well," Mr. Moody said, "they misunderstood me. I'll
explain what I meant." So he announced an after-meeting in the room below,
explaining who were invited: only those who wanted to be Christians; and
putting pretty clearly what he understood that to mean, and dismissed the
service.

They went to the lower room. And the people came crowding, jamming in
below, filling all available space, seats, aisles and standing room. Mr.
Moody talked again a few minutes, and then asked those who would be
Christians to rise. This time he knew he had made his meaning clear. They
got up in clumps, in groups, by fifties! Mr. Moody said, "I turned and
said to the minister, 'What \textit{does} this mean?' He said, 'I'm sure I don't
know.'" Then the minister said to Mr. Moody, "What'll I do with these
people? I don't know what to do with them; this is something new." And he
said, "Well. I'd announce a meeting for to-morrow night, and Tuesday
night, and see what comes of it; I'm going across the channel to Dublin."
And he went, but he had barely stepped off the boat when a cablegram was
handed him from the minister saying, "Come back at once. Church packed."
So he went back, and stayed ten days. And the result of that ten days, as
I recall Mr. Moody's words, was that four hundred were added to that
church, and that every church near by felt the impulse of those ten days.
Then Mr. Moody dropped his head, as though thinking back, and said: "I had
no plans beyond this church. I supposed my life work was here. But the
result with me was that I was given a roving commission and have been
working under it ever since."

Now what was the explanation of that marvellous Sunday and days following?
It was not Mr. Moody's doing, though he was a leader whom God could and
did mightily use. It was not the minister's doing; for he was as greatly
surprised as the leader. There was some secret hidden beneath the surface
of those ten days. With his usual keenness Mr. Moody set himself to ferret
it out.

By and by this incident came to him. A member of the church, a woman, had
been taken sick some time before. Then she grew worse. Then the physician
told her that she would not recover. That is, she would not die at once,
so far as he could judge, but she would be shut in her home for years.
And she lay there trying to think what that meant: to be shut in for
years. And she thought of her life, and said, "How little I've done for
God: practically nothing: and now what can I do shut in here on my back."
And she said, "I can pray."

May I put this word in here as a parenthesis in the story--that God
oftentimes allows us to be shut in--He does not shut us in--He does not
need to--simply take His hand off partly--there is enough disobedience to
His law of our bodies all the time to shut us aside--no trouble on that
side of the problem--\textit{with pain to Himself}, against His own first will
for us, He allows us to be shut in, because only so \textit{can} He get our
attention from other things to what He wants done; get us to see things,
and think things His way. I am compelled to think it is so.

She said, "I \textit{will} pray." And she was led to pray for her church. Her
sister, also a member of the church, lived with her, and was her link with
the outer world. Sundays, after church service, the sick woman would ask,
"Any special interest in church to-day?" "No," was the constant reply.
Wednesday nights, after prayer-meetings, "Any special interest in the
service to-night? there must have been." "No; nothing new; same old
deacons made the same old prayers."

But one Sunday noon the sister came in from service and asked, "Who do you
think preached to-day?" "I don't know, who?" "Why, a stranger from
America, a man called Moody, I think was the name." And the sick woman's
face turned a bit whiter, and her eye looked half scared, and her lip
trembled a bit, and she quietly said: "I know what that means. There's
something coming to the old church. Don't bring me any dinner. I must
spend this afternoon in prayer." And so she did. And that night in the
service that startling change came.

Then to Mr. Moody himself, as he sought her out in her sick room, she told
how nearly two years before there came into her hands a copy of a paper
published in Chicago called the \textit{Watchman} that contained a talk by Mr.
Moody in one of the Chicago meetings, Farwell Hall meetings, I think. All
she knew was that talk that made her heart burn, and there was the name
M-o-o-d-y. And she was led to pray that God would send that man into their
church in London. As simple a prayer as that.

And the months went by, and a year, and over; still she prayed. Nobody
knew of it but herself and God. No change seemed to come. Still she
prayed. And of course her prayer wrought its purpose. Every
Spirit-suggested prayer does. And that is the touchstone of true prayer.
And the Spirit of God moved that man of God over to the seaboard, and
across the water and into London, and into their church. Then a bit of
special siege-prayer, a sort of last charge up the steep hill, and that
night the victory came.

Do you not believe--I believe without a doubt, that some day when the
night is gone and the morning light comes up, and we know as we are known,
that we shall find that the largest single factor, in that ten days' work,
and in the changing of tens of thousands of lives under Moody's leadership
is that woman in her praying. Not the only factor, mind you. Moody a man
of rare leadership, and consecration, and hundreds of faithful ministers
and others rallying to his support. But behind and beneath Moody and the
others, and to be reckoned with as first this woman's praying.

Yet I do not know her name. I know Mr. Moody's name. I could name scores
of faithful men associated with him in his campaigns, but the name of this
one in whom humanly is the secret of it all I do not know. Ah! It is a
secret service. We do not know who the great ones are. They tell me she is
living yet in the north end of London, and still praying. Shall we pray!
Shall we not pray! If something else must slip out, something important,
shall we not see to it that intercession has first place!



\underline{Making God's Purpose Our Prayer.}


With that thought in mind let me this evening suggest a bit of how to
pray. As simple a subject as that: how to pray: the how of method.

The first thing in prayer is to find God's purpose, the trend, the swing
of it; the second thing to make that purpose our prayer. We want to find
out what God is thinking, and then to claim that that shall be done. God
is seated up yonder on the throne. Jesus Christ is sitting by His side
glorified. Everywhere in the universe God's will is being done except in
this corner, called the earth, and its atmosphere, and that bit of the
heavens above it where Satan's headquarters are.

It has been done down here by one person--Jesus. He came here to this
prodigal planet and did God's will perfectly. He went away. And He has
sought and seeks to have men down upon the earth so fully in touch with
Himself that He may do in them and through them just what He will. That He
may reproduce Himself in these men, and have God's will done again down on
the earth. Now prayer is this: finding out God's purpose for our lives,
and for the earth and insisting that that shall be done here. The great
thing then is to find out and insist upon God's will. And the "how" of
method in prayer is concerned with that.

Many a time I have met with a group of persons for prayer. Various special
matters for prayer are brought up. Here is this man, needing prayer, and
this particular matter, and this one, and this. Then we kneel and pray.
And I have many a time thought--not critically in a bad sense--as I have
listened to their prayers, as though this is the prayer I must
offer:--"Blessed Holy Spirit, Thou knowest this man, and what the lacking
thing is in him. There is trouble there. Thou knowest this sick woman, and
what the difficulty is there. This problem, and what the hindrance is in
it. Blessed Spirit, pray in me the prayer Thou art praying for this man,
and this thing, and this one. The prayer Thou art praying, I pray that, in
Jesus' name. Thy will be done here under these circumstances."

Sometimes I feel clear as to the particular prayer to offer, but many a
time I am puzzled to know. I put this fact with this, but I may not know
\textit{all} the facts. I know this man who evidently needs praying for, a
Christian man perhaps, his mental characteristics, his conceptions of
things, the kind of a will he has, but there may be some fact in there
that I do not know, that seriously affects the whole difficulty. And I am
compelled to fall back on this: I don't know how to pray as I ought. But
the Spirit within me will make intercession for this man as I allow Him to
have free swing in me as the medium of His prayer. And He who is listening
above as He hears His will for this man being repeated down on the
battle-field will recognize His own purpose, of course. And so that thing
will be working out because of Jesus' victory over the evil one.

But I may become so sensitive to the Spirit's thoughts and presence, that
I shall know more keenly and quickly what to pray for. In so far as I do
I become a more skillful partner of His on the earth in getting God's will
done.



\underline{The Trysting Place.}


There are six suggestions here on how to pray. First--we need \textit{time} for
prayer, unhurried time, daily time, time enough to forget about how much
time it is. I do not mean now: rising in the morning at the very last
moment, and dressing, it may be hurriedly, and then kneeling a few moments
so as to feel easier in mind: not that. I do not mean the last thing at
night when you are jaded and fagged, and almost between the sheets, and
then remember and look up a verse and kneel a few moments: not that. That
is good so far as it goes. I am not criticising that. Better sweeten and
sandwich the day with all of that sort you can get in. But just now I mean
this: \textit{taking time} when the mind is fresh and keen, and the spirit
sensitive, to thoughtfully pray. We haven't time. Life is so crowded. It
must be taken from something else, something important, but still less
important than this.

Sacrifice is the continual law of life. The important thing must be
sacrificed to the more important. One needs to cultivate a mature
judgment, or his strength will be frizzled away in the less important
details, and the greater thing go undone, or be done poorly with the
fag-ends of strength. If we would become skilled intercessors, and know
how to pray simply enough, we must take quiet time daily to get off alone.

The second suggestion: we need a \textit{place} for prayer. Oh! you can pray
anywhere, on the street, in the store, travelling, measuring dry goods,
hands in dishwater,--where not. But you are not likely to unless you have
been off in some quiet place shut in alone with God. The Master said:
"Enter into thine inner chamber, and having shut thy door": that door is
important. It shuts out, and it shuts in. "Pray to thy Father who is in
secret." God is here in this shut-in spot. One must get alone to find out
that he never is alone. The more alone we are as far as men are concerned
the least alone we are so far a; God is concerned.

The quiet place and time are needful to train the ears for keen hearing. A
mother will hear the faintest cry of her babe just awaking. It is
up-stairs perhaps; the tiniest bit of a sound comes; nobody else hears;
but quick as a flash the mother's hands are held quiet, the head alert,
then she is off. Her ears are trained beyond anybody's else; love's
training. We need trained ears. A quiet place shuts out the outer sounds,
and gives the inner ear a chance to learn other sounds.

A man was standing in a telephone booth trying to talk, but could not make
out the message. He kept saying, "I can't hear, I can't hear." The other
man by and by said sharply, "If you'll shut that door you can hear." \textit{His}
door was shut and he could hear not only the man's voice but the street
and store noises too. Some folks have gotten their hearing badly confused
because their doors have not been shut enough. Man's voice and God's voice
get mixed in their ears. They cannot tell between them. The bother is
partly with the door. If you'll shut that door you can hear.

The third suggestion needs much emphasis to-day: \textit{give the Book of God its
place in prayer.} Prayer is not talking to God--simply. It is listening
first, then talking. Prayer needs three organs of the head, an ear, a
tongue and an eye. First an ear to hear what God says, then a tongue to
speak, then an eye to look out for the result. Bible study is the
listening side of prayer. The purpose of God comes in through the ear,
passes through the heart taking on the tinge of your personality, and goes
out at the tongue as prayer. It is pathetic what a time God has getting a
hearing down here. He is ever speaking but even where there may be some
inclination to hear the sounds of earth are choking in our ears the sound
of His voice. God speaks in His Word. The most we know of God comes to us
here. This Book is God in print. It was inspired, and it \textit{is} inspired.
God Himself speaks in this Book. That puts it in a list by itself, quite
apart from all others. Studying it keenly, intelligently, reverently will
reveal God's great will. What He says will utterly change what you will
say.



\underline{Our Prayer Teacher.}


The fourth suggestion is this: \textit{Let the Spirit teach you how to pray}. The
more you pray the more you will find yourself saying to yourself, "I don't
know how to pray." Well God understands that. Paul knew that out of his
own experience before he wrote it down. And God has a plan to cover our
need there. There is One who is a master intercessor. He understands
praying perfectly. He is the Spirit of prayer. God has sent Him down to
live inside you and me, partly for this, to teach us the fine art of
prayer. The suggestion is this: let Him teach you.

When you go alone in the quiet time and place with the Book quietly pray:
"blessed Prayer-Spirit, Master-Spirit, teach me how to pray," and He will.
Do not be nervous, or agitated, wondering if you will understand. Study to
be quiet; mind quiet, body quiet. Be still and listen. Remember Luther's
version of David's words,[31] "Be silent to God, and let Him mould thee."

You will find your praying changing. You will talk more simply, like a man
transacting business or a child asking, though of course with a reverence
and a deepness of feeling not in those things. You will quit asking for
some things. Some of the old forms of prayer will drop from your lips
likely enough. You will use fewer words, maybe, but they will be spoken
with a quiet absolute faith that this thing you are asking is being worked
out.

This thing of \textit{letting the Spirit teach} must come first in one's praying,
and remain to the last, and continue all along as the leading dominant
factor. He is a Spirit of prayer peculiarly. The highest law of the
Christian life is obedience to the leading of the Holy Spirit. There needs
to be a cultivated judgment in reading His leading, and not mistaking our
haphazard thoughts as His voice. He should be allowed to teach us how to
pray and more, to dominate our praying. The whole range and intensity of
the spirit conflict is under His eye. He is God's General on the field of
action. There come crises in the battle when the turn of the tide wavers.
He knows when a bit of special praying is needed to turn the tide and
bring victory. So there needs to be special seasons of persistent prayer,
a continuing until victory is assured. Obey His promptings. Sometimes
there comes an impulse to pray, or to ask another to pray. And we think,
"Why, I have just been praying," \textit{or}, "he does pray about this anyway. It
is not necessary to pray again. I do not just like to suggest it." Better
obey the impulse quietly, with fewest words of explanation to the other
one concerned, or no words beyond simply the request.

Let Him, this wondrous Holy Spirit teach you how to pray. It will take
time. You may be a bit set in your way, but if you will just yield and
patiently wait, He will teach what to pray, suggest definite things, and
often the very language of prayer.

You will notice that the chief purpose of these four suggestions is to
learn God's will. The quiet place, the quiet time, the Book, the
Spirit--this is the schoolroom as Andrew Murray would finely put it. Here
we learn His will. Learning that makes one eager to have it done, and
breathes anew the longing prayer that it may be done.

There is a fine word much used in the Psalms, and in Isaiah for this sort
of thing--\textit{waiting}. Over and over again that is the word used for that
contact with God which reveals to us His will, and imparts to us anew His
desires. It is a word full of richest and deepest meaning. Waiting is not
an occasional nor a hurried thing. It means \textit{steadfastness}, that is
holding on; \textit{patience}, that is holding back; \textit{expectancy}, that is
holding the face up to see; \textit{obedience}, that is holding one's self in
readiness to go or do; it means \textit{listening}, that is holding quiet and
still so as to hear.



\underline{The Power of a Name.}


The fifth suggestion has already been referred to, but should be repeated
here. Prayer must be \textit{in Jesus' name}. The relationship of prayer is
through Jesus. And the prayer itself must be offered in His name, because
the whole strength of the case lies in Jesus. I recall distinctly a
certain section of this country where I was for awhile, and very rarely
did I hear Jesus' name used in prayer. I heard men, that I knew must be
good men, praying in church, in prayer-meeting and elsewhere with no
mention of Jesus. Let us distinctly bear in mind that we have no standing
with God except through Jesus.

If the keenest lawyer of London, who knew more of American law, and of
Illinois statute and of Chicago ordinance--suppose such a case--were to
come here, could he plead a case in your court-house? you know he could
not. He would have no legal standing here. Now you and I have no standing
at yonder bar. We are disbarred through sin. Only as we come through one
who has recognized standing there can we come.

But turn that fact around. As we do come in Jesus' name, it is the same as
though Jesus prayed. It is the same as though--let me be saying it very
softly so it may seem very reverent--as though Jesus put His arm in yours
and took you up to the Father, and said, "Father, here is a friend of
mine; we're on good terms. Please give him anything he asks, for My sake."
And the Father would quickly bend over and graciously say, "What'll you
have? You may have anything you ask when My Son asks for it." That is the
practical effect of asking in Jesus' name.

But I am very, very clear of this, and I keep swinging back to it that in
the ultimate analysis the force of using Jesus' name is that He is the
victor over the traitor prince. Prayer is repeating the Victor's name into
the ears of Satan and insisting upon his retreat. As one prays
persistently in Jesus' name, the evil one must go. Reluctantly, angrily,
he must loosen his clutches, and go back.



\underline{The Birthplace of Faith.}


The sixth suggestion is a familiar one, and yet one much misunderstood.
Prayer must be \textit{in faith}. But please note that faith here is not
believing that God \textit{can}, but that He \textit{will}. It is kneeling and making
the prayer, and then saying, "Father, I thank Thee for this; that it will
be so, I thank Thee." Then rising and going about your duties, saying,
"that thing is settled." Going again and again, and repeating the prayer
with the thanks, and then saying as you go off, "that matter is assured."
Not going repeatedly to persuade God. But because prayer is the deciding
factor in a spirit conflict and each prayer is like a fresh blow between
the eyes of the enemy, a fresh broadside from your fleet upon the fort.

"Well," some one will say, "now you are getting that keyed up rather high.
Can we all have faith like that? Can a man \textit{make} himself believe?" There
should be no unnatural mechanical insisting that you do believe. Some
earnest people make a mistake there. And we will not all have faith like
that. That is quite true, and I can easily tell you why. The faith that
believes that God \textit{will} do what you ask is not born in a hurry; it is not
born in the dust of the street, and the noise of the crowd. But I can tell
where that faith will have a birthplace and keep growing stronger: in
every heart that takes quiet time off habitually with God, and listens to
His voice in His word. Into that heart will come a simple strong faith
that the thing it is led to ask shall be accomplished.

That faith has four simple characteristics. It is \textit{intelligent}. It finds
out what God's will is. Faith is never contrary to reason. Sometimes it is
a bit higher up; the reasoning process has not yet reached up to it.
Second, it is \textit{obedient}. It fits its life into God's will. There is apt
to be a stiff rub here all the time. Then it is \textit{expectant}. It looks out
for the result. It bows down upon the earth, but sends a man to keep an
eye on the sea. And then it is \textit{persistent}. It hangs on. It says, "Go
again seven times; seventy times seven." It reasons that having learned
God's will, and knowing that He does not change, the delay must be caused
by the third person, the enemy, and that stubborn persistence in the
Victor's name routs him, and leaves a clear field.




\chapter{The Listening Side of Prayer}



\underline{A Trained Ear.}


In prayer the ear is an organ of first importance. It is of equal
importance with the tongue, but must be named first. For the ear leads the
way to the tongue. The child hears a word before it speaks it. Through the
ear comes the use of the tongue. Where the faculties are normal the tongue
is trained only through the ear. This is nature's method. The mind is
moulded largely through the ear and eye. It reveals itself, and asserts
itself largely through the tongue. What the ear lets in, the mind works
over, and the tongue gives out.

This is the order in Isaiah's fiftieth chapter[32] in those words,
prophetic of Jesus. "The Lord God hath given me the tongue of them that
are taught.... He wakeneth my ear to hear as they that are taught." Here
the taught tongue came through the awakened ear. One reason why so many of
us do not have taught tongues is because we give God so little chance at
our ears.

It is a striking fact that the men who have been mightiest in prayer have
known God well. They have seemed peculiarly sensitive to Him, and to be
overawed with the sense of His love and His greatness. There are three of
the Old Testament characters who are particularly mentioned as being
mighty in prayer. Jeremiah tells that when God spoke to him about the deep
perversity of that nation He exclaimed, "Though Moses and Samuel stood
before Me My heart could not be towards this people."[33] When James wants
an illustration of a man of prayer for the scattered Jews, he speaks of
Elijah, and of one particular crisis in his life, the praying on Carmel's
tip-top. These three men are Israel's great men in the great crises of its
history. Moses was the maker and moulder of the nation. Samuel was the
patient teacher who introduced a new order of things in the national life.
Elijah was the rugged leader when the national worship of Jehovah was
about to be officially overthrown. These three men, the maker, the
teacher, the emergency leader are singled out in the record as peculiarly
men of prayer.

Now regarding these men it is most interesting to observe what \textit{listeners}
they were to God's voice. Their ears were trained early and trained long,
until great acuteness and sensitiveness to God's voice was the result.
Special pains seem to have been taken with the first man, the nation's
greatest giant, and history's greatest jurist. There were two distinct
stages in the training of his ears. First there were the forty years of
solitude in the desert sands, alone with the sheep, and the stars,
and--God. His ears were being trained by silence. The bustle and confusion
of Egypt's busy life were being taken out of his ears. How silent are
God's voices. How few men are strong enough to be able to endure silence.
For in silence God is speaking to the inner ear.

    "Let us then labour for an inward stillness--
    An inward stillness and an inward healing;
    That perfect silence where the lips and heart
    Are still, and we no longer entertain
    Our own imperfect thoughts and vain opinions,
    But God alone speaks in us, and we wait
    In singleness of heart, that we may know
    His will, and in the silence of our spirits,
    That we may do His will, and do that only."[34]

A gentleman was asked by an artist friend of some note to come to his
home, and see a painting just finished. He went at the time appointed, was
shown by the attendant into a room which was quite dark, and left there.
He was much surprised, but quietly waited developments. After perhaps
fifteen minutes his friend came into the room with a cordial greeting, and
took him up to the studio to see the painting, which was greatly admired.
Before he left the artist said laughingly, "I suppose you thought it queer
to be left in that dark room so long." "Yes," the visitor said. "I did."
"Well," his friend replied, "I knew that if you came into my studio with
the glare of the street in your eyes you could not appreciate the fine
colouring of the picture. So I left you in the dark room till the glare
had worn out of your eyes."

The first stage of Moses' prayer-training was wearing the noise of Egypt
out of his ears so he could hear the quiet fine tones of God's voice. He
who would become skilled in prayer must take a silence course in the
University of Arabia. Then came the second stage. Forty years were
followed by forty days, twice over, of listening to God's speaking voice
up in the mount. Such an ear-course as that made a skilled famous
intercessor.

Samuel had an earlier course than Moses. While yet a child before his ears
had been dulled by earth sounds they were tuned to the hearing of God's
voice. The child heart and ear naturally open upward. They hear easily and
believe readily. The roadway of the ear has not been beaten down hard by
much travel. God's rains and dews have made it soft, and impressionable.
This child's ear was quickly trained to recognize God's voice. And the
tented Hebrew nation soon came to know that there was a man in their midst
to whom God was talking. O, to keep the heart and inner ear of a child as
mature years come!

Of the third of these famous intercessors little is known except of the
few striking events in which he figured. Of these, the scene that finds
its climax in the opening on Carmel's top of the rain-windows, occupies by
far the greater space. And it is notable that the beginning of that long
eighteenth chapter of first Kings which tells of the Carmel conflict
begins with a message to Elijah from God: "The word of the Lord came to
Elijah: ... I will send rain upon the earth." That was the foundation of
that persistent praying and sevenfold watching on the mountaintop. First
the ear heard, then the voice persistently claimed, and the eye
expectantly looked. First the voice of God, then the voice of man. That is
the true order. Tremendous results always follow that combination.



\underline{Through the Book to God.}


With us the training is of the \textit{inner} ear. And its first training, after
the early childhood stage is passed, must usually be through the eye. What
God has spoken to others has been written down for us. We hear through our
eyes. The eye opens the way to the inner ear. God spoke in His word. He is
still speaking in it and through it. The whole thought here is to get \textit{to
know God.} He reveals Himself in the word that comes from His own lips,
and through His messengers' lips. He reveals Himself in His dealings with
men. Every incident and experience of these pages is a mirror held up to
God's face. In them we may come to see Him.

This is studying the Bible not for the Bible's sake but for the purpose of
knowing God. The object aimed at is not the Book but the God revealed in
the Book. A man may go to college and take lectures on the English Bible,
and increase his knowledge, and enrich his vocabulary, and go away with
utterly erroneous ideas of God. He may go to a law school and study the
codes of the first great jurist, and get a clear understanding and firm
grasp of the Mosaic enactments, as he must do to lay the foundation of
legal training, yet he may remain ignorant of God.

He may even go to a Bible school, and be able to analyze and synthesize,
give outlines of books, and contents of chapters and much else of that
invaluable and indispensable sort of knowledge and yet fail to understand
God and His marvellous love-will. It is not the Book with which we are
concerned here but the God through the Book. Not to learn truth but
through truth to know Him who is Himself the Truth.

There is a fascinating bit of story told of one of David's mighty men.[35]
One day there was a sudden attack upon the camp by the Philistines when
the fighting men were all away. This man alone was there. The Philistines
were the traditional enemy. The very word "Philistines" was one to strike
terror to the Hebrew heart. But this man was reckoned one of the first
three of David's mighty men because of his conduct that day. He quietly,
quickly gripped his sword and fought the enemy single-handed. Up and down,
left and right, hip and thigh he smote with such terrific earnestness and
drive that the enemy turned and fled. And we are told that the muscles of
his hand became so rigid around the handle of his sword that he could not
tell by the feeling where his hand stopped, and the sword began. Man and
sword were one that day in the action of service against the nation's
enemy. When we so absorb this Book, and the Spirit of Him who is its life
that people cannot tell the line of division between the man, and the God
within the man, then shall we have mightiest power as God's intercessors
in defeating the foe. God and man will be as one in the action of service
against the enemy.



\underline{A Spirit Illumined Mind.}


I want to make some simple suggestions for studying this Book so as to get
to God through it. There will be the emphasis of doubling back on one's
tracks here. For some of the things that should be said have already been
said with a different setting. First there must be the \textit{time} element.
One must get at least a half hour daily when the mind is fresh. A tired
mind does not readily \textit{absorb}. This should be persisted in until there is
a habitual spending of at least that much time daily over the Book, with a
spirit at leisure from all else, so it can take in. Then the time should
be given to \textit{the Book itself}. If other books are consulted and read as
they will be let that be \textit{after} the reading of this Book. Let God talk to
you direct, rather than through somebody else. Give Him first chance at
your ears. This Book in the central place of your table, the others
grouped about it. First time given to it.

A third suggestion brings out the circle of this work. \textit{Read prayerfully.}
We learn how to pray by reading prayerfully. This Book does not reveal its
sweets and strength to the keen mind merely, but to the Spirit enlightened
mind. All the mental keenness possible, \textit{with the bright light of the
Spirit's illumination}--that is the open sesame. I have sometimes sought
the meaning of some passage from a keen scholar who could explain the
orientalisms, the fine philological distinctions, the most accurate
translations, and all of that, who yet did not seem to know the simple
spiritual meaning of the words being discussed. And I have asked the same
question of some old saint of God, who did not know Hebrew from a hen's
tracks, but who seemed to sense at once the deep spiritual truth taught.
The more knowledge, the keener the mind, the better \textit{if} illumined by the
Spirit that inspired these writings.

There is a fourth word to put in here. We must read \textit{thoughtfully}.
Thoughtfulness is in danger of being a lost art. Newspapers are so
numerous, and literature so abundant, that we are becoming a bright, but a
\textit{not thoughtful} people. Often the stream is very wide but has no depth.
Fight shallowness. Insist on reading thoughtfully. A very suggestive word
in the Bible for this is "\textit{meditate}." Run through and pick out this word
with its variations. The word underneath that English word means to
mutter, as though a man were repeating something over and over again, as
he turned it over in his mind. We have another word, with the same
meaning, not much used now--ruminate. We call the cow a ruminant because
she chews the cud. She will spend hours chewing the cud, and then give us
the rich milk and cream and butter which she has extracted from her food.
That is the word here--ruminate. Chew the cud, if you would get the
richest cream and butter here.

And it is remarkable how much chewing this Book of God will stand, in
comparison with other books. You chew a while on Tennyson, or Browning, or
Longfellow. And I am not belittling these noble writings. I have my own
favourite among these men. But they do not yield the richest and yet
richer cream found here. This Book of God has stood more of that sort of
thing than any other, yet it is the freshest book to be found to-day. You
read a passage over the two hundredth time and some new fine bit of
meaning comes that you had not suspected to be there.

There is a fifth suggestion, that is easier to make than to follow. \textit{Read
obediently.} As the truth appeals to your conscience \textit{let it change your
habit and life}.

    "Light obeyed, increased light:
    Light resisted, bringeth night
    Who shall give us power to choose
    If the love of light we lose?"[36]

Jesus gives the law of knowledge in His famous words, "If any man willeth
to do His will he shall know of the teaching."[37] If we do what we know
to do, we will know more. If we know to do, and hesitate and hold back,
and do not obey, the inner eye will surely go blind, and the sense of
right be dulled and lost. Obedience to truth is the eye of the mind.



\underline{Wide Reading.}


Then one needs to have a \textit{plan} of reading. A consecutive plan gathers up
the fragments of time into a strong whole. Get a good plan, and stick to
it. Better a fairly good plan faithfully followed, than the best plan if
used brokenly or only occasionally. Probably all the numerous methods of
study may be grouped under three general heads, wide reading, topical
study, and textual. We all do some textual study in a more or less small
way. Digging into a sentence or verse to get at its true and deep meaning.
We all do some topical study probably. Gathering up statements on some one
subject, studying a character. The more pretentious name is Biblical
Theology, finding and arranging all that is taught in the whole range of
the Bible on any one theme.

But I want especially to urge \textit{wide reading}, as being the basis of all
study. It is the simple, the natural, the scientific method. It is adapted
to all classes of persons. I used to suppose it was suited best to college
students, and such; but I was mistaken. It is \textit{the} method of all for all.
It underlies all methods of getting a grasp of this wonderful Book, and so
coming to as full and rounded an understanding of God as is possible to
men down here.

By wide reading is meant a \textit{rapid reading through} regardless of verse,
chapter, or book divisions. Reading it as \textit{a narrative}, a story. As you
would read any book, "The Siege of Pekin," "The Story of an Untold Love,"
to find out the story told, and be able to tell to another. There will be
a reverence of spirit with this book that no other inspires, but with the
same intellectual method of running through to see what is here. No book
is so fascinating as the Bible when read this way. The revised version is
greatly to be preferred here simply because it is a \textit{paragraph} version.
It is printed more like other books. Some day its printed form will be yet
more modernized, and so made easier to read.

To illustrate, begin at the first of Genesis, and read rapidly through \textit{by
the page}. Do not try to understand all. You will not. Never mind that
now. Just push on. Do not try to remember all. Do not think about that.
Let stick to you what will. You will be surprised to find how much will.
You may read ten or twelve pages in your first half hour. Next time start
in where you left off. You may get through Genesis in three or four times,
or less or more, depending on your mood, and how fast your habit of
reading may be. You will find a whole Bible in Genesis. A wonderfully
fascinating book this Genesis. For love stories, plotting, swift action,
beautiful language it more than matches the popular novel.

But do not stop at the close of Genesis. Push on into Exodus. The
connection is immediate. It is the same book. And so on into Leviticus.
Now do not try to understand Leviticus the first time. You will not the
hundredth time perhaps. But you can easily group its contents: these
chapters tell of the offerings: these of the law of offerings: here is an
incident put in: here sanitary regulations: get the drift of the book. And
in it all be getting the picture of God--\textit{that is the one point}. And so
on through.

A second stage of this wide reading is fitting together the parts. You
know the arrangement of our Bible is not chronological wholly, but
topical. The Western mind is almost a slave to chronological order. But
the Oriental was not so disturbed. For example, open your Bible to the
close of Esther, and again at the close of Malachi. This from Genesis to
Esther we all know is the historical section: and this second section the
poetical and prophetical section. There is some history in the prophecy,
and some prophecy and poetry in the historical part. But in the main this
first is historical, and this second poetry and prophecy. These two parts
belong together. This first section was not written, and then this second.
The second belongs in between the leaves of the first. It was taken out
and put by itself because the arrangement of the whole Book is topical
rather than chronological.

Now the second stage of wide reading is this: fit these parts together.
Fit the poetry and the prophecy into the history. Do it on your own
account, as though it had never been done. It has been done much better
than you will do it. And you will make some mistakes. You can check those
up afterwards by some of the scholarly books. And you cannot tell where
some parts belong. But meanwhile the thing to note is this: you are
absorbing the Book. It is becoming a part of you, bone of your bone, and
flesh of your flesh, mentally, and spiritually. You are drinking in its
spirit in huge draughts. There is coming a new vision of God, which will
transform radically the reverent student. In it all seek to acquire \textit{the
historical sense}. That is, put yourself back and see what this thing, or
this, meant to these men, as it was first spoken, under these immediate
circumstances.

And so push on into the New Testament. Do not try so much to fit the four
gospels into one connected story, dovetailing all the parts; but try
rather to get a clear grasp of Jesus' movements those few years as told by
these four men. Fit Paul's letters into the book of Acts, the best you
can. The best book to help in checking up here is Conybeare and Howson's
"Life and Letters of St. Paul." That may well be one of the books in your
collection.

You see at once that this is a method not for a month, nor for a year, but
for years. The topical and textual study grow naturally out of it. And
meanwhile you are getting an intelligent grasp of this wondrous classic,
you are absorbing the finest literature in the English tongue, and
infinitely better yet, you are breathing into your very being a new, deep,
broad, tender conception of \textit{God}.



\underline{A Mirror Held up to God's Face.}


It is simply fascinating too, to find what light floods these pages as
they are read back in their historical setting, so far as that is
possible. For example turn to the third Psalm, fifth verse,

    "I laid me down and slept;
    I awaked; for the Lord sustaineth me."

I was brought up in an old-fashioned church where that was sung. I knew it
by heart. As a boy I supposed it meant that night-time had come, and David
was sleepy; he had his devotions, and went to bed, and had a good night's
sleep. That was all it had suggested to me.

But on my first swing through of the wide reading, my eye was caught, as
doubtless yours has often been, by the inscription at the beginning of the
psalm: "A psalm of David, \textit{when he fled from Absalom his son}." Quickly I
turned back to Second Samuel to find that story. And I got this picture.
David, an old white-haired man, hurrying one day, barefooted, out of his
palace, and his capital city, with a few faithful friends, fleeing for his
life, because Absalom his favourite son was coming with the strength of
the national army to take the kingdom, and his own father's life. And that
night as the king lay down to try to catch some sleep, it was upon the
bare earth, with only heaven's blue dome for a roof. And as he lay he
could almost hear the steady tramp, tramp of the army, over the hills,
seeking his throne and his life. Let me ask you, honestly now; do you
think you would have slept much that night? I fear I would have been
tempted sorely to lie awake thinking: "here I am, an old man, driven from
my kingdom, and my home, by my own boy, that I have loved better than my
own life." Do you think \textit{you} would have slept much? Tell me.

But David speaking of that night afterwards wrote this down:--"I laid me
down, and \textit{slept; I awaked}; (the thought is, I awaked \textit{refreshed}) for
the Lord sustaineth me." And I thought, as first that came to me, "I never
will have insomnia again: I'll trust." And so you see a lesson of trust in
God came, in my wide reading, out of the historical setting, that greatly
refreshed and strengthened, and that I have never forgotten. What a God,
to give sleep under such circumstances!

A fine illustration of this same thing is found in the New Testament in
Paul's letter to the Philippians. At one end of that epistle is this
scene: Paul, lying in the inner damp cell of a prison, its small creeping
denizens familiarly examining this newcomer, in the darkness of midnight,
his back bleeding from the stripes, his bones aching, and his feet fast in
the stocks. That is one half of the historical setting of this book. And
here is the other half: Paul, a prisoner in Rome. If he tries to ease his
body by changing his position, swinging one limb over the other, a chain
dangling at his ankle reminds him of the soldier by his side. As he picks
up a quill to put a last loving word out of his tender heart for these old
friends, a chain pulls at his wrist. That is Philippians, the prison
epistle, resounding with clanking chain.

What is the keyword of the book, occurring oftener than any other?
Patience? Surely that would be appropriate. Long-suffering? Still more
fitting would that seem. But, no, the keyword stands in sharpest contrast
to these surroundings. Paul used clouds to make the sun's shining more
beautiful. Joy, rejoice, rejoicing, is the music singing all the way
through these four chapters. What a wondrous Master, this Jesus, so to
inspire His friend doing His will!

Every incident and occurrence of these pages becomes a mirror held up to
God's face that we may see how wondrous He is.

    "Upon Thy Word I rest
      Each pilgrim day.
    This golden staff is best
      For all the way.
    What Jesus Christ hath spoken,
      Cannot be broken!

    "Upon Thy Word I rest;
      So strong, so sure,
    So full of comfort blest,
      So sweet, so pure:
    The charter of salvation:
      Faith's broad foundation.

    "Upon Thy Word I stand:
      That cannot die.
    Christ seals it in my hand.
      He cannot lie.
    Thy Word that faileth never:
      Abiding ever."[38]




\chapter{Something about God's Will in Connection With Prayer}



\underline{He Came to His Own.}


The purpose of prayer is to get God's will done. What a stranger God is in
His own world! Nobody is so much slandered as He. He comes to His own, and
they keep Him standing outside the door, like a pilgrim of the night,
staff in hand, while they peer suspiciously at Him through the crack of
the hinges.

Some of us shrink back from making a full surrender of life to God. And if
the real reason were known it would be found to be that we are \textit{afraid} of
God. We fear He will put something bitter in the cup, or some rough thing
in the road. And without doubt the reason we are afraid of God is because
we do not \textit{know} God. The great prayer of Jesus' heart that night with the
eleven was, "that they may \textit{know} Thee the only true God, and Jesus
Christ whom Thou didst send."

To understand God's will we must understand something of His character,
Himself. There are five common every-day words I want to bring you to
suggest something of who God is. They are familiar words, in constant use.
The first is the word \textit{father}. "Father" stands for strength, loving
strength. A father plans, and provides for, and protects his loved ones.
All fathers are not good. How man can extract the meaning out of a fine
word, and use the word without its meaning. If you will think of the
finest father ever you knew that anybody ever had; think of him now. Then
remember this, God is a father, only He is so much finer a father than the
finest father you ever knew of. And His will for your \textit{life}--I am not
talking about heaven, and our souls just now, that is in it too--His will
for your life down here these days is a father's will for the one most
dearly loved.

The second word is a finer word. Because woman is finer than man, and was
made, and meant to be, this second word is finer than the first. I mean
the word \textit{mother}. If father stands for strength, mother stands for
love,--great, patient, tender, fine-fibred, enduring love. What would she
not do for her loved one! Why, not unlikely she went down into the valley
of the shadow that that life might come; and did it gladly with the
love-light shining out of her eyes. Yes, and would do it again, that the
life may remain if need be. That is a mother. You think of the finest
mother ever you knew. And the suggestion brings the most hallowed memories
to my own heart. Then remember this: God is a mother, only He is so much
finer a mother than the finest mother you ever knew.

The references in scripture to God as a mother are numerous. "Under His
wings" is a mother figure. The mother-bird gathers her brood up under her
wings to feel the heat of her body, and for protection. The word mother is
not used for God in the Bible. I think it is because with God "father"
includes "mother." It takes more of the human to tell the story than of
the divine. With God, all the strength of the father and all the fine love
of the mother are combined in that word "father." And His will for us is a
mother's will, a wise loving mother's will for the darling of her heart.

The third word is \textit{friend}. I do not mean to use it in the cheaper
meaning. There is a certain kindliness of speech in which all
acquaintances are called friends. Tupper says, we call all men friends who
are not known to be enemies. But I mean to use the word in its finer
meaning. Here, a friend is one who loves you for your sake only and
steadfastly loves without regard to any return, even a return-love. The
English have a saying that you may fill a church with your acquaintances,
and not fill the pulpit seats with your friends. If you may have in your
life one or two real friends you are very wealthy. If you will think for a
moment of the very best friend you ever knew anybody to have. Then
remember this: God is a friend. Only He is ever so much better a friend
than the best friend you ever knew of. And the plan He has thought out for
your life is such a one as that word would suggest.

The fourth word, I almost hesitate to use, yet I am sure I need not here.
The hesitancy is because the word and its relationship are spoken of
lightly, frivolously, so much, even in good circles. I mean that rare fine
word \textit{lover}. Where two have met, and acquaintance has deepened into
friendship, and that in turn into the holiest emotion, the highest
friendship. What would he not do for her! She becomes the new human centre
of his life. In a good sense he worships the ground she treads upon. And
she--she will leave wealth for poverty if only so she may be with him in
the coming days. She will leave home and friends, and go to the ends of
the earth if his service calls him there. You think of the finest lover,
man or woman, you ever knew anybody to have. Then remember this, and let
me say it in soft, reverent tones, God is a lover--shall I say in yet more
reverent voice, a sweetheart-lover. Only He is so much finer a lover than
the finest lover you ever knew of. And His will, His plan for your life
and mine--it hushes my heart to say it--is a lover's plan for his only
loved one.

The fifth word is this fourth word a degree finer spun, a stage farther
on, and higher up, the word \textit{husband}. This is the word on the man side
for the most hallowed relationship of earth. This is the lover
relationship in its perfection stage. With men husband is not always a
finer word than lover. The more's the pity. How man does cheapen God's
plan of things; leaves out the kernel, and keeps only an empty shell
sometimes. In God's thought a husband is a lover \textit{plus}. He is all that
the finest lover is, and more; more tender, more eager, more thoughtful.
Two lives are joined, and begin living one life. Two wills, yet one. Two
persons, yet one purpose. Duality in unity. Will you call to mind for a
moment the best husband you ever knew any woman to have. Then remember
this that God is a husband; only He is an infinitely more thoughtful
husband than any you ever knew. And His will for your life is a husband's
will for his life's friend and companion.

Now, please, do not \textit{you} take one of these words, and say, "I like that";
and \textit{you} another and say, "That conception of God appeals to me," and
\textit{you} another. How we do whittle God down to our narrow conceptions! You
must take all five words, and think the finest meaning into each, and then
put them all together, to get a close up idea of God. He is all that, \textit{and
more}.

You see God is so much that it takes a number of earth's relationships put
together to get a good suggestion of what He is. He is a father, a
mother, a friend, a lover, a husband. I have not brought book, and
chapter, and verse. But you know I could spend a long time with you
reading over the numerous passages giving these conceptions of God.

And God's will for us is the plan of such a God as that. It includes the
body, health and strength; the family and home matters; money and business
matters; friendships, including the choice of life's chief friend; it
includes service, what service and where; and constant guidance; it
includes the whole life, and the world of lives. All this He has thought
into, lovingly, carefully. Does a wise mother think of her child's needs
into the details, the necessities and the loving extras? That is God.



\underline{The One Purpose of Prayer.}


Now, the whole thought in prayer is to get the will of a God like that
done in our lives and upon this old earth. The greatest prayer any one can
offer is, "Thy will be done." It will be offered in a thousand different
forms, with a thousand details, as needs arise daily. But every true
prayer comes under those four words. There is not a good desirable thing
that you have thought of that He has not thought of first, and probably
with an added touch not in your thought. Not to grit your teeth and lock
your jaw and pray for grace to say, "Thy will be \textit{endured}: it is bitter,
but I must be resigned; that is a Christian grace; Thy will be
\textit{endured}." Not that, please. Do not slander God like that. There is a
superficial idea among men that charges God with many misfortunes and ills
for which He is not at all responsible. He is continually doing the very
best that can be done under the circumstances for the best results. He has
a bad mixture of stubborn warped human wills to deal with. With infinite
patience and skill and diplomacy and success too He is ever working at the
tangled skein of human life, through the human will.

It may help us here to remember that God has a first and a second will for
us: a first choice and a second. He always prefers that His first will
shall be accomplished in us. But where we will not be wooed up to that
height, He comes down to the highest level we will come up to, and works
with us there. For instance, God's first choice for Israel was that He
Himself should be their king. There was to be no human, visible king, as
with the surrounding nations. He was to be their king. They were to be
peculiar in this. But to Samuel's sorrow and yet more to God's, they
insisted upon a king. And so God gave them a king. And David the great
shepherd-psalmist-king was a man after God's own heart, and the world's
Saviour came of the Davidic line. God did His best upon the level they
chose and a great best it was. Yet the human king and line of kings was
not God's first will, but a second will yielded to because the first
would not be accepted. God is ever doing the best for human lives that can
be done through the human will.

His first will for our bodies, without doubt, is that there should be a
strong healthy body for each of us. But there is a far higher thing being
aimed at in us than that. And with keen pain to His own heart, He oft
times permits bodily weakness and suffering because in the conditions of
our wills only so can these higher and highest things be gotten at. And
where the human will comes into intelligent touch with Himself, and the
higher can so be reached, with great gladness and eagerness the bodily
difficulty is removed by Him.

There are two things, at least, that modify God's first will for us. First
of all the degree of our intelligent willingness that He shall have His
full sway. And second, the circumstances of one's life. Each of us is the
centre of a circle of people, an ever changing circle. If we be in touch
with Him God is speaking through each of us to his circle. Our experiences
with God: His dealings with us, under the varying circumstances are a part
of His message to that circle. God is trying to win men. It takes
marvellous diplomacy on His part. And God is a wondrous tactician.
But--very reverently--He is a needy God. He needs us to help Him, each in
his circle. We must be perfectly willing to have His will done; and more,
we must trust Him to know what is best to do in us and with us in the
circle of our circumstances. God is a great economist. He wastes no
forces. Every bit is being conserved towards the great end in view.

There may be a false submission to His supposed will in some affliction; a
not reaching out after \textit{all} that He has for us. And at the other swing of
the pendulum there may be a sort of \textit{logical praying} for some desirable
thing because a friend tells us we should claim it. By logical praying I
mean the studying of a statement of God's word, and possibly some one's
explanation of it, and hearing or knowing how somebody else has claimed a
certain thing through that statement and then concluding that therefore we
should so claim. The trouble with that is that it stops too soon. Praying
in the Spirit as opposed to logical praying is doing this logical
thinking: \textit{then} quietly taking all to God, to learn what His will is for
\textit{you}, under your circumstances, and in the circle of people whom He
touches through you.



\underline{The Spirit's Prayer Room.}


There is a remarkable passage in Paul's Roman letter about prayer and
God's will.[39] "And in like manner the Spirit also helpeth our infirmity:
for we know not how to pray as we ought; but the Spirit Himself maketh
intercession for us with groanings which cannot be uttered; and He that
searcheth the hearts knoweth what is the mind of the Spirit, that He
maketh intercession for the saints according to the will of God."

Please notice: these words connect back with the verses ending with verse
seventeen. Verses eighteen to twenty-five are a parenthesis. As the Spirit
within breathes out the "Father" cry of a child, which is the prayer-cry,
so He helps us in praying. It is our infirmity that we do not know how to
pray \textit{as we ought}. There is willingness and eagerness too. No bother
there. But a lack of knowledge. We don't know how. But the Spirit knows
how. He is the Master-prayor. He knows God's will perfectly. He knows what
best to be praying under all circumstances. And He is within you and me.
He is there as a prayer-spirit. He prompts us to pray. He calls us away to
the quiet room to our knees. He inclines to prayer wherever we are. He is
thinking thoughts that find no response in us. They cannot be expressed in
our lips for they are not in our thinking. He prays with an intensity
quite beyond the possibility of language to express. And the
heart-searcher--God listening above--knows fully what this praying Spirit
is thinking within me, and wordlessly praying, for they are one. He
recognizes His own purposes and plans being repeated in this man down on
the earth by His own Spirit.

And the great truth is that the Spirit within us prays God's will. He
teaches us God's will. He teaches us how to pray God's will. And He
Himself prays God's will in us. And further that He seeks to pray God's
will--that is to pray for the thing God has planned--in us before we have
yet reached up to where we know ourselves what that will is.

We should be ambitious to cultivate a healthy sensitiveness to this
indwelling Spirit. And when there comes that quick inner wooing away to
pray let us faithfully obey. Even though we be not clear what the
particular petition is to be let us remain in prayer while He uses us as
the medium of His praying.

Oftentimes the best prayer to offer about some friend, or some particular
thing, after perhaps stating the case the best we can is this: "Holy
Spirit, be praying in me the thing the Father wants done. Father, what the
Spirit within me is praying, that is my prayer in Jesus' name. Thy will,
what Thou art wishing and thinking, may that be fully done here."



\underline{How to Find God's Will.}


We should make a study of God's will. We ought to seek to become skilled
in knowing His will. The more we know Him the better shall we be able to
read intelligently His will.

It may be said that God has two wills for each of us, or, better, there
are two parts to His will. There is His will of grace, and His will of
government. His will of grace is plainly revealed in His Word. It is that
we shall be saved, and made holy, and pure, and by and by glorified in his
own presence. His will of government is His particular plan for my life.
God has every life planned. The highest possible ambition for a life is to
reach God's plan. He reveals that to us bit by bit as we need to know. If
the life is to be one of special service He will make that plain, what
service, and where, and when. Then each next step He will make plain.

Learning His will here hinges upon three things, simple enough but
essential. I must keep \textit{in touch} with Him so He has an open ear to talk
into. I must \textit{delight} to do His will, \textit{because it is His}. The third
thing needs special emphasis. Many who are right on the first two stumble
here, and sometimes measure their length on the ground. \textit{His Word must be
allowed to discipline my judgment as to Himself and His will}. Many of us
stumble on number one and on number two. And very many willing earnest men
sprawl badly when it comes to number three. The bother with these is the
lack of a disciplined judgment about God and His will. If we would
prayerfully \textit{absorb} the Book, there would come a better poised judgment.
We need to get a broad sweep of God's thought, to breathe Him in as He
reveals Himself in this Book. The meek man--that is the man willing to
yield his will to a higher will--will He guide in his judgment, that is,
in his mental processes.[40]

This is John's standpoint in that famous passage in his first epistle.[41]
"And this is the boldness that we have towards Him, that, if we ask
anything according to His will, He heareth us: and if we know that He
heareth us whatsoever we ask, we know that we have the petitions that we
have asked of Him." These words dovetail with great nicety into those
already quoted from Paul in the eighth of Romans. The whole supposition
here is that we have learned His will about the particular matter in hand.
Having gotten that footing, we go to prayer with great boldness. For if He
wants a thing and I want it and we join--that combination cannot be
broken.




\chapter{May we Pray With Assurance for the Conversion of Our Loved Ones}



\underline{God's Door into a Home.}


The heart of God hungers to redeem the world. For that He gave His own,
only Son though the treatment He received tore that father's heart to the
bleeding. For that He sent the Holy Spirit to do in men what the Son had
done for them. For that He placed in human hands the mightiest of all
forces--prayer, that so we might become partners with Him.

For that too He set man in the relationships of kinship and friendship. He
wins men through men. Man is the goal, and he is also the road to the
goal. Man is the object aimed at. And he is the medium of approach,
whether the advance be by God or by Satan. God will not enter a man's
heart without his consent, and Satan \textit{can}not. God would reach men through
men, and Satan must. And so God has set us in the strongest relation that
binds men, the relation of love, that He may touch one through another.
Kinship is a relation peculiar to man, and to the earth.

I have at times been asked by some earnest sensitive persons if it is not
selfish to be especially concerned for one's own, over whom the heart
yearns much, and the prayer offered is more tender and intense and more
frequent. Well, if \textit{you} do not pray for them who will? Who \textit{can} pray for
them with such believing persistent fervour as you! God has set us in the
relationship of personal affection and of kinship for just such a purpose.
He binds us together with the ties of love that we may be concerned for
each other. If there be but one in a home in touch with God, that one
becomes God's door into the whole family.

Contact means opportunity, and that in turn means responsibility. The
closer the contact the greater the opportunity and the greater too the
responsibility. Unselfishness does not mean to exclude one's self, and
one's own. It means right proportions in our perspective. Humility is not
whipping one's self. It is forgetting one's self in the thought of others.
Yet even that may be carried to a bad extreme. Not only is it not selfish
so to pray, it is a part of God's plan that we should so pray. I am most
responsible for the one to whom I am most closely related.



\underline{A Free Agent Enslaved.}


One of the questions that is more often asked in this connection than any
other perhaps is this: may we pray with assurance for the conversion of
our loved ones? No question sets more hearts in an audience to beating
faster than does that. I remember speaking in the Boston noonday meeting,
in the old Broomfield Street M. E. Church on this subject one week.
Perhaps I was speaking rather positively. And at the close of the meeting
one day a keen, cultured Christian woman whom I knew came up for a word.
She said, "I do not think we can pray like that." And I said, "Why not?"
She paused a moment, and her well-controlled agitation revealed in eye and
lip told me how deeply her thoughts were stirred. Then she said quietly,
"I have a brother. He is not a Christian. The theatre, the wine, the club,
the cards--that is his life. And he laughs at me. I would rather than
anything else that my brother were a Christian. But," she said, and here
both her keenness and the training of her early teaching came in, "I do
not think I can pray positively for his conversion, for he is a free
agent, is he not? And God will not save a man against his will."

I want to say to you to-day what I said to her. Man \textit{is} a free agent, to
use the old phrase, so far as God is concerned; utterly, wholly free.
\textit{And}, he is the most enslaved agent on the earth, so far as sin, and
selfishness and prejudice are concerned. The purpose of our praying is not
to force or coerce his will; never that. It is to \textit{free} his will of the
warping influences that now twist it awry. It is to get the dust out of
his eyes so his sight shall be clear. And once he is free, able to see
aright, to balance things without prejudice, the whole probability is in
favour of his using his will to choose the only right.

I want to suggest to you the ideal prayer for such a one. It is an
adaptation of Jesus' own words. It may be pleaded with much variety of
detail. It is this: deliver him from the evil one; and work in him \textit{Thy
will} for him, by Thy power to Thy glory in Jesus, the Victor's name. And
there are three special passages upon which to base this prayer. First
Timothy, second chapter, fourth verse (American version), "God our
Saviour, who would have all men to be saved." That is God's will for your
loved one. Second Peter, third chapter, ninth verse, "not wishing (or
willing) that any should perish but that all should come to repentance."
That is God's will, or desire, for the one you are thinking of now. The
third passage is on our side who do the praying. It tells who may offer
this prayer with assurance. John, fifteenth chapter, seventh verse, "If ye
abide in Me, and My words abide in you, you ask what it is your will to
ask, and I will bring it to pass for you."

There is a statement of Paul's in second Timothy that graphically pictures
this:[42] "The Lord's servant must not strive "--not argue, nor
combat--"but be gentle towards all, apt to teach"--ready and skilled in
explaining, helping--"in meekness correcting (or, instructing) them that
oppose themselves; if peradventure God may give them repentance unto the
knowledge of the truth, and \textit{they may recover themselves out of the snare
of the devil}, having been taken captive by him unto his will."

That word "deliver" in this prayer, as used by Jesus, the word under our
English, has a picturesque meaning. It means \textit{rescue}. Here is a man taken
captive, and in chains. But he has become infatuated with his captor, and
is befooled regarding his condition. Our prayer is, "rescue him from the
evil one," and because Jesus is Victor over the captor, the rescue will
take place.

Without any doubt we may assure the conversion of these laid upon our
hearts by such praying. The prayer in Jesus' name drives the enemy off the
battle-field of the man's will, and leaves him free to choose aright.
There is one exception to be noted, a very, very rare exception. There may
be \textit{extreme} instances where such a prayer may not be offered; where the
spirit of prayer is withdrawn. But such are very rare and extreme, and the
conviction regarding that will be unmistakable beyond asking any
questions.

And I cannot resist the conviction--I greatly dislike to say this, I would
much rather not if I regarded either my own feelings or yours. But I
cannot resist the conviction--listen very quietly, so I may speak in
quietest tones--that there are people ... in that lower, lost world ...
who are there ... because some one failed to put his life in touch with
God, and pray.



\underline{The Place Where God is Not.}


Having said that much let me go on to say this further, and please let me
say it all in softest sobbing voice--there is a hell. There must be a
hell. You may leave this Bible sheer out of your reckoning in the matter.
Still there must be a place for which that word of ugliest associations is
the word to use. \textit{Philosophically} there must be a hell. That is the name
for the place where God is not; for the place where they will gather
together who insist on leaving God out. God out! There can be no worse
hell than that! God away! Man held back by no restraints!

I am very clear it is \textit{not} what men have pictured it to be. It is not
what my childish fancy saw and shrank from terrified. And, please let us
be very careful that we never consign anybody there, in our thinking or
speaking about them. When that life whose future might be questioned has
gone the most we can say is that we leave it with a God infinitely just
and the personification of love.

There has been in some quarters an unthinking consigning of persons to a
lost world. And there has been in our day a clean swing of the pendulum
to the other extreme. Both drifts are to be dreaded. Let us deal very
tenderly here, yet with a right plainness in our tenderness. We are to
warn men faithfully. We know the Book's plain teaching that these who
prefer to leave God out "shall go away." The going is of their own accord
and choice. Regarding particular ones we do not know and are best silent.
The grave is closing. Let us deal with the living.

One day at the close of the morning hour at a Bible conference in the
Alleghany Mountains a young woman came up for a moment's conversation. She
spoke about a friend, not a professing Christian, for whom she had prayed
much, and who had died unexpectedly. He had passed away during
unconsciousness, with no opportunity for exchange of words. She was much
agitated as the facts were recited, and then said as she finished, "he is
lost and in hell: and I can never pray again."

We talked quietly awhile and I gathered the following facts. He was of a
Christian family, perfectly familiar with the Bible, was a thoughtful man,
of outwardly correct life in the main, had talked about these matters with
others but had never either in conversation or more openly confessed
personal faith in Christ. He was not in good health. Then came the sudden
end. One other fact came out. She had prayed for his conversion for a long
time. She was herself an earnest Christian woman, solicitous for others.
There were four facts to go upon regarding him. He knew the way to God. He
was thoughtful. He had never openly accepted. Some one had prayed.

Can one \textit{know} anything certainly about that man's condition? There are
two sorts of knowledge, direct and inferential. I know there is such a
city as London for I have walked its streets. That is direct knowledge. I
know there is such a city as St. Petersburg because though I have never
been there, yet through my reading, pictures I have seen, and friends who
have been there I am clear of its existence to the point of \textit{knowledge}.
That is inferential knowledge.

Now regarding this man after he slipped from the grasp of his friends, I
have no direct knowledge. But I have very positive inferential knowledge
based upon these four facts. Three of the facts, namely, the first,
second, and fourth were favourable to the end desired. The third swings
neither way. The great dominant fact in the case is the fourth, and a
great and dominating fact it is in judging--some one in touch with God had
been persistently, believingly praying up to the time of the quick end.
That fact with the others gives strong inferential knowledge regarding the
man. It is sufficient to comfort a heart, and give one renewed faith in
praying for others.



\underline{Saving the Life.}


We cannot know a man's mental processes. This is surely true, that if in
the very last half-twinkling of an eye a man look up towards God
longingly, that look is the turning of the will to God. And that is quite
enough. God is eagerly watching with hungry eyes for the quick turn of a
human eye up to Himself. Doubtless many a man has so turned in the last
moment of his life when we were not conscious of his consciousness, nor
aware of the movements of his outwardly unconscious sub-consciousness. One
may be unconscious of outer things, and yet be keenly conscious towards
God.

At another of these summer gatherings this incident came to me. A man
seemingly of mature mind and judgment told me of a friend of his. That was
as close as I got to the friend himself. This friend was not a professing
Christian, was thrown from a boat, sank twice and perhaps three times, and
then was rescued, and after some difficulty resuscitated. He told
afterwards how swiftly his thoughts came as they are said to do to one in
such circumstances. He thought surely he was drowning, was quiet in his
mind, thought of God and how he had not been trusting Him, and in his
thought he prayed for forgiveness. He lived afterwards a consistent
Christian life. This illustrates simply the possibilities open to one in
his keen inner mental processes.

Here is surely enough knowledge to comfort many a bereft heart, and
enough too to make us pray persistently and believingly for loved ones
because of prayer's uncalculated and incalculable power. Be sure the
prayer-fact is in the case of \textit{your} friend, \textit{and in strong}.

Yet let us be wary, very wary of letting this influence us one bit
farther. That man is nothing less than a fool who presumes upon such
statements to resist God's gracious pleadings for his life. And on our
side, we must not fail to warn men lovingly, tenderly yet with plainness
of the tremendous danger of delay, in coming to God. A man may be so
stupefied at the close as to shut out of his range what has been suggested
here. And further even if a man's soul be saved he is responsible to God
for his life. We want men to \textit{live} for Jesus, and win others to Him. And
further, yet, reward, preferment, honour in God's kingdom depends upon
faithfulness to Him down here. Who would be saved by the skin of his
teeth!

The great fact to have burned in deep is that we may assure the coming to
God of our loved ones with their lives, as well as for their souls if we
will but press the battle.



\underline{Giving God a Clear Road for Action.}


Out in one of the trans-Mississippi states I ran across an illustration of
prayer in real life that caught me at once, and has greatly helped me in
understanding prayer.

Fact is more fascinating than fiction. If one could know what is going on
around him, how surprised and startled he would be. If we could get \textit{all}
the facts in any one incident, and get them colourlessly, and have the
judgment to sift and analyze accurately, what fascinating instances of the
power of prayer would be disclosed.

There is a double side to this story. The side of the man who was changed,
and the side of the woman who prayed. He is a New Englander, by birth and
breeding, now living in this western state: almost a giant physically,
keen mentally, a lawyer, and a natural leader. He had the conviction as a
boy that if he became a Christian he was to preach. But he grew up a
skeptic, read up and lectured on skeptical subjects. He was the
representative of a district of his western home state in congress; in his
fourth term or so I think at this time.

The experience I am telling came during that congress when the
Hayes-Tilden controversy was up, the intensest congress Washington has
known since the Civil War. It was not a time specially suited to
meditation about God in the halls of congress. And further he said to me
that somehow he knew all the other skeptics who were in the lower house
and they drifted together a good bit and strengthened each other by their
talk.

One day as he was in his seat in the lower house, in the midst of the
business of the hour, there came to him a conviction that God--the God in
whom he did not believe, whose existence he could keenly disprove--God was
right there above his head thinking about him, and displeased at the way
he was behaving towards Him. And he said to himself: "this is ridiculous,
absurd. I've been working too hard; confined too closely; my mind is
getting morbid. I'll go out, and get some fresh air, and shake myself."
And so he did. But the conviction only deepened and intensified. Day by
day it grew. And that went on for weeks, into the fourth month as I recall
his words. Then he planned to return home to attend to some business
matters, and to attend to some preliminaries for securing the nomination
for the governorship of his state. And as I understand he was in a fair
way to securing the nomination, so far as one can judge of such matters.
And his party is the dominant party in the state. A nomination for
governor by his party has usually been followed by election.

He reached his home and had hardly gotten there before he found that his
wife and two others had entered into a holy compact of prayer for his
conversion, and had been so praying for some months. Instantly he thought
of his peculiar unwelcome Washington experience, and became intensely
interested. But not wishing them to know of his interest, he asked
carelessly when "this thing began." His wife told him the day. He did some
quick mental figuring, and he said to me, "I knew almost instantly that
the day she named fitted into the calendar with the coming of that
conviction or impression about God's presence."

He was greatly startled. He wanted to be thoroughly honest in all his
thinking. And he said he knew that if a single fact of that sort could be
established, of prayer producing such results, it carried the whole
Christian scheme of belief with it. And he did some stiff fighting within.
Had he been wrong all those years? He sifted the matter back and forth as
a lawyer would the evidence in any case. And he said to me, "As an honest
man I was compelled to admit the facts, and I believe I might have been
led to Christ that very night."

A few nights later he knelt at the altar in the Methodist meeting-house in
his home town and surrendered his strong will to God. Then the early
conviction of his boyhood days came back. He was to preach the gospel. And
like Saul of old, he utterly changed his life, and has been preaching the
gospel with power ever since.

Then I was intensely fascinated in getting the other side, the
praying-side of the story. His wife had been a Christian for years, since
before their marriage. But in some meetings in the home church she was
led into a new, a full surrender to Jesus Christ as Master, and had
experienced a new consciousness of the Holy Spirit's presence and power.
Almost at once came a new intense desire for her husband's conversion. The
compact of three was agreed upon, of daily prayer for him until the change
came.

As she prayed that night after retiring to her sleeping apartment she was
in great distress of mind in thinking and praying for him. She could get
no rest from this intense distress. At length she rose, and knelt by the
bedside to pray. As she was praying and distressed a voice, an exquisitely
quiet inner voice said, "will you abide the consequences?" She was
startled. Such a thing was wholly new to her. She did not know what it
meant. And without paying any attention to it, went on praying. Again came
the same quietly spoken words to her ear, "will you abide the
consequences?" And again the half frightened feeling. She slipped back to
bed to sleep. But sleep did not come. And back again to her knees, and
again the patient, quiet voice.

This time with an earnestness bearing the impress of her agony she said,
"Lord, I will abide any consequence that may come if only my husband may
be brought to Thee." And at once the distress slipped away, and a new
sweet peace filled her being, and sleep quickly came. And while she prayed
on for weeks and months patiently, persistently, day by day, the distress
was gone, the sweet peace remained in the assurance that the result was
surely coming. And so it was coming all those days down in the thick air
of Washington's lower house, and so it did come.

What \textit{was} the consequence to her? She was a congressman's wife. She would
likely have been, so far as such matters may be judged, the wife of the
governor of her state, the first lady socially of the state. She is a
Methodist minister's wife changing her home every few years. A very
different position in many ways. No woman will be indifferent to the
social difference involved. Yet rarely have I met a woman with more of
that fine beauty which the peace of God brings, in her glad face, and in
her winsome smile.

Do you see the simple philosophy of that experience. Her surrender gave
God a clear channel into that man's will. When the roadway was cleared,
her prayer was a spirit-force traversing instantly the hundreds of
intervening miles, and affecting the spirit-atmosphere of his presence.

Shall we not put our wills fully in touch with God, and sheer out of
sympathy with the other one, and persistently plead and claim for each
loved one, "deliver him from the evil one, and work in him Thy will, to
Thy glory, by Thy power, in the Victor's name." And then add amen--so it
\textit{shall} be. Not so \textit{may} it be--a wish, but so it \textit{shall} be--an
expression of confidence in Jesus' power. \textit{And these lives shall be won,
and these souls saved}.




\part{Jesus' Habits of Prayer}


% 1. A Pen Sketch.
% 2. Dissolving Views.
% 3. Deepening Shadows.
% 4. Under the Olive Trees.
% 5. A Composite Picture.

%   \begin{enumerate}
%   \item A Pen Sketch
%   \item Dissolving Views
%   \item Deepening Shadows
%   \item Under the Olive Trees
%   \item A Composite Picture
%   \end{enumerate}


\chapter{Jesus' Habits of Prayer}



\underline{A Pen Sketch.}


When God would win back His prodigal world He sent down a Man. That Man
while more than man insisted upon being truly a man. He touched human life
at every point. No man seems to have understood prayer, and to have prayed
as did He. How can we better conclude these quiet talks on prayer than by
gathering about His person and studying His habits of prayer.

A habit is an act repeated so often as to be done involuntarily; that is,
without a new decision of the mind each time it is done.

Jesus prayed. He loved to pray. Sometimes praying was His way of resting.
He prayed so much and so often that it became a part of His life. It
became to Him like breathing--involuntary.

There is no thing we need so much as to learn how to pray. There are two
ways of receiving instruction; one, by being told; the other, by watching
some one else. The latter is the simpler and the surer way. How better can
we learn how to pray than by watching how Jesus prayed, and then trying
to imitate Him. Not, just now, studying what He \textit{said} about prayer,
invaluable as that is, and so closely interwoven with the other; nor yet
how He received the requests of men when on earth, full of inspiring
suggestion as that is of His \textit{present} attitude towards our prayers; but
how He Himself prayed when down here surrounded by our same circumstances
and temptations.

There are two sections of the Bible to which we at once turn for light,
the gospels and the Psalms. In the gospels is given chiefly the \textit{outer}
side of His prayer-habits; and in certain of the Psalms, glimpses of the
\textit{inner} side are unmistakably revealed.

Turning now to the gospels, we find the picture of the praying Jesus like
an etching, a sketch in black and white, the fewest possible strokes of
the pen, a scratch here, a line there, frequently a single word added by
one writer to the narrative of the others, which gradually bring to view
the outline of a lone figure with upturned face.

Of the fifteen mentions of His praying found in the four gospels, it is
interesting to note that while Matthew gives three, and Mark and John each
four, it is Luke, Paul's companion and mirror-like friend, who, in eleven
such allusions, supplies most of the picture.

Does this not contain a strong hint of the explanation of that other
etching plainly traceable in the epistles which reveals Paul's own
marvellous prayer-life?

Matthew, immersed in the Hebrew Scriptures, writes to the Jews of their
promised Davidic King; Mark, with rapid pen, relates the ceaseless
activity of this wonderful servant of the Father. John, with imprisoned
body, but rare liberty of vision, from the glory-side revealed on Patmos,
depicts the Son of God coming on an errand from the Father into the world,
and again, leaving the world and going back home unto the Father. But Luke
emphasizes the \textit{human} Jesus, a \textit{Man}--with reverence let me use a word in
its old-fashioned meaning--a \textit{fellow}, that is, one of ourselves. And the
Holy Spirit makes it very plain throughout Luke's narrative that the \textit{man}
Christ Jesus \textit{prayed}; prayed \textit{much; needed} to pray; \textit{loved} to pray.

Oh! when shall we men down here, sent into the world as He was sent into
the world, with the same mission, the same field, the same Satan to
combat, the same Holy Spirit to empower, find out that power lies in
keeping closest connection with the Sender, and completest insulation from
the power-absorbing world!



\underline{Dissolving Views.}


Let me rapidily sketch those fifteen mentions of the gospel writers,
attempting to keep their chronological order.

\textit{The first mention} is by Luke, in chapter three. The first three gospels
all tell of Jesus' double baptism, but it is Luke who adds, "and praying."
It was while waiting in prayer that He received the gift of the Holy
Spirit. He \textit{dared} not begin His public mission without that anointing. It
had been promised in the prophetic writings. And now, standing in the
Jordan, He waits and prays until the blue above is burst through by the
gleams of glory-light from the upper-side and the dove-like Spirit wings
down and abides upon Him. \textit{Prayer brings power.} Prayer \textit{is} power. The
time of prayer is the time of power. The place of prayer is the place of
power. Prayer is tightening the connections with the divine dynamo so that
the power may flow freely without loss or interruption.

\textit{The second mention} is made by Mark in chapter one. Luke, in chapter
four, hints at it, "when it was day He came out and went into a desert
place." But Mark tells us plainly "in the morning a great while before the
day (or a little more literally, 'very early while it was yet very dark')
He arose and went out into the desert or solitary place and there prayed."
The day before, a Sabbath day spent in His adopted home-town Capernaum,
had been a very busy day for Him, teaching in the synagogue service, the
interruption by a demon-possessed man, the casting out amid a painful
scene; afterwards the healing of Peter's mother-in-law, and then at
sun-setting the great crowd of diseased and demonized thronging the
narrow street until far into the night, while He, passing amongst them, by
personal touch, healed and restored every one. It was a long and
exhausting day's work. One of us spending as busy a Sabbath would probably
feel that the next morning needed an extra hour's sleep if possible. One
must rest surely. But this man Jesus seemed to have another way of resting
in addition to sleep. Probably He occupied the guest-chamber in Peter's
home. The house was likely astir at the usual hour, and by and by
breakfast was ready, but the Master had not appeared yet, so they waited a
bit. After a while the maid slips to His room door and taps lightly, but
there's no answer; again a little bolder knock, then pushing the door ajar
she finds the room unoccupied. Where's the Master? "Ah!" Peter says; "I
think I know. I have noticed before this that He has a way of slipping off
early in the morning to some quiet place where He can be alone." And a
little knot of disciples with Peter in the lead starts out on a search for
Him, for already a crowd is gathering at the door and filling the street
again, hungry for more. And they "tracked Him down" here and there on the
hillsides, among clumps of trees, until suddenly they come upon Him
quietly praying with a wondrous calm in His great eyes. Listen to Peter as
he eagerly blurts out, "Master, there's a big crowd down there, all asking
for you." But the Master's quiet decisive tones reply, "Let us go into
the next towns that I may preach there also; for to this end came I
forth." Much easier to go back and deal again with the old crowd of
yesterday; harder to meet the new crowds with their new skepticism, but
there's no doubt about what \textit{should} be done. Prayer wonderfully clears
the vision; steadies the nerves; defines duty; stiffens the purpose;
sweetens and strengthens the spirit. The busier the day for Him the more
surely must the morning appointment be kept,[43] and even an earlier start
made, apparently. The more virtue went forth from Him, the more certainly
must He spend time, and even \textit{more} time, alone with Him who is the source
of power.

\textit{The third mention} is in Luke, chapter five. Not a great while after the
scene just described, possibly while on the trip suggested by His answer
to Peter, in some one of the numerous Galilean villages, moved with the
compassion that ever burned His heart, He had healed a badly diseased
leper, who, disregarding His express command, so widely published the fact
of His remarkable healing that great crowds blocked Jesus' way in the
village and compelled Him to go out to the country district, where the
crowds which the village could not hold now throng about Him. Now note
what the Master does. The authorized version says, "He withdrew into the
wilderness and prayed." A more nearly literal reading would be, "He was
retiring in the deserts and praying"; suggesting not a single act, but
rather \textit{a habit of action} running through several days or even weeks.
That is, being compelled by the greatness of the crowds to go into the
deserts or country, districts, and being constantly thronged there by the
people, He had \textit{less opportunity} to get alone, and yet more need, and so
while He patiently continues His work among them He studiously seeks
opportunity to retire at intervals from the crowds to pray.

How much His life was like ours. Pressed by duties, by opportunities for
service, by the great need around us, we are strongly tempted to give less
time to the inner chamber, with door shut. "Surely this work must be
done," we think, "though it does crowd and flurry our prayer time some."
"\textit{No}," the Master's practice here says with intense emphasis. Not work
first, and prayer to bless it. But the \textit{first} place given to prayer and
then the service growing out of such prayer will be charged with
unmeasured power. The greater the outer pressure on His closet-life, the
more jealously He guarded against either a shortening of its time or a
flurrying of its spirit. The tighter the tension, the more time must there
be for unhurried prayer.

\textit{The fourth mention} is found in Luke, chapter six. "It came to pass in
these days that He went out into the mountains to pray, and He continued
all night in prayer to God." The time is probably about the middle of the
second year of His public ministry. He had been having very exasperating
experiences with the national leaders from Judea who dogged His steps,
criticising and nagging at every turn, sowing seeds of skepticism among
His simple-minded, intense-spirited Galileans. It was also the day
\textit{before} He selected the twelve men who were to be the leaders after His
departure, and preached the mountain sermon. Luke does not say that He
\textit{planned} to spend the entire night in prayer. Wearied in spirit by the
ceaseless petty picking and Satanic hatred of His enemies, thinking of the
serious work of the morrow, there was just one thing for Him to do. He
knew where to find rest, and sweet fellowship, and a calming presence, and
wise counsel. Turning His face northward He sought the solitude of the
mountain not far off for quiet meditation and prayer. And as He prayed and
listened and talked without words, daylight gradually grew into twilight,
and that yielded imperceptibly to the brilliant Oriental stars spraying
down their lustrous fire-light. And still He prayed, while the darkness
below and the blue above deepened, and the stilling calm of God wrapped
all nature around, and hushed His heart into a deeper peace. In the
fascination of the Father's loving presence He was utterly lost to the
flight of time, but prayed on and on until, by and by, the earth had once
more completed its daily turn, the gray streaks of dawnlight crept up the
east, and the face of Palestine, fragrant with the deep dews of an
eastern night, was kissed by a sun of a new day. And then, "when it was
day"--how quietly the narrative goes on--"He called the disciples and
\textit{chose} from them twelve,--and a great multitude of disciples and of the
people came,--and He \textit{healed} all--and He opened His mouth and \textit{taught}
them--\textit{for power came forth from Him."} Is it any wonder, after such a
night! If all our exasperations and embarrassments were followed, and all
our decisions and utterances preceded, by unhurried prayer, what power
would come forth from us, too. Because as He is even so are we in this
world.

\textit{The fifth mention} is made by Matthew, chapter fourteen, and Mark,
chapter six, John hinting at it in chapter six of his gospel. It was about
the time of the third passover, the beginning of His last year of service.
Both He and the disciples had been kept exceedingly busy with the great
throng coming and going incessantly. The startling news had just come of
the tragic death of His forerunner. There was need of bodily rest, as well
as of quiet to think over the rapidly culminating opposition. So taking
boat they headed towards the eastern shore of the lake. But the eager
crowds watched the direction taken and spreading the news, literally "ran"
around the head of the lake and "out-went them," and when He stepped from
the boat for the much-needed rest there was an immense company, numbering
thousands, waiting for Him. Did some feeling of impatience break out among
the disciples that they could not be allowed a little leisure? Very
likely, for they were so much like us. But \textit{He} was "moved with
compassion" and, wearied though He was, patiently spent the entire day in
teaching, and then, at eventime when the disciples proposed sending them
away for food, He, with a handful of loaves and fishes, satisfied the
bodily cravings of as many as five thousand.

There is nothing that has so appealed to the masses in all countries and
all centuries as ability to furnish plenty to eat. Literally tens of
thousands of the human race fall asleep every night hungry. So here. At
once it is proposed by a great popular uprising, under the leadership of
this wonderful man as king, to throw off the oppressive Roman yoke.
Certainly if only His consent could be had it would be immensely
successful, they thought. Does this not rank with Satan's suggestion in
the wilderness, and with the later possibility coming through the visit of
the Greek deputation, of establishing the kingdom without suffering? It
was a temptation, even though it found no response within Him. With the
over-awing power of His presence so markedly felt at times He quieted the
movement, "constrained"[44] the disciples to go by boat before Him to the
other side while He dismissed the throng. "And after He had \textit{taken leave
of them}"--what gentle courtesy and tenderness mingled with His
irrevocable decision--"He went up in the mountain \textit{to pray}," and
"\textit{continued in prayer}" until the morning watch. A second night spent in
prayer! Bodily weary, His spirit startled by an event which vividly
foreshadowed His own approaching violent death, and now this vigorous
renewal of His old temptation, again He had recourse to His one unfailing
habit of getting off alone \textit{to pray.} Time alone \textit{to pray; more} time to
pray, was His one invariable offset to all difficulties, all temptations,
and all needs. How much more there must have been in prayer as He
understood and practiced it than many of His disciples to-day know.



\underline{Deepening Shadows.}


We shall perhaps understand better some of the remaining prayer incidents
if we remember that Jesus is now in the last year of His ministry, the
acute state of His experiences with the national leaders preceding the
final break. The awful shadow of the cross grows deeper and darker across
His path. The hatred of the opposition leader gets constantly intenser.
The conditions of discipleship are more sharply put. The inability of the
crowds, of the disciples, and others to understand Him grows more marked.
Many followers go back. He seeks to get more time for intercourse with
the twelve. He makes frequent trips to distant points on the border of the
outside, non-Jewish world. The coming scenes and experiences--\textit{the} scene
on the little hillock outside the Jerusalem wall--seem never absent from
His thoughts. \textit{The sixth mention} is made by Luke, chapter nine. They are
up north in the neighbourhood of the Roman city of Cæsarea Philippi. "And
it came to pass as He was praying alone, the disciples were with Him."
Alone, so far as the multitudes are concerned, but seeming to be drawing
these twelve nearer to His inner life. Some of these later incidents seem
to suggest that he was trying to woo them into something of the same love
for the fascination of secret prayer that He had. How much they would need
to pray in the coming years when He was gone. Possibly, too, He yearned
for a closer fellowship with them. He loved human fellowship, as Peter and
James and John, and Mary and Martha and many other gentle women well knew.
And there is no fellowship among men to be compared with fellowship \textit{in
prayer}.

    "There is a place where \textit{spirits blend},
    Where \textit{friend holds fellowship with friend},
    A place than all beside more sweet,
    It is the blood-bought mercy-seat."

\textit{The seventh mention} is in this same ninth chapter of Luke, and records a
third night of prayer. Matthew and Mark also tell of the transfiguration
scene, but it is Luke who explains that He went up into the mountain \textit{to
pray}, and that it was \textit{as He was praying} that the fashion of His
countenance was altered. Without stopping to study the purpose of this
marvellous manifestation of His divine glory to the chosen three at a time
when desertion and hatred were so marked, it is enough now to note the
significant fact that it was while \textit{He was praying} that the wondrous
change came. \textit{Transfigured while praying!} And by His side stood one who
centuries before on the earth had spent so much time alone with God that
the glory-light of that presence transfigured \textit{his} face, though he was
unconscious of it. A shining face caused by contact with God! Shall not
we, to whom the Master has said, "follow Me," get alone with Him and His
blessed Word, so habitually, with open or uncovered face, that is, with
eyesight unhindered by prejudice or self-seeking, that mirroring the glory
of His face we shall more and more come to bear His very likeness upon our
faces?[45]

    "And the face shines bright
    With a glow of light
    From His presence sent
    Whom she loves to meet.

    "Yes, the face beams bright
    With an inner light
    As by day so by night,
    In shade as in shine,
    With a beauty fine,
    That she wist not of,
    From some source within.
       And above.

    "Still the face shines bright
    With the glory-light
    From the mountain height.
    Where the resplendent sight
    Of His face
    Fills her view
    And illumines in turn
    First the few,
    Then the wide race."

\textit{The eighth mention} is in the tenth chapter of Luke. He had organized a
band of men, sending them out in two's into the places he expected to
visit. They had returned with a joyful report of the power attending their
work; and standing in their midst, His own heart overflowing with joy, He
looked up and, as though the Father's face was visible, spake out to Him
the gladness of His heart. He seemed to be always conscious of His
Father's presence, and the most natural thing was to speak to Him. They
were always within speaking distance of each other, and always on speaking
terms.

\textit{The ninth mention} is in the eleventh chapter of Luke, very similar to
the sixth mention, "It came to pass as He was praying in a certain place
that when He ceased one of His disciples said unto Him, 'Lord, teach us
to pray.'" Without doubt these disciples were praying men. He had already
talked to them a great deal about prayer. But as they noticed how large a
place prayer had in His life, and some of the marvellous results, the fact
came home to them with great force that there must be some fascination,
some power, some secret in prayer, of which \textit{they were ignorant.} This Man
was a master in the fine art of prayer. \textit{They} really did not know how to
pray, they thought. How their request must have delighted Him! At last
they were being aroused concerning \textit{the} great secret of power. May it be
that this simple recital of His habits of prayer may move every one of us
to get alone with Him and make the same earnest request. For the first
step in \textit{learning} to pray is to pray,--"Lord, teach me to pray." And who
\textit{can} teach like Him?

\textit{The tenth mention} is found in John, chapter eleven, and is the second of
the four instances of ejaculatory prayer. A large company is gathered
outside the village of Bethany, around a tomb in which four days before
the body of a young man had been laid away. There is Mary, still weeping,
and Martha, always keenly alive to the proprieties, trying to be more
composed, and their personal friends, and the villagers, and the company
of acquaintances and others from Jerusalem. At His word, after some
hesitation, the stone at the mouth of the tomb is rolled aside. And Jesus
lifted up His eyes and said, "Father, I thank Thee that Thou heardest Me;
and I knew that Thou hearest Me always; but because of the multitude that
standeth around I said it that they may believe that Thou didst send Me!"
Clearly before coming to the tomb He had been praying in secret about the
raising of Lazarus, and what followed was in answer to His prayer. How
plain it becomes that all the marvellous power displayed in His brief
earthly career \textit{came through prayer}. What inseparable intimacy between
His life of activity at which the multitude then and ever since has
marvelled, and His hidden closet-life of which only these passing glimpses
are obtained. Surely the greatest power entrusted to man is prayer-power.
But how many of us are untrue to the trust, while this strangely
omnipotent power put into our hands lies so largely unused.

Note also the certainty of His faith in the Hearer of prayer: "I thank
Thee that Thou heardest Me." There was nothing that could be \textit{seen} to
warrant such faith. There lay the dead body. But He trusted as \textit{seeing}
Him who is \textit{invisible}. Faith is blind, except upward. It is blind to
impossibilities and deaf to doubt. It listens only to God and sees only
His power and acts accordingly. Faith is not believing that He \textit{can} but
that He \textit{will}. But such faith comes only of close continuous contact with
God. Its birthplace is in the secret closet; and time and the open Word,
and an awakened ear and a reverent quiet heart are necessary to its
growth.

\textit{The eleventh mention} is found in the twelfth chapter of John. Two or
three days before the fated Friday some Greek visitors to the Jewish feast
of Passover sought an interview with Him. The request seemed to bring to
His mind a vision of the great outside world, after which His heart
yearned, coming to Him so hungry for what only He could give. And
instantly athwart that vision like an ink-black shadow came the other
vision, never absent now from His waking thoughts, \textit{of the cross} so
awfully near. Shrinking in horror from the second vision, yet knowing that
only through its realization could be realized the first,--seemingly
forgetful for the moment of the by-standers, as though soliloquizing, He
speaks--"now is My soul troubled; and what shall I say? Shall I say,
Father \textit{save} Me from this hour? But for this cause came I unto this hour:
\textit{this} is what I will say (and the intense conflict of soul merges into
the complete victory of a wholly surrendered will) \textit{Father, glorify Thy
name}." Quick as the prayer was uttered, came the audible voice out of
heaven answering, "I have both glorified it and will glorify it again."
How near heaven must be! How quickly the Father hears! He must be bending
over, intently listening, eager to catch even faintly whispered prayer.
Their ears, full of earth-sounds, unaccustomed to listening to a heavenly
voice, could hear nothing intelligible. He had a \textit{trained ear}. Isaiah
50:4 revised (a passage plainly prophetic of Him), suggests how it was
that He could understand this voice so easily and quickly. "He wakeneth
morning by morning, He wakeneth mine ear to hear as they that are taught."
A taught ear is as necessary to prayer as a taught tongue, and the daily
morning appointment with God seems essential to both.



\underline{Under the Olive Trees.}


\textit{The twelfth mention} is made by Luke, chapter twenty-two. It is Thursday
night of Passion week, in the large upper room in Jerusalem where He is
celebrating the old Passover feast, and initiating the new memorial feast.
But even that hallowed hour is disturbed by the disciples' self-seeking
disputes. With the great patience of great love He gives them the
wonderful example of humility of which John thirteen tells, speaking
gently of what it meant, and then turning to Peter, and using his old
name, He says, "Simon, Simon, behold Satan asked to have you that he might
sift you as wheat, but I made supplication for thee that thy faith fail
not." \textit{He had been praying for Peter by name!} That was one of His
prayer-habits, praying for others. And He has not broken off that blessed
habit yet. He is able to save to the uttermost them that draw near to God
through Him \textit{seeing He ever liveth to make intercession for them}. His
occupation now seated at His Father's right hand in glory is \textit{praying for
each of us} who trust Him. By name? Why not?

\textit{The thirteenth mention} is the familiar one in John, chapter seventeen,
and cannot be studied within these narrow limits, but merely fitted into
Us order. The twelfth chapter contains His last words to the world. In the
thirteenth and through to the close of this seventeenth He is alone with
His disciples. If this prayer is read carefully in the revised version it
will be seen that its standpoint is that of one who thinks of His work
down in the world as already done (though the chief scene is yet to come)
and the world left behind, and now He is about re-entering His Father's
presence to be re-instated in glory there. It is really, therefore, a sort
of specimen of the praying for us in which He is \textit{now} engaged, and so is
commonly called the intercessory or high-priestly prayer. For thirty years
He lived a perfect life. For three and a half years He was a prophet
speaking to men for God. For nineteen centuries He has been high priest
speaking to God for men. When He returns it will be as King to reign over
men for God.

\textit{The fourteenth mention} brings us within the sadly sacred precincts of
Gethsemane garden, one of His favourite prayer-spots, where He frequently
went while in Jerusalem. The record is found in Matthew twenty-six, Mark
fourteen, and Luke twenty-one. Let us approach with hearts hushed and
heads bared and bowed, for this is indeed hallowed ground. It is a little
later on that same Thursday night, into which so much has already been
pressed and so much more is yet to come. After the talk in the upper room,
and the simple wondrous prayer, He leads the little band out of the city
gate on the east across the swift, muddy Kidron into the inclosed grove of
olive trees beyond. There would be no sleep for Him that night. Within an
hour or two the Roman soldiers and the Jewish mob, led by the traitor,
will be there searching for Him, and He meant to spend the intervening
time in \textit{prayer}. With the longing for sympathy so marked during these
latter months, He takes Peter and James and John and goes farther into the
deeply-shadowed grove. But now some invisible power tears him away and
plunges Him alone still farther into the moonlit recesses of the garden;
and there a strange, awful struggle of soul ensues. It seems like a
renewal of the same conflict He experienced in John twelve when the Greeks
came, but immeasurably intenser. He who in Himself knew no sin was now
beginning to realize in His spirit what within a few hours He realized
\textit{actually}, that He was in very deed to be made sin for us. And the awful
realization comes in upon Him with such terrific intensity that it seems
as though His physical frame cannot endure the strain of mental agony. The
\textit{actual} experience of the next day produced such mental agony that His
physical strength gave way. For He died not of His physical suffering,
excruciating as that was, but literally of a broken heart, its walls burst
asunder by the strain of soul. It is not possible for a sinning soul to
appreciate with what nightmare dread and horror the sinless soul of Jesus
must have approached the coming contact with the sin of a world. With
bated breath and reverent gaze one follows that lonely figure among the
trees; now kneeling, now falling upon His face, lying prostrate, "He
prayed that \textit{if} it were possible the hour might pass away from Him." One
snatch of that prayer reaches our ears: "Abba, Father, all things are
possible unto Thee--\textit{if} it be possible let this cup pass away from Me;
nevertheless not as I will, but as Thou wilt." How long He remained so in
prayer we do not know, but so great was the tension of spirit that a
messenger from heaven appeared and strengthened Him. Even after that
"being in an agony He prayed more earnestly (literally, more stretched
out, more strainedly) and His sweat became as it were great clots of blood
falling down upon the ground." When at length He arises from that season
of conflict and prayer, the victory seems to be won, and something of the
old-time calm reasserts itself. He goes to the sleeping disciples, and
mindful of their coming temptation, admonishes them to pray; then returns
to the lonely solitude again for more prayer, but the change in the form
of prayer tells of the triumph of soul, "O My Father, if this cup
\textit{cannot} pass away except I drink it, Thy will be done." The victory is
complete. The crisis is past. He yields Himself to that dreaded experience
through which alone the Father's loving plan for a dying world can be
accomplished. Again He returns to the poor, weak disciples, and back again
for another bit of strengthening communion, and then the flickering glare
of torches in the distance tells Him that "the hour is come." With steady
step and a marvellous peace lighting His face He goes out to meet His
enemies. He overcame in this greatest crisis of His life \textit{by prayer}.

\textit{The fifteenth mention} is the final one. Of the seven sentences which He
spake upon the cross, three were prayers. Luke tells us that while the
soldiers were driving the nails through His hands and feet and lifting the
cross into place, He, thinking even then not of self, but of others, said,
"Father, forgive them, they know not what they do."

It was as the time of the daily evening sacrifice drew on, near the close
of that strange darkness which overcast all nature, after a silence of
three hours, that He loudly sobbed out the piercing, heart-rending cry,
"My God, My God, why didst Thou forsake Me?" A little later the triumphant
shout proclaimed His work done, and then the very last word was a prayer
quietly breathed out, as He yielded up His life, "Father, into Thy hands
I commend My spirit." And so His expiring breath was vocalized into
prayer.



\underline{A Composite Picture.}


It may be helpful to make the following summary of these allusions.

1. \textit{His times of prayer}: His regular habit seems plainly to have been to
devote the early morning hour to communion with His Father, and to depend
upon that for constant guidance and instruction. This is suggested
especially by Mark 1:35; and also by Isaiah 50:4-6 coupled with John 7:16
l.c., 8:28, and 12:49.

In addition to this regular appointment, He sought other opportunities for
secret prayer as special need arose; late at night after others had
retired; three times He remained in prayer all the night; and at irregular
intervals between times. Note that it was usually a \textit{quiet} time when the
noises of earth were hushed. He spent special time in prayer \textit{before}
important events and also \textit{afterwards}. (See mentions 1, 2, 3, 4, 5, 10
and 14.)

2. \textit{His places of prayer}: He who said, "Enter into thine inner chamber
and when thou hast shut the door, pray to thy Father in secret," Himself
had no fixed inner chamber, during His public career, to make easier the
habitual retirement for prayer. Homeless for the three and a half years of
ceaseless travelling, His place of prayer was a desert place, "the
deserts," "the mountains," "a solitary place." He loved nature. The
hilltop back of Nazareth village, the slopes of Olivet, the hillsides
overlooking the Galilean lake, were His favourite places. Note that it was
always a \textit{quiet} place, shut away from the discordant sounds of earth.

3. \textit{His constant spirit of prayer}: He was never out of the spirit of
prayer. He could be alone in a dense crowd. It has been said that there
are sorts of solitude, namely, of time, as early morning, or late at
night; solitude of place, as a hilltop, or forest, or a secluded room; and
solitude of spirit, as when one surrounded by a crowd may watch them
unmoved, or to be lost to all around in his own inner thought. Jesus used
all three sorts of solitude for talking with His Father. (See mentions 8,
10, 11 and 15.)

4. \textit{He prayed in the great crises of His life}: Five such are mentioned:
Before the awful battle royal with Satan in the Quarantanian wilderness at
the outset; before choosing the twelve leaders of the new movement; at the
time of the Galilean uprising; before the final departure from Galilee for
Judea and Jerusalem; and in Gethsemane, the greatest crisis of all. (See
mentions 1, 4, 5, 7 and 14.)

5. He prayed for others by name, and still does. (See mention 13.)

6. \textit{He prayed with others}: A habit that might well be more widely copied.
A few minutes spent in quiet prayer by friends or fellow-workers before
parting wonderfully sweetens the spirit, and cements friendships, and
makes difficulties less difficult, and hard problems easier of solution.
(See mentions 7, 9 and 13.)

7. \textit{The greatest blessings of His life came during prayer}: Six incidents
are noted: while praying, the Holy Spirit came upon Him; He was
transfigured; three times a heavenly voice of approval came; and in His
hour of sorest distress in the garden a heavenly messenger came to
strengthen Him. (See mentions 1, 7, 11 and 14.)

How much prayer meant to Jesus! It was not only His \textit{regular habit}, but
His resort in \textit{every emergency}, however slight or serious. When perplexed
He \textit{prayed}. When hard pressed by work He \textit{prayed}. When hungry for
fellowship He found it in \textit{prayer}. He chose His associates and received
His messages \textit{upon His knees}. If tempted, He \textit{prayed}. If criticised, He
\textit{prayed}. If fatigued in body or wearied in spirit, He had recourse to His
one unfailing habit of \textit{prayer. Prayer} brought Him \textit{unmeasured power} at
the beginning, and \textit{kept} the flow unbroken and undiminished. There was no
emergency, no difficulty, no necessity, no temptation that would not yield
to prayer, as He practiced it. Shall not we, who have been tracing these
steps in His prayer life, go back over them again and again until we
breathe in His very spirit of prayer? And shall we not, too, ask Him daily
to teach us how to pray, and then plan to get alone with Him regularly
that He may have opportunity to teach us, and we the opportunity to
practice His teaching?


\chapter{Footnotes}

[1] John 15:16.

[2] "Demon Possession," by J. L. Nevius.

[3] Psalm 24:1.

[4] Psalm 29:10.

[5] Genesis 1:26, 28. Psalms 8:6. See quotations of this, referring to the
Man who will restore original conditions, in 1 Cor. 15:27. Ephesians 1:32,
Hebrews 2:8. Psalms 115:16.

[6] John 12:31; 14:30; 16:11.

[7] Revelation 11:15.

[8] John 14:30.

[9] Jeremiah 33:3.

[10] Psalm 50:15.

[11] Matthew 7:7.

[12] Isaiah 1:15.

[13] Isaiah 59:1-3.

[14] Psalm 66:18.

[15] James 4:2, 3.

[16] Matthew 5:23, 24.

[17] Matthew 6:9-15.

[18] Matthew 18:19-35.

[19] Acts 16:6.

[20] Acts 16:7.

[21] John 7:8.

[22] Acts 22:17-21.

[23] 2 Cor. 5:21.

[24] Sidney Lanier.

[25] Ephesians 2:2.

[26] Luke 11:5-13.

[27] Luke 18:1-8.

[28] 1 Peter 5:8.

[29] Matthew 17:14-20; Mark 9:14-29; Luke 9:37-43.

[30] Matthew 16:24.

[31] Psalm 37:7.

[32] Isaiah 50:4.

[33] Jeremiah 15:1.

[34] Longfellow.

[35] 2 Samuel 23:9, 10.

[36] Joseph Cook.

[37] John 7:17.

[38] Frances Ridley Havergal.

[39] Romans 8:26-28.

[40] Psalm 25:9.

[41] 1 John 5:14, 15.

[42] 2 Timothy 2:24-26.

[43] Isaiah 50:4, Revised.

[44] Does not this very strong language suggest that possibly the
disciples had been conferred with by the revolutionary leaders?

[45] 2 Cor. 3:18.







